% Generated by GrindEQ Word-to-LaTeX 2012 
% ========== UNREGISTERED! ========== Please register! ==========
% LaTeX/AMS-LaTeX

\documentclass{article}

%%% remove comment delimiter ('%') and specify encoding parameter if required,
%%% see TeX documentation for additional info (cp1252-Western,cp1251-Cyrillic)
%\usepackage[cp1252]{inputenc}

%%% remove comment delimiter ('%') and select language if required
%\usepackage[english,spanish]{babel}

\usepackage{amssymb}
\usepackage{amsmath}

%%% remove comment delimiter ('%') and select graphics package
%%% for DVI output:
\usepackage[dvips]{graphicx}
%%% or for PDF output:
%\usepackage[pdftex]{graphicx}
%%% or for old LaTeX compilers:
%\usepackage[dvips]{graphics}

\begin{document}

%%% remove comment delimiter ('%') and select language if required
%\selectlanguage{spanish} 

\noindent 

\noindent 

\noindent 

\noindent 

\noindent \includegraphics*[width=1.34in, height=0.38in, keepaspectratio=false]{image1}

\noindent 

\noindent 

\noindent 

\noindent 

\noindent 

\noindent 

\noindent 

\noindent 

\noindent 

\noindent 

\noindent 

\noindent 

\noindent 

\noindent 

\noindent 

\begin{tabular}{|p{3.2in}|} \hline 
\textbf{XBeach Manual\newline \newline } \\ \hline 
\end{tabular}



\noindent 

\noindent 

\noindent 

\begin{tabular}{|p{3.9in}|} \hline 
 \\ \hline 
\end{tabular}



\noindent 

\noindent 

\noindent 

\begin{tabular}{|p{3.9in}|} \hline 
\copyright  Deltares, 2015, B \\ \hline 
\end{tabular}



\noindent 

\noindent 
\[1\] 


\noindent 

\noindent 

\noindent 

\noindent 

\noindent 

\noindent 

\noindent 

\noindent 

\noindent 

\noindent 

\noindent 

\noindent 

\noindent 

\noindent 

\noindent 

\begin{tabular}{|p{2.5in}|} \hline 
\textbf{XBeach Manual\newline \newline } \\ \hline 
\end{tabular}



\noindent 

\noindent 

\noindent 

\noindent 

\noindent 

\noindent 

\noindent 

\noindent 

\noindent 

\noindent 

\noindent 

\noindent 

\begin{tabular}{|p{1.3in}|p{0.6in}|p{0.6in}|p{1.2in}|p{0.4in}|} \hline 
\multicolumn{2}{|p{1in}|}{} & \multicolumn{3}{|p{2.2in}|}{\textbf{Title\newline }XBeach Manual} \\ \hline 
\multicolumn{3}{|p{1in}|}{} & \multicolumn{2}{|p{1.6in}|}{} \\ \hline 
\textbf{Pages\newline }8 & \multicolumn{2}{|p{1.2in}|}{} &  &  \\ \hline 
\multicolumn{5}{|p{1in}|}{} \\ \hline 
\end{tabular}

\includegraphics*[width=1.34in, height=0.38in, keepaspectratio=false]{image2}

\noindent \includegraphics*[width=1.34in, height=0.38in, keepaspectratio=false]{image3}

\noindent 

\noindent 

\noindent 

\begin{tabular}{|p{1.3in}|p{1.2in}|p{1.2in}|p{0.4in}|} \hline 
\multicolumn{4}{|p{1in}|}{\textbf{Title\newline }XBeach Manual} \\ \hline 
\multicolumn{2}{|p{1in}|}{} & \multicolumn{2}{|p{1.6in}|}{} \\ \hline 
\textbf{Pages\newline }135 &  &  &  \\ \hline 
\multicolumn{4}{|p{1in}|}{} \\ \hline 
\end{tabular}



\noindent 

\begin{tabular}{|p{3.6in}|p{0.4in}|} \hline 
\textit{XBeach Manual} &  \\ \hline 
\end{tabular}



\noindent 

\begin{tabular}{|p{3.6in}|p{0.4in}|} \hline 
\textit{XBeach Manual} &  \\ \hline 
\end{tabular}


\[1\] 

\noindent 

\noindent 

\noindent 

\begin{tabular}{|p{1.2in}|p{1.2in}|p{1.2in}|p{0.5in}|} \hline 
\multicolumn{4}{|p{1in}|}{} \\ \hline 
\multicolumn{2}{|p{1in}|}{} & \multicolumn{2}{|p{1.7in}|}{} \\ \hline 
 &  &  &  \\ \hline 
\end{tabular}



\noindent 

\begin{tabular}{|p{3.6in}|p{0.4in}|} \hline 
\textit{} &  \\ \hline 
\end{tabular}



\noindent \textbf{Keywords}

\noindent Place keywords here

\noindent 

\noindent \textbf{Summary}

\noindent Place summary here

\noindent 

\noindent \textbf{References}

\noindent Place references here

\noindent 

\noindent 

\noindent 

\noindent 

\noindent 

\begin{tabular}{|p{0.3in}|p{0.5in}|p{0.7in}|p{0.3in}|p{0.7in}|p{0.3in}|p{0.7in}|p{0.3in}|} \hline 
\textbf{Version} & \textbf{Date} & \textbf{Author} & \textbf{Initials} & \textbf{Review} & \textbf{Initials} & \textbf{Approval} & \textbf{Initials} \\ \hline 
 & feb. 2015 &  &  &  &  &  &  \\ \hline 
 &  &  &  &  &  &  &  \\ \hline 
 &  &  &  &  &  &  &  \\ \hline 
\end{tabular}



\begin{tabular}{|p{3.9in}|} \hline 
\textbf{State\newline }draft\newline This is a draft report, intended for discussion purposes only. No part of this report may be relied upon by either principals or third parties. \\ \hline 
\end{tabular}



\noindent 

\noindent 

\noindent \includegraphics*[width=1.34in, height=0.38in, keepaspectratio=false]{image4}

\noindent 

\noindent 

\noindent 

\begin{tabular}{|p{2.6in}|} \hline 
\textit{27 January 2015, draft} \\ \hline 
\end{tabular}

\includegraphics*[width=1.34in, height=0.38in, keepaspectratio=false]{image5}

\noindent 

\begin{tabular}{|p{0.4in}|} \hline 
ii \\ \hline 
\end{tabular}



\noindent 

\begin{tabular}{|p{3.5in}|} \hline 
\textit{XBeach Manual} \\ \hline 
\end{tabular}


\[1\] \textbf{Contents}

\noindent \textbf{1 Introduction 12 Processes and model formulation 22.1 Domain and definitions 22.1.1 Coordinate system 22.1.2 Grid set-up 22.2 Hydrodynamics options 32.2.1 Stationary mode 42.2.2 Surf beat mode (instationary) 52.2.3 Non-hydrostatic mode (wave resolving) 72.3 Short wave action 92.3.1 Short wave action balance 92.3.2 Dissipation 102.3.3 Radiation stresses 152.3.4 Wave shape 152.3.5 Turbulence 172.3.6 Roller energy balance 182.4 Shallow water equations 202.4.1 Horizontal viscosity 202.4.2 Bed shear stress 212.4.3 Damping by vegetation 222.4.4 Wind 222.5 Non-hydrostatic pressure correction 232.6 Groundwater flow 242.6.1 Continuity 242.6.2 Equation of motions 242.7 Sediment transport 282.7.1 Advection-diffusion equation 282.7.2 General parameters 292.7.3 Transport formulations 302.7.4 Effects of wave nonlinearity 322.7.5 Hindered erosion by dilatancy 332.7.6 Bed slope effect 332.8 Bottom updating 342.8.1 Due to sediment fluxes 342.8.2 Avalanching 352.8.3 Bed composition 362.9 Ship-induced wave motions 373 Boundary conditions 393.1 Waves 393.1.1 Spectral conditions 393.1.2 Non-spectral conditions 393.1.3 Special conditions 403.1.4 Lateral boundary conditions 403.2 Shallow water equations 413.2.1 Offshore boundary 413.2.2 Lateral boundaries 423.2.3 Tide and surge 433.2.4 River and point discharge 433.3 Sediment transport 434 Input description 444.1 General 444.2 Physical processes 454.3 Grid and bathymetry 464.4 Waves input 494.4.1 Spectral wave boundary conditions 524.4.2 Non-spectral wave boundary conditions 594.4.3 Special types of wave boundary conditions 614.4.4 Temporally and/or spatially varying wave boundary conditions 614.5 Flow, tide and surge input 624.5.1 Flow boundary conditions 624.5.2 Time-varying tide/surge 644.6 Water level (dam break) 654.7 Wind input 674.8 Sediment input 674.9 Vegetation input 704.10 Discharge input 714.11 Drifters input 734.12 Ship-induced wave motions 734.13 Output selection 754.13.1 Output types 774.13.2 Output times 854.13.3 Output format 864.14 Time parameters 885 Bibliography 896 Appendices 896.1 Hands on exercises (based on basic XBeach of the Delft Software Days 2014) 946.1.1 Dune erosion at Delfland, Netherlands (1D) 946.1.2 Nourishment scenarios near Kijkduin, Holland (1D) 956.1.3 Overwash at Santa Rosa Island , USA (2DH) 966.1.4 Yanchep perched beach and natural breakwater (2DH) 966.2 Advanced model coefficients 986.2.1 Wave numerics 986.2.2 Wave dissipation 986.2.3 Rollers 1006.2.4 Wave-current interaction 1006.2.5 Bed friction and viscosity 1006.2.6 Flow numerics 1016.2.7 Sediment transport 1026.2.8 Sediment transport numerics 1066.2.9 Quasi-3D sediment transport 1066.2.10 Morphology 1076.2.11 Bed update 1086.2.12 Groundwater flow 1106.2.13 Non-hydrostatic correction 1126.2.14 Physical constants 1136.2.15 Coriolis force 1136.2.16 MPI  1146.2.17 Output projection 1146.3 Numerical implementation 1166.3.1 Grid set-up 1166.3.2 Wave action balance 1166.3.3 Shallow water equations 1176.3.4 Groundwater flow 1206.3.5 Sediment transport 1266.3.6 Bottom updating schemes 1276.3.7 Boundary conditions 1316.3.8 Non-hydrostatic 132}

\noindent 

\noindent 

\noindent 

\noindent \includegraphics*[width=1.34in, height=0.38in, keepaspectratio=false]{image6}

\noindent 

\noindent 

\noindent 

\noindent \includegraphics*[width=1.34in, height=0.38in, keepaspectratio=false]{image7}

\begin{tabular}{|p{2.7in}|} \hline 
\textit{27 January 2015, draft} \\ \hline 
\end{tabular}



\noindent 

\begin{tabular}{|p{3.5in}|} \hline 
\textit{XBeach Manual} \\ \hline 
\end{tabular}



\noindent 

\begin{tabular}{|p{3.5in}|} \hline 
\textit{XBeach Manual} \\ \hline 
\end{tabular}




\section{  Introduction}

\noindent XBeach is an open-source numerical model which is originally developed to simulate hydrodynamic and morphodynamic processes and impacts on sandy coasts with a domain size of kilometers and on the time scale of storms. Since then, the model has been applied to other types of coasts and purposes.

\noindent 

\noindent The model includes the hydrodynamic processes of short wave transformation (refraction, shoaling and breaking), long wave (infragravity wave) transformation (generation, propagation and dissipation), wave-induced setup and unsteady currents, as well as overwash and inundation. The morphodynamic processes include bed load and suspended sediment transport, dune face avalanching, bed update and breaching. Effects of vegetation and of hard structures have been included. The model has been validated with a series of analytical, laboratory and field test cases using a standard set of parameter settings. 

\noindent 

\noindent XBeach has two modes: a hydrostatic and a non-hydrostatic mode. In the hydrostatic mode, the short wave amplitude variation is solved separately from the long waves, currents and morphological change. This saves considerable computational time, with the expense that the phase of the short waves is not simulated. A more complete model is the non-hydrostatic model which solves all processes including short wave motions, but with more computational demand.  

\noindent 

\noindent The original application, funded by the U.S. Corps of Engineers in the framework of the Morphos project and the U.S. Geological Survey, was to be able to assess hurricane impacts on sandy beaches. Since then with funding from the Dutch Public Works Department, the model has been extended, applied and validated for storm impacts on dune and urbanized coasts for the purpose of dune safety assessments. With support from the European Commission XBeach has been validated on a number of dissipative and reflective beaches bordering all regional seas in the EU. 

\noindent 

\noindent Beyond sandy coasts, the model has been applied to coral fringing and atoll reefs, in cooperation with and with funding by the University of Western Australia, the USGS and the Asian Development Bank. The model now also includes vegetative damping effects, with support of the U.S. Office of Naval Research.

\noindent 

\noindent The non-hydrostatic model has been developed initially by the TU Delft (as a prototype version of the SWASH model). For the purpose of simulating the morphodynamic processes on gravel beaches, the model was extended and validated with support from the University of Plymouth. In this mode, ship-induced waves can be simulated as well, demonstrating the flight that the model has taken since its first inception.

\noindent 

\noindent This development of XBeach could not have been possible without all of the above mentioned funding agencies and partners. It would also not have been possible without the enthusiastic, critical and constructive approach of all consultants, researchers, M.Sc. and Ph.D. students who have taken up XBeach, and made it into the tool that it is today.

\noindent 

\noindent This manual serves as an introduction to the model and a reference guide to its many functionalities, options and parameters. We sincerely hope that this document will help existing and new researchers apply the model for their purposes and advance our knowledge of coastal hydro- and morphodynamics.\eject 


\section{ Processes and model formulation}


\subsection{ Domain and definitions}


\subsubsection{ Coordinate system}

\noindent XBeach uses a coordinate system where the computational x-axis is always oriented towards the coast, approximately perpendicular to the coastline, and the y-axis is alongshore, see Figure 2.1 and Figure 2.2. This coordinate system is defined in world coordinates. The grid size in x- and y-direction may be variable but the grid must be curvilinear. Alternatively, in case of a rectangular grid (a special case of a curvilinear grid) the user can provide coordinates in a local coordinate system that is oriented with respect to world coordinates (xw, yw) through an origin (xori, yori) and an orientation (alfa) as depicted in Figure 2.1. The orientation is defined counter-clockwise w.r.t. the xw-axis (East). 

\noindent 

\noindent \includegraphics*[width=4.61in, height=3.78in, keepaspectratio=false, trim=2.05in 1.98in 0.00in 2.00in]{image8}

\noindent \textit{Figure 2.1 Rectangular coordinate system of XBeach}


\subsubsection{ Grid set-up}

\noindent The grid applied is a staggered grid, where the bed levels, water levels, water depths and concentrations are defined in cell centers, and velocities and sediment transports are defined in u- and v-points, viz. at the cell interfaces. In the wave energy balance, the energy, roller energy and radiation stress are defined at the cell centers, whereas the radiation stress gradients are defined at u- and v-points.

\noindent 

\noindent Velocities at the u- and v-points are denoted by the output variables uu and vv respectively; velocities u and v at the cell centers are obtained by interpolation and are for output purpose only. The water level, zs, and the bed level, zb, are both defined positive upward. uv and vu are the u-velocity at the v-grid point and the v-velocity at the u-grid point respectively. These are obtained by interpolation of the values of the velocities at the four surrounding grid points. 

\noindent 

\noindent The model solves coupled 2D horizontal equations for wave propagation, flow, sediment transport and bottom changes, for varying (spectral) wave and flow boundary conditions.

\noindent 

\noindent 

\noindent \includegraphics*[width=4.76in, height=3.39in, keepaspectratio=false, trim=1.66in 2.97in 0.11in 1.81in]{image9}

\noindent \textit{Figure 2.2 Curvilinear coordinate system of XBeach}


\subsection{ Hydrodynamics options}

\noindent XBeach was originally developed as a short-wave averaged but wave-group resolving model, allowing resolving the short wave variations on the wave group scale and the long waves associated with them. Since the original paper by Roelvink et al. (2009) a number of additional model options have been implemented, thereby allowing users to choose which time-scales to resolve:

\begin{enumerate}
\item  Stationary wave model (keyword: \textit{wavemodel = stationary}), efficiently solving wave-averaged equations but neglecting infragravity waves;

\item  Surfbeat mode (instationary) (keyword: \textit{wavemodel = surfbeat}), where the short wave variations on the wave group scale (short wave envelope) and the long waves associated with them are resolved;

\item  Non-hydrostatic mode (wave-resolving) (keyword: \textit{wavemodel = nonh}), where a combination of the non-linear shallow water equations with a pressure correction term is applied, allowing to model the propagation and decay of individual waves.
\end{enumerate}

\noindent 

\noindent In the following these options are discussed in more detail. Important to note that all times in XBeach are prescribed on input in morphological time. If you apply a morphological acceleration factor (keyword: \textit{morfac}) all input time series and other time parameters are divided internally by \textit{morfac}. This way, you can specify the time series as real times, and vary the \textit{morfac} without changing the rest of the input files (keyword: \textit{morfacopt = 1}).

\noindent 

\noindent \includegraphics*[width=6.04in, height=3.12in, keepaspectratio=false]{image10}

\noindent \textit{Figure 2.3 Principle sketch of the relevant wave processes}


\subsubsection{ Stationary mode}

\noindent In stationary mode the wave-group variations and thereby all infragravity motions are neglected. This is useful for conditions where the incident waves are relatively small and/or short, and these motions would be small anyway. The model equations are similar to HISWA (Holthuijsen et al., 1989) but do not include wave growth or wave period variations. Processes that are resolved are wave propagation, directional spreading, shoaling, refraction, bottom dissipation and wave breaking, and a roller model is included; these processes are usually dominant in nearshore areas of limited extent. For the breaking dissipation we use the Baldock et al. (1998) model, which is valid for wave-averaged modeling. The radiation stress gradients from the wave and roller model force the shallow water equations, drive currents and lead to wave setdown and setup. Additionally, wind and tidal forcing can be applied. 

\noindent The mean return flow due to mass flux and roller is included in the model and affects the sediment transport, leading to an offshore contribution. To balance this, effects of wave asymmetry and skewness are included as well. Bed slope effects can further modify the cross-shore behavior. A limited number of model coefficients allow the user to calibrate the profile shape resulting from these interactions.  

\noindent 

\noindent A typical application would be to model morphological changes during moderate wave conditions, often in combination with tides. The wave boundary conditions can be specified as constant (keyword: \textit{wbctype = stat}) or as a time-series of wave conditions (keyword: \textit{wbctype = stat\_table)}. Typical examples of such model applications are given below for tombolo formation behind an offshore breakwater (left panel) and development of an ebb delta at a tidal inlet (right panel). A big advantage of the stationary XBeach wave model over other models is that the lateral boundaries are entirely without disturbance if the coast is longshore uniform near these boundaries.

\noindent 

\noindent 

\noindent 

\noindent \includegraphics*[width=2.97in, height=3.67in, keepaspectratio=false, trim=0.16in 0.13in 0.24in 0.26in]{image11}\includegraphics*[width=2.96in, height=3.62in, keepaspectratio=false, trim=0.13in 0.16in 0.22in 0.25in]{image12}\includegraphics*[width=2.93in, height=2.26in, keepaspectratio=false, trim=0.19in 0.19in 0.32in 0.13in]{image13}\includegraphics*[width=2.96in, height=2.32in, keepaspectratio=false, trim=0.16in 0.22in 0.32in 0.04in]{image14}

\noindent \textit{Figure 2.4 Root-mean square wave height (left panels) and final bathymetry (right panels) for an offshore breakwater case (upper panels) and a tidal inlet with waves from 330 degrees (lower panels).}


\subsubsection{ Surf beat mode (instationary) }

\noindent The short-wave motion is solved using the wave action equation which is a time-dependent forcing of the HISWA equations (Holthuijsen et al., 1989). This equation solves the variation of short-waves envelope (wave height) on the scale of wave groups. It employs a dissipation model for use with wave groups (Roelvink, 1993a; Daly et al., 2012) and a roller model (Svendsen, 1984; Nairn et al., 1990; Stive and de Vriend, 1994) to represent momentum stored at the surface after breaking. These variations, through radiation stress gradients (Longuet-Higgins and Stewart, 1962, 1964) exert a force on the water column and drive longer period waves (infragravity waves) and unsteady currents, which are solved by the nonlinear shallow water equations (e.g. Phillips, 1977; Svendsen, 2003). Thus, wave-driven currents (longshore current, rip currents and undertow), and wind-driven currents (stationary and uniform) for local wind set-up, long (infragravity) waves, and runup and rundown of long waves (swash) are included. 

\noindent 

\noindent Using the surfbeat mode is necessary when the focus is on swash zone processes rather than time-averaged currents and setup. It is fully valid on dissipative beaches, where the short waves are mostly dissipated by the time they are near the shoreline. On intermediate beaches and during extreme events the swash motions are still predominantly in the infragravity band and so is the runup. 

\noindent 

\noindent Under this surfbeat mode, several options are available, depending on the circumstances:

\begin{enumerate}
\item  \textbf{1D cross-shore; }in this case the longshore gradients are ignored and the domain reduces to a single gridline (keyword: \textit{ny = 0).} Within this mode the following options are available:\textbf{}

\begin{enumerate}
\item \textbf{ }Retaining directional spreading (keyword: \textit{dtheta} $<$ \textit{thetamax -- thetamin}); this has a limited effect on the wave heights because of refraction, but can also allow obliquely incident waves and the resulting longshore currents;

\item  Using a single directional bin (keyword: \textit{dtheta} = \textit{thetamax -- thetamin}); this leads to perpendicular waves always and ignores refraction. If the keyword \textit{snells = 1} is applied, the mean wave direction is determined based on Snell's law. In this case also longshore currents are generated.
\end{enumerate}
\end{enumerate}

\noindent 

\begin{enumerate}
\item  \textbf{2DH area; }the model is solved on a curvilinear staggered grid (rectilinear is a special case). The incoming short wave energy will vary along the seaward boundary and in time, depending on the wave boundary conditions. This variation is propagated into the model domain. Within this mode the following options are available:\textbf{}

\begin{enumerate}
\item \textbf{ }Resolving the wave refraction 'on the fly' using the propagation in wave directional space. For large directional spreading or long distances this can lead to some smoothing of groupiness since the waves from different directions do not interfere but their energy is summed up. This option is possible for arbitrary bathymetry and any wave direction. The user must specify the width of the directional bins for the surfbeat mode (keyword: \textit{dtheta})\textbf{}

\item \textbf{ }Solving the wave direction at regular intervals using the stationary solver, and then propagating the wave energy along the mean wave direction. This preserves the groupiness of the waves therefore leads to more forcing of the infragravity waves (keyword: \textit{single\_dir = 1}). The user must now specify a single directional bin for the instationary mode (\textit{dtheta = thetamax - thetaminn) }and a smaller bin size for the stationary solver (keyword: \textit{dtheta\_s)}.\textbf{}

\item \textbf{ }For schematic, longshore uniform cases the mean wave direction can also be computed using Snell's law (keyword: \textit{snells = 1}). This will then give comparable results to the \textit{single\_dir} option.\textbf{}
\end{enumerate}
\end{enumerate}

\noindent 

\noindent In the figures below some typical applications of 1D and 2D models are shown; a reproduction of a large-scale flume test, showing the ability of XBeach to model both short-wave (HF) and long-wave (LF) wave heights and velocities; and a recent 2DH simulation (Nederhoff et al., 2015) of the impact of hurricane Sandy on Camp Osborne, Brick, NJ.

\noindent 

\noindent 

\noindent \textbf{\textit{\includegraphics*[width=5.92in, height=4.48in, keepaspectratio=false, trim=0.00in 0.08in 0.02in 0.00in]{image15}}}\textit{}

\noindent \textit{Figure 2.5 Computed and observed hydrodynamic parameters for test 2E of the LIP11D experiment. Top left: bed level and mean water level. Top right: measured (dots) and computed}

\noindent 

\noindent 

\noindent \includegraphics*[width=2.95in, height=1.35in, keepaspectratio=false, trim=0.05in 2.26in 0.14in 0.13in]{image16}\includegraphics*[width=2.96in, height=1.38in, keepaspectratio=false, trim=0.04in 0.71in 0.17in 1.68in]{image17}

\noindent 

\noindent \textit{Figure 2.6 Pre (left) and post-Sandy (right) in a three dimensional plot with both bed and water levels as simulated by XBeach (Nederhoff et al. 2015) }


\subsubsection{ Non-hydrostatic mode (wave resolving)  }

\noindent For non-hydrostatic XBeach calculations (keyword: \textit{wavemodel = nonh}) depth-averaged flow due to waves and currents are computed using the non-linear shallow water equations, including a non-hydrostatic pressure. The depth-averaged normalized dynamic pressure (\textit{q}) is derived in a method similar to a one-layer version of the SWASH model (Zijlema et al. 2011). The depth averaged dynamic pressure is computed from the mean of the dynamic pressure at the surface and at the bed by assuming the dynamic pressure at the surface to be zero and a linear change over depth. 

\noindent 

\noindent Under these formulations dispersive behavior is added to the long wave equations and the model can be used as a short-wave resolving model. Wave breaking is implemented by disabling the non-hydrostatic pressure term when waves exceed a certain steepness, after which the bore-like breaking implicit in the momentum-conserving shallow water equations takes over. 

\noindent 

\noindent In case the non-hydrostatic mode is used, the short wave action balance is no longer required. This saves computation time. However, in the wave-resolving mode we need much higher spatial resolution and associated smaller time steps, making this mode much more computationally expensive than the surfbeat mode. 

\noindent 

\noindent The main advantages of the non-hydrostatic mode are that the incident-band (short wave) runup and overwashing are included, which is especially important on steep slopes such as gravel beaches. Another advantage is that the wave asymmetry and skewness are resolved by the model and no approximate local model or empirical formulation is required for these terms. Finally, in cases where diffraction is a dominant process, wave-resolving modeling is needed as it is neglected in the short wave averaged mode. The XBeach-G formulations for gravel beaches (McCall et al, 2014) are based on the non-hydrostatic mode. Although sandy morphology can be simulated using the wave-resolving mode, it has not been extensively validated and it is likely that changes in the sediment transport formulations will be implemented in the near future.

\noindent 

\noindent An interesting recent application that has been validated for a number of cases concerns the modeling of primary waves generated by large ships, see Section 2.9. 

\noindent 

\noindent \includegraphics*[width=5.92in, height=4.21in, keepaspectratio=false]{image18}

\noindent \textit{Figure 2.7 Measured (black) and modeled (red) time series of overtopping during BARDEX experiment, from McCall et al, 2014.}

\noindent \eject 


\subsection{ Short wave action }


\subsubsection{ Short wave action balance}

\noindent The wave forcing in the shallow water momentum equation is obtained from a time dependent version of the wave action balance equation. Similar to Delft University's (stationary) HISWA model (Holthuijsen et al., 1989) the directional distribution of the action density is taken into account, whereas the frequency spectrum is represented by a frequency, best represented by the spectral parameter\textit{ f${}_{m-1,0}$}. The wave action balance (keyword: \textit{swave}) is then given by:

\noindent 
\begin{equation} \label{ZEqnNum871591} 
\frac{\partial A}{\partial t} +\frac{\partial c_{x} A}{\partial x} +\frac{\partial c_{y} A}{\partial y} +\frac{\partial c_{\theta } A}{\partial \theta } =-\frac{D_{w} +D_{f} +D_{v} }{\sigma }  
\end{equation} 


\noindent In which the wave action \textit{A} is calculated as:

  
\begin{equation} \label{2.2)} 
A(x,y,t,\theta )=\frac{S_{w} (x,y,t,\theta )}{\sigma (x,y,t)}  
\end{equation} 


\noindent where \textit{$\theta $} represents the angle of incidence with respect to the x-axis, \textit{S${}_{w}$} represents the wave energy density in each directional bin and \textit{$\sigma $} the intrinsic wave frequency. The intrinsic frequency \textit{$\sigma $} and group velocity \textit{c${}_{g}$} is obtained from the linear dispersion relation. The intrinsic frequency is for example obtained with:

\noindent 
\begin{equation} \label{2.3)} 
\sigma =\sqrt{gk\tanh kh}  
\end{equation} 


\noindent The wave action propagation speeds in \textit{x}, \textit{y} and directional space are given by:

\noindent 
\begin{equation} \label{2.4)} 
\begin{array}{l} {c_{x} (x,y,t,\theta )=c_{g} \cos (\theta )} \\ {c_{y} (x,y,t,\theta )=c_{g} \sin (\theta )} \\ {c_{\theta } (x,y,t,\theta )=\frac{\sigma }{\sinh 2kh} \left(\frac{\partial h}{\partial x} \sin \theta -\frac{\partial h}{\partial y} \cos \theta \right)} \end{array} 
\end{equation} 


\noindent where \textit{h} represents the local water depth and \textit{k} the wave number. The intrinsic wave frequency $\sigma $ is determined without wave current interaction (keyword: wci=1, see Section 2.3.1.1), which means it is equal to the absolute radial frequency $\omega $.


\paragraph{ Wave current interaction (wci)}

\noindent Wave-current interaction is the interaction between waves and the mean flow. The interaction implies an exchange of energy, so after the start of the interaction both the waves and the mean flow are affected by each other. This feature is especially of importance in gullies and rip-currents (Reniers et al., 2007).

\noindent 

\noindent In XBeach this is taken into account by correcting the wave number \textit{k} with the use of Eikonal equations, which will have impact on the group and wave propagation speed (x, y, and directional space). The cross-shore and alongshore wave numbers, \textit{k${}_{x}$} and \textit{k${}_{y}$}, are defined according to equation \eqref{ZEqnNum901888}. In these formulations the subscripts refer to the direction of the wave vector components.

\noindent 
\begin{equation} \label{ZEqnNum901888} 
\begin{array}{l} {k_{x} =k_{x}^{n-1} +k_{x}^{\sim } } \\ {k_{y} =k_{y}^{n-1} +k_{y}^{\sim } } \end{array} 
\end{equation} 


\noindent Where subscript \textit{n-1} refers the wave number of the previous time step, $k_{x}^{\sim } $ and $k_{y}^{\sim } $ are the wave number corrections and \textit{k${}_{x}$} and \textit{k${}_{y}$} are the corrected wave numbers that take into account the presence of a current. The correction terms are determined with a second set of equations, the Eikonal equations: 

 
\begin{equation} \label{ZEqnNum227968} 
\begin{array}{l} {\frac{\partial k_{x} }{\partial t} +\frac{\partial \omega }{\partial x} =0} \\ {\frac{\partial k_{y} }{\partial t} +\frac{\partial \omega }{\partial y} =0} \end{array} 
\end{equation} 


\noindent The wave number is then given by:

\noindent 
\begin{equation} \label{2.7)} 
k=\sqrt{k_{x}^{2} +k_{y}^{2} }  
\end{equation} 


\noindent The absolute radial frequency \textit{$\omega $} is calculated with:

\noindent  
\begin{equation} \label{2.8)} 
\omega =\sigma +k_{x} u^{L} +k_{y} v^{L}  
\end{equation} 


\noindent \textit{where u${}^{L}$} and \textit{v${}^{L}$} are the cross-shore and alongshore depth-averaged Lagrangian velocities respectively. The wave action propagation speed (in x and y direction) is given by:

\noindent 
\begin{equation} \label{2.9)} 
\begin{array}{l} {c_{x} (x,y,t,\theta )=c_{g} \cos (\theta )+u^{L} } \\ {c_{y} (x,y,t,\theta )=c_{g} \sin (\theta )+v^{L} } \end{array} 
\end{equation} 


\noindent The propagation speed in directional (\textit{$\theta $)} space, where bottom refraction (first term) and current refraction (last two terms) are taken into account, is obtained from:

\noindent 
\begin{equation} \label{2.10)} 
\begin{array}{r} {c_{\theta } (x,y,t,\theta )=\frac{\sigma }{\sinh 2kh} \left(\frac{\partial h}{\partial x} \sin \theta -\frac{\partial h}{\partial y} \cos \theta \right)+\cos \theta \left(\sin \theta \frac{\partial u}{\partial x} -\cos \theta \frac{\partial u}{\partial y} \right)} \\ {+\, \sin \theta \left(\sin \theta \frac{\partial v}{\partial x} -\cos \theta \frac{\partial v}{\partial y} \right)} \end{array} 
\end{equation} 


\subsubsection{ Dissipation}

\noindent In XBeach there are three short wave dissipation processes that can be accounted for: wave breaking (\textit{D${}_{w}$}), bottom friction (\textit{D${}_{f}$}) and vegetation (\textit{D${}_{v}$}). The three processes are explained in more detail in the following subsections.


\paragraph{ Wave breaking}

\noindent Five different wave breaking formulations are implemented in XBeach. The formulations can be selected using the keyword \textit{break }(\eject Table 2.1). 

\noindent 

\noindent \eject \textit{}

\noindent \textit{Table 2.1 Different wave breaking formulations implemented}

\begin{tabular}{|p{1.3in}|p{1.4in}|p{1.4in}|} \hline 
\textbf{Wave breaking formula} & \textbf{Type of waves} & \textbf{keyword} \\ \hline 
\textbf{Roelvink (1993a)} & \textbf{Instationary} & \textbf{roelvink1} \\ \hline 
\textbf{Roelvink (1993a) extended} & \textbf{Instationary} & \textbf{roelvink2} \\ \hline 
\textbf{Daly et al. (2010)} & \textbf{Instationary} & \textbf{roelvink\_daly} \\ \hline 
\textbf{Baldock et al. (1998)} & \textbf{Stationary} & \textbf{baldock} \\ \hline 
\textbf{Janssen \& Battjes (2007)} & \textbf{Stationary} & \textbf{janssen} \\ \hline 
\end{tabular}



\noindent For the surf beat approach the total wave energy dissipation, i.e. directionally integrated, due to wave breaking can be modeled according to Roelvink (1993a, keyword: \textit{break=roelvink1}). In the formulation of the dissipation due to wave breaking the idea is to calculate the dissipation with a fraction of breaking waves (\textit{Q${}_{b}$}) multiplied by the dissipation per breaking event. In this formulation \textit{$\alpha $ }is applied as wave dissipation coefficient of O\eqref{GrindEQ__1_} (keyword: \textit{alpha}), \textit{T${}_{rep}$ }is the representative wave period and \textit{E${}_{w}$} is the energy of the wave. The fraction of wave breaking is determined with the root-mean-square wave height (\textit{H${}_{rms}$}) and the maximum wave height (\textit{H${}_{max}$}). The maximum wave height is calculated as ratio of the water depth (\textit{h}) plus a fraction of the wave height \textit{($\delta $H${}_{rms}$}, keyword: \textit{delta}) using a breaker index \textit{$\gamma $ }(keyword:\textit{ gamma}). In the formulation for \textit{H${}_{rms}$} the \textit{$\rho $} represents the water density and g the gravitational constant. The total wave energy \textit{E${}_{w}$} is calculated by integrating over the wave directional bins. 

\noindent 
\begin{equation} \label{ZEqnNum240661} 
\begin{array}{c} {\bar{D}_{w} =2\frac{\alpha }{T_{rep} } Q_{b} E_{w} {\rm \; }} \\ {Q_{b} {\rm =1-exp}\left(-\left(\frac{H_{rms} }{H_{\max } } \right)^{n} \right),\quad H_{rms} =\sqrt{\frac{8E_{w} }{\rho g} } ,\quad H_{\max } =\gamma \cdot (h+\delta H_{rms} )} \\ {E_{w} (x,y,t)=\int _{0}^{2\pi }S_{w} (x,y,t,\theta )d\theta  } \end{array} 
\end{equation} 


\noindent In variation of \eqref{ZEqnNum240661}, one could also use another wave breaking formulation, presented in \eqref{ZEqnNum862942}. This formulation is somewhat different than the formulation of Roelvink (1993a) and selected using keyword \textit{break=roelvink2.} The main difference with the original formulation is that wave dissipation with \textit{break=roelvink2} is proportional to H${}^{3}$/h instead of H${}^{2}$.

\noindent 
\begin{equation} \label{ZEqnNum862942} 
\bar{D}_{w} =2\frac{\alpha }{T_{rep} } Q_{b} E_{w} \frac{H_{rms} }{h}  
\end{equation} 


\noindent Alternatively the formulation of Daly et al. (2010) states that waves are fully breaking if the wave height exceeds a threshold (\textit{$\gamma $}) and stop breaking if the wave height fall below another threshold (\textit{$\gamma $${}_{2}$}). This formulation is selected by \textit{break=roelvink\_daly} and the second threshold, \textit{$\gamma $${}_{2}$}, can be set using keyword: \textit{gamma2.}

\noindent 
\begin{equation} \label{2.13)} 
\left\{\begin{array}{l} {Q_{b} =1\quad if\quad H_{rms} >\gamma h} \\ {Q_{b} =0\quad if\quad H_{rms} <\gamma _{2} h} \end{array}\right.  
\end{equation} 


\noindent In case of stationary waves Baldock et al. (1998) is applied (keyword: \textit{break=baldock}), which is presented in \eqref{ZEqnNum650667}. In this breaking formulation the fraction breaking waves \textit{Q${}_{b}$} and breaking wave height \textit{H${}_{b}$} are calculated differently compared to the breaking formulations used for the non-stationary situation. In \eqref{ZEqnNum650667} \textit{$\alpha $} is applied as wave dissipation coefficient, \textit{f${}_{rep}$} represents a representative intrinsic frequency and \textit{y} is a calibration factor.

\noindent 
\begin{equation} \label{ZEqnNum650667} 
\begin{array}{l} {\bar{D}_{w} =\frac{1}{4} \alpha Q_{b} \rho gf_{rep} \left(H_{b}^{2} +H_{rms}^{2} \right)} \\ {Q_{b} =\exp \left[-\left(\frac{H_{b}^{2} }{H_{rms}^{2} } \right)\right]{\rm \; ,\; \; }H_{b} =\frac{0.88}{k} \tanh \left[\frac{\gamma kh}{0.88} \right]} \end{array} 
\end{equation} 


\noindent Finally, it is possible to use the Janssen \& Battjes (2007) formulation for wave breaking of stationary waves (keyword: \textit{break=janssen}). This formulation is a revision of Baldock's formulation. 

\noindent 

 
\begin{equation} \label{2.15)} 
\begin{array}{l} {\bar{D}_{w} =\frac{3\sqrt{\pi } \alpha f_{rep} \rho gH_{rms}^{3} }{16} Q_{b} } \\ {Q_{b} =1+\frac{4}{3\sqrt{\pi } } \left(R^{3} +\frac{3}{2} R\right)\exp \left(-R^{2} \right)-erf\left(R\right)} \\ {R=\frac{H_{b} }{H_{rms} } } \end{array} 
\end{equation} 

 

\noindent In both the instationary or stationary case the total wave dissipation is distributed proportionally over the wave directions with the formulation in \eqref{ZEqnNum200641}.

\noindent 
\begin{equation} \label{ZEqnNum200641} 
D_{w} (x,y,t,\theta )=\frac{S_{w} (x,y,t,\theta )}{E_{w} (x,y,t)} \bar{D}_{w} (x,y,t) 
\end{equation} 


\paragraph{ Bottom friction}

\noindent The short wave dissipation by bottom friction is modeled as

\noindent 
\begin{equation} \label{ZEqnNum711028} 
D_{f} =\frac{2}{3\pi } \rho f_{w} \left(\frac{\pi H_{rms} }{T_{m01} \sinh kh} \right)^{3}  
\end{equation} 


\noindent In \eqref{ZEqnNum711028} the \textit{f${}_{w}$} is the short-wave friction coefficient. This value only affects the wave action equation and is unrelated to bed friction in the flow equation. Studies conducted on reefs (e.g. Lowe et al., 2007) indicate that \textit{f${}_{w}$} should be an order of magnitude (or more) larger than the friction coefficient for flow (\textit{c${}_{f}$})${}_{ }$due to the dependency of wave frictional dissipation rates on the frequency of the motion.\textit{}

\noindent 

\noindent The derivation of the short wave dissipation term is based time-averaged instantaneous bottom dissipation using the Johnson friction factor \textit{f${}_{w}$}${}_{ }$of the bed shear stress:

\noindent 
\begin{equation} \label{ZEqnNum889680} 
\tilde{D}_{f} =\left|\tau u\right|=\frac{1}{2} \rho f_{w} \left|\tilde{u}\right|^{3}  
\end{equation} 
The evaluation of the term $\left\langle \left|\tilde{u}\right|^{3} \right\rangle $, the so-called third even velocity moment, depends on the situation. First we need expressions for the orbital velocity amplitude, which is expressed as:

\noindent 
\begin{equation} \label{ZEqnNum105456} 
u_{orb} =\frac{\pi H_{rms} }{T_{p} \sinh (kh)}  
\end{equation} 


\noindent In this formulation \textit{T${}_{p}$} is the peak wave period, \textit{H${}_{rms}$} is the root-mean-square wave height, \textit{k} is the wave number and \textit{h} is the local water depth. 

\noindent 

\noindent If we consider the slowly-varying dissipation in wave groups, we need only to average over a single wave period and we can use a monochromatic (regular wave) expression. If we want to have the time-average dissipation over a full spectrum we get the best approximation from considering a linear Gaussian distribution. Guza and Thornton (1985) give pragmatic expressions for both cases.

\noindent 

\noindent For the monochromatic case:

 
\begin{equation} \label{2.20)} 
\left\langle \left|\tilde{u}\right|^{3} \right\rangle =1.20\left\langle \left|\tilde{u}\right|^{2} \right\rangle ^{3/2} =1.20\left(\frac{1}{2} u_{orb}^{2} \right)^{3/2} =0.42u_{orb}^{3}  
\end{equation} 


\noindent For the linear Gaussian approximation:

\noindent 
\begin{equation} \label{ZEqnNum278204} 
\left\langle \left|\tilde{u}\right|^{3} \right\rangle =1.60\left\langle \left|\tilde{u}\right|^{2} \right\rangle ^{3/2} =1.60\left(\frac{1}{2} u_{orb}^{2} \right)^{3/2} =0.57u_{orb}^{3}  
\end{equation} 


\noindent Combining \eqref{ZEqnNum889680} and \eqref{ZEqnNum862942} we get:

\noindent 
\begin{equation} \label{2.22)} 
\left\langle \tilde{D}_{f} \right\rangle =0.21\rho f_{w} u_{orb}^{3}  
\end{equation} 


\noindent In XBeach the orbital velocity amplitude is computed as in \eqref{ZEqnNum105456} end the dissipation according to \eqref{ZEqnNum862942} which is correct for the case of instationary simulations on wave-group scale. 

\noindent 

\noindent For the stationary case formulations \eqref{ZEqnNum105456} and \eqref{ZEqnNum278204} are similarly combined into: 

\noindent 
\begin{equation} \label{ZEqnNum720951} 
\left\langle \tilde{D}_{f} \right\rangle =0.28\rho f_{w} u_{orb}^{3}  
\end{equation} 


\paragraph{ Vegetation}

\noindent The presence of aquatic vegetation within the area of wave propagation or wave breaking results in an additional dissipation mechanism for short waves. This is modeled using the approach of Mendez \& Losada (2004), which was adjusted by Suzuki et al., (2011) to take into account vertically heterogeneous vegetation, see Van Rooijen et al. (2015). The short wave dissipation due to vegetation is calculated as function of the local wave height and several vegetation parameters. The vegetation can be schematized in a number of vertical elements with each specific property. In this way the wave damping effect of vegetation such as mangrove trees, with a relatively dense root system but sparse stem area, can be modeled. The dissipation term is then computed as the sum of the dissipation per vegetation layer (Suzuki et al, 2011):
\begin{equation} \label{2.24)} 
D_{v} =\sum _{i=1}^{n_{v} }D_{v,i}   
\end{equation} 


\noindent where \textit{D${}_{v,i}$}${}_{ }$is the dissipation by vegetation in vegetation layer \textit{i} and \textit{n${}_{v}$} is the number of vegetation layers. The dissipation per layer is given by:

\noindent 
\begin{equation} \label{2.25)} 
\begin{array}{l} {D_{v,i} =A_{v} \cdot \frac{\rho \widetilde{C}_{D,i} b_{v,i} N_{v,i} }{2\sqrt{\pi } } \left(\frac{kg}{2\sigma } \right)^{3} H_{rms} ^{3} ,{\rm \; with}} \\ {A_{v} =\frac{\left(\sinh ^{3} k\alpha _{i} h-\sinh ^{3} k\alpha _{i-1} h\right)+3\left(\sinh k\alpha _{i} h-\sinh k\alpha _{i-1} h\right)}{3k\cosh ^{3} kh} } \end{array} 
\end{equation} 


\noindent where $\widetilde{C}_{D,i} $is a (bulk) drag coefficient, \textit{b${}_{v,i}$} is the vegetation stem diameter, \textit{N${}_{v,i}$} is the vegetation density, and \textit{$\alpha $${}_{i}$} is the relative vegetation height (= \textit{h${}_{v}$}${}_{ }$/ \textit{h}) for layer \textit{i}. In case only one vegetation layer is specified, the plants are assumed to be vertically uniform, which would for example typically apply in case of modeling sea grass.


\subsubsection{ Radiation stresses}

\noindent Given the spatial distribution of the wave action (and therefore wave energy) the radiation stresses can be evaluated by using linear wave theory as described by:

\noindent 
\begin{equation} \label{2.26)} 
\begin{array}{l} {S_{xx,r} (x,y,t)=\int \cos ^{2} \theta S_{r} d\theta  } \\ {S_{xy,r} (x,y,t)=S_{yx,r} (x,y,t)=\int \sin \theta \cos \theta S_{r}  d\theta } \\ {S_{yy,r} (x,y,t)=\int \sin ^{2} \theta S_{r} d\theta  } \end{array} 
\end{equation} 


\subsubsection{ Wave shape}

\noindent The morphodynamic model considered is (short) wave averaged and resolves hydrodynamics associated with the wave group time scale. As a result the short wave shape is not solved for. However, as waves propagate from deep water onto beaches, their surface form and orbital water motion become increasingly non-linear because of the amplification of the higher harmonics.

\noindent 

\noindent There are two wave forms implemented to take this non-linearity into account:

\begin{enumerate}
\item  A formulation of Ruessink et al. (2012) based on a parameterization with the Ursell number. (keyword: \textit{waveform = ruessink\_vanrijn})

\item  A formulation of Van Thiel de Vries (2009) based on the parameterized wave shape model of Rienecker and Fenton (1981) (keyword: \textit{waveform = vanthiel})
\end{enumerate}

\noindent 

\noindent The formulation of Ruessink et al. (2012) relies on parameterizations for the non-linearity parameter \textit{r} and phase \textit{$\Phi $}. The parameterizations are based on a data set of 30.000+ field observations of the orbital skewness \textit{S${}_{k}$} and asymmetry \textit{A${}_{s}$}, collected under non-breaking and breaking wave conditions. The only variable parameter is the Ursell number, since according to Ruessink et al. (2012) the Ursell that includes \textit{H${}_{s}$,} \textit{T} and \textit{h}, describes the variability in \textit{S${}_{k}$} and \textit{A${}_{s}$} well. The Ursell number is calculated with the equation below.

\noindent 
\begin{equation} \label{2.27)} 
U_{r} =\frac{3}{4} \frac{0.5H_{s} k}{(kh)^{3} }  
\end{equation} 


\noindent The value for the skewness and asymmetry is calculated with the use of a Boltzmann sigmoid. The skewness and asymmetry are a function of \textit{$\Psi $}. In the formulation of Ruessink et al. (2012) the \textit{p${}_{1:6}$} are used as parameterized factors on the data set of field observations.

\noindent 
\begin{equation} \label{2.28)} 
\begin{array}{c} {B=p_{1} +\frac{p_{2} -p_{1} }{1+\exp \frac{p_{3} -\log Ur}{p_{4} } } } \\ {\psi =-90+90\tanh (p_{5} /Ur^{p_{6} } )} \\ {} \\ {S_{k} =B\cos \psi {\rm \; \; \; and\; \; \; }A_{s} =B\sin \psi } \end{array} 
\end{equation} 


\noindent Alternatively, Van Thiel de Vries (2009) utilized and extended the wave shape model of Rienecker and Fenton (1981). In this model the short wave shape is described by the weighted sum of eight sine and cosine functions

\noindent 
\begin{equation} \label{2.29)} 
u_{bed} =\sum _{i=1}^{i=8}wA_{i} \cos (i\omega t)+{\rm \; }(1-w) A_{i} \sin (i\omega t) 
\end{equation} 


\noindent where u${}_{bed}$ is the near-bed short wave flow velocity, \textit{i} refers to the \textit{i}th harmonic, \textit{$\omega $} is the angular wave frequency, \textit{A${}_{i}$} is the amplitude of a specific harmonic and \textit{w} is a weighting function affecting the wave shape. The amplitudes \textit{A${}_{1:8}$}${}_{ }$are computed from stream function theory and vary with the dimensionless wave height and dimensionless wave period.

\noindent 

\noindent The wave skewness of near bed flow velocities is computed according to \eqref{ZEqnNum619512}. The wave asymmetries (\textit{A${}_{s}$}) can be computed with the same expression replacing u${}_{bed}$ by its Hilbert transform.

\noindent 
\begin{equation} \label{ZEqnNum619512} 
S_{k} =\frac{\overline{u_{bed}^{3} }}{\sigma _{u_{bed} }^{3} }  
\end{equation} 


\noindent For \textit{w} equals one a skewed (Stokes) wave is obtained with high peaks and flat troughs whereas \textit{w} equals zero results in an asymmetric (saw tooth) wave with steep wave fronts. It is hypothesized that the weighting \textit{w} can be expressed as a function of wave skewness and asymmetry. The relation between the phase and the weighting is studied in more detail by Van Thiel de Vries (2009) by varying \textit{w} between zero and one in small steps and computing the amplitudes \textit{A${}_{1:8}$}${}_{ }$with Rienecker and Fenton for a range of wave heights, wave periods and water depths. It is found that a unique relation between \textit{w} and \textit{$\Phi $} exists for any combination of wave height, wave period and water depth that is described by:

\noindent 
\begin{equation} \label{2.31)} 
w=0.2719\ln \left(\left|\frac{\phi -1.8642}{0.2933-\phi } \right|\right)+0.5 
\end{equation} 


\noindent As explained in the next section, short-wave turbulence can be computed averaged over the bore interval (T${}_{bore}$). The bore interval is directly related to the wave shape and hence requires the weighting function \textit{w} is determined. For the formulation of Ruessink et al. (2012) no exact wave shape is determined and therefore no bore interval can be calculated. Therefore this approach cannot be combined with bore averaged short-wave turbulence.


\subsubsection{ Turbulence}

\noindent Wave breaking induced turbulence at the water surface has to be transported towards the bed in order to affect the up-stirring of sediment. Roelvink and Stive (1989) used an exponential decay model with the mixing length proportional to \textit{H${}_{rms}$} to estimate the time averaged turbulence energy at the bed from turbulence at the water surface:

\noindent 
\begin{equation} \label{2.32)} 
k_{b} =\frac{k}{\exp (h/H_{rms} )-1}  
\end{equation} 


\noindent where \textit{k${}_{b}$} is turbulence variance at the bed and \textit{k} is the time averaged turbulence variance at the water surface. 

\noindent 

\noindent There are three possibilities for the turbulence variance at the bed (\textit{k${}_{b}$}) implemented into XBeach:

\noindent 

\begin{enumerate}
\item  Wave averaged near-bed turbulence energy (keyword: \textit{turb = wave\_averaged}):
\begin{equation} \label{2.33)} 
k_{b} =\frac{\overline{k_{s} }}{\exp (h/L_{mix} )-1}  
\end{equation} 
\end{enumerate}


\begin{enumerate}
\item  Bore-averaged near-bed turbulence energy\footnote{ $ Currently, this formulation is only possible when the wave shape formulation of Van Thiel de Vries (2009) is applied, see section $  $2.3.4.$ } (keyword: \textit{turb = bore\_averaged})
\begin{equation} \label{2.34)} 
k_{b} =\frac{\overline{k_{s} }\cdot T_{rep} /T_{bore} }{\exp (h/L_{mix} )-1}  
\end{equation} 
\end{enumerate}


\begin{enumerate}
\item  Not taking into account the turbulence variance at the bed (keyword: \textit{turb = none})
\end{enumerate}

\noindent 

\noindent Both formulations make use of the wave-averaged turbulence energy (\textit{k${}_{s}$}) and a mixing length (\textit{L${}_{mix}$}). The wave averaged turbulence energy at the surface is computed from the roller energy dissipation and following Battjes (1975) in which \textit{D${}_{r}$}${}_{ }$is roller dissipation:

\noindent 
\begin{equation} \label{2.35)} 
\overline{k_{s} }=\left(D_{r} /\rho _{w} \right)^{2/3}  
\end{equation} 


\noindent The mixing length (\textit{L${}_{mix}$}) is expressed as thickness of the surface roller near the water surface and depends on the roller volume \textit{A${}_{r}$} (Svendsen, 1984):

\noindent 
\begin{equation} \label{2.36)} 
L_{mix} =\sqrt{A_{r} } =\sqrt{\frac{2E_{r} T_{rep} }{\rho _{w} c_{w} } }  
\end{equation} 


\subsubsection{ Roller energy balance}

\noindent While the short wave action balance adequately describes the propagation and decay of organized wave energy, it has often been found that there is a delay between the point where the waves start to break (which is where you would expect the strongest radiation stress gradients to occur) and the point where the wave set-up and longshore current start to build. This transition zone effect is generally attributed to the temporary storage of shoreward momentum in the surface rollers. Several authors have analyzed the typical dimensions of such rollers and their effect on the radiation stress (e.g. Longuet-Higgins and Turner, 1974, Svendsen, 1984, Roelvink and Stive, 1989, Nairn et al., 1990, Deigaard 1993, Stive and De Vriend, 1994). 

\noindent 

\noindent The rollers can be represented as a blob of water with cross-sectional area A that slides down the front slope of a breaking wave. The roller exerts a shear stress on the water beneath it equal to:

\noindent 
\begin{equation} \label{ZEqnNum145546} 
\tau _{roller} =\frac{\rho gR}{L} \beta _{s}  
\end{equation} 


\noindent where $\beta _{s} $ is the slope of the breaking wave front, \textit{R} is the roller area and \textit{L} is the wave length. The roller has a kinetic energy equal to:

\noindent 
\begin{equation} \label{ZEqnNum992472} 
E_{r} =\frac{1}{2} \frac{\rho R(\overline{u_{roller}^{2} +w_{roller}^{2} })}{L}  
\end{equation} 


\noindent and a contribution to the radiation stress equal to:

\noindent 
\begin{equation} \label{2.39)} 
S_{xx} =\frac{\rho R\overline{\left(u_{roller}^{2} -w_{roller}^{2} \right)}}{L}  
\end{equation} 


\noindent We can now formulate an energy balance for the roller as follows:

\noindent 
\begin{equation} \label{ZEqnNum923440} 
\frac{dE_{r} }{dt} =\frac{\partial E_{r} }{\partial t} +\frac{\partial E_{r} c\cos \theta }{\partial x} +\frac{\partial E_{r} c\sin \theta }{\partial y} =S-D 
\end{equation} 


\noindent where \textit{S} is the loss of organized wave motion due to breaking and \textit{D} is the dissipation.  The latter is equal to the work done by the shear stress between the roller and the wave:

\noindent 
\begin{equation} \label{ZEqnNum821204} 
D_{r} =\tau _{roller} c_{g}  
\end{equation} 


\noindent Given the complex motion in the breaking waves, we can only give approximate estimates of the order of magnitude of the parameters in the equations \eqref{ZEqnNum145546} till \eqref{ZEqnNum821204}. Various authors have suggested that the velocity in the roller can be approximated as purely horizontal and equal to the wave celerity \textit{c${}_{g}$. }In that case we get (for waves travelling in x-direction):

\noindent 
\begin{equation} \label{2.42)} 
S_{xx,roller} =2E_{r}  
\end{equation} 


\noindent However, this must be seen as an (unrealistic) upper limit on the radiation stress contribution as this can only be valid for \textit{w${}_{roller}$=0}. Nairn et al. (1990) showed that the conceptual model of Roelvink and Stive (1989) would lead to a factor 0.22 instead of 2. However, a ratio in the order of 1 seems more realistic. Stive and De Vriend (1994) found a discrepancy between the roller shear stress derived from an energy balance and that derived from the momentum balance, in the order of a factor two. They explained this by a complicated analysis of the effect of water entering and leaving the roller, which led to a modification of the propagation term in the roller energy balance by a factor two. As this leads to the unphysical result that rollers would propagate at twice the wave celerity, we believe that the discrepancy must be sought in the ratio between roller energy and radiation stress contribution. Therefore we stick to the roller energy balance suggested by Nairn et al. (1990) in equation \eqref{ZEqnNum156223} and the roller contribution to the radiation stress:

\noindent 
\begin{equation} \label{ZEqnNum156223} 
\begin{array}{l} {S_{xx,roller} \approx E_{r} \cos ^{2} \theta } \\ {S_{xy,roller} \approx E_{r} \cos \theta \sin \theta } \\ {S_{yy,roller} \approx E_{r} \sin ^{2} \theta } \end{array} 
\end{equation} 


\noindent This leads to an elegant and consistent distribution of the wave-induced forcing through the surfzone. To close the roller energy balance we need to express the dissipation of the roller as a function of \textit{E${}_{r}$}. This can be done by introducing:

\noindent 
\begin{equation} \label{ZEqnNum157006} 
\overline{\left(u_{roller}^{2} +w_{roller}^{2} \right)}=\beta _{2} c_{g} ^{2}  
\end{equation} 


\noindent Combining this with equations \eqref{ZEqnNum992472} and \eqref{ZEqnNum157006} we then find:

\noindent 
\begin{equation} \label{2.45)} 
D_{r} =2\beta _{s} \beta _{u} \frac{g}{c_{g} } E_{r}  
\end{equation} 


\noindent The coefficients $\beta _{s} $ and $\beta _{u} $ are usually lumped together into a single coefficient. This coefficient $\beta $ is in the O\eqref{GrindEQ__0_1_} (keyword: \textit{beta}), which may vary through the surf zone. The forcing of the longshore current by the radiation stress gradient can be derived from the wave and roller energy balances \eqref{ZEqnNum871591} and \eqref{ZEqnNum923440}:

 
\begin{equation} \label{2.46)} 
\begin{array}{l} {F_{y} =-\frac{\partial S_{xy} }{\partial x} =-\frac{\partial }{\partial x} \left[\frac{c_{g} }{c} \left(E\cos \left(\theta \right)\sin \left(\theta \right)\right)+E_{r} \cos \left(\theta \right)\sin \left(\theta \right)\right]} \\ {\, \, \, \, \, \, \, \, \, \, \, =-\frac{\partial }{\partial x} \left[\frac{\sin \left(\theta \right)}{c} \left(Ec_{g} \cos \left(\theta \right)+E_{r} c\cos \left(\theta \right)\right)\right]} \\ {\, \, \, \, \, \, \, \, \, \, \, =-\left(Ec_{g} \cos \left(\theta \right)+E_{r} c\cos \left(\theta \right)\right)\frac{\partial }{\partial x} \left[\frac{\sin \left(\theta \right)}{c} \right]\, \, } \\ {\, \, \, \, \, \, \, \, \, \, \, \, \, \, \, -\frac{\sin \left(\theta \right)}{c} \frac{\partial }{\partial x} \left(Ec_{g} \cos \left(\theta \right)+E_{r} c\cos \left(\theta \right)\right)} \end{array} 
\end{equation} 


\noindent In a longshore uniform situation, according to Snell's law, the first term on the right-hand side equals zero; the second term exactly equals the sum of the wave energy dissipation and the roller energy input and dissipation terms, so the forcing term reduces to:

\noindent 
\begin{equation} \label{ZEqnNum515341} 
F_{y} =\frac{D_{w} +(-D_{w} +D_{r} )}{c} \sin (\alpha )=\frac{D_{r} }{c} \sin (\alpha ) 
\end{equation} 


\subsection{ Shallow water equations}

\noindent For the low-frequency waves and mean flows we use the shallow water equations. To account for the wave induced mass-flux and the subsequent (return) flow these are cast into a depth-averaged Generalized Lagrangian Mean (GLM) formulation (Andrews and McIntyre, 1978, Walstra et al, 2000). In such a framework, the momentum and continuity equations are formulated in terms of the Lagrangian velocity\textit{ u${}^{L}$ }which is defined as the distance a water particle travels in one wave period, divided by that period. This velocity is related to the Eulerian velocity (the short-wave-averaged velocity observed at a fixed point) by:

\noindent 
\begin{equation} \label{2.48)} 
u^{L} =u^{E} +u^{S} \quad and\quad v^{L} =v^{E} +v^{S}  
\end{equation} 


\noindent where u${}^{S}$ and v${}^{S}$ represent the Stokes drift in \textit{x}- and \textit{y}-direction respectively (Phillips, 1977). The Strokes drift is calculated with \eqref{ZEqnNum369529} in which the wave-group varying short wave energy \textit{E${}_{w}$} and direction are obtained from the wave-action balance.

\noindent 
\begin{equation} \label{ZEqnNum369529} 
u^{S} =\frac{E_{w} \cos \theta }{\rho hc} \quad and\quad v^{S} =\frac{E_{w} \sin \theta }{\rho hc}  
\end{equation} 


\noindent The resulting GLM-momentum equations are given by:  

\noindent 
\begin{equation} \label{ZEqnNum488519} 
\begin{array}{c} {\frac{\partial u^{L} }{\partial t} +u^{L} \frac{\partial u^{L} }{\partial x} +v^{L} \frac{\partial u^{L} }{\partial y} -f\, v^{L} \, -\, \nu _{h} \left(\frac{\partial ^{2} u^{L} }{\partial x^{2} } +\frac{\partial ^{2} u^{L} }{\partial y^{2} } \right)=\frac{\tau _{sx} }{\rho h} -\frac{\tau _{bx}^{E} }{\rho h} -g\frac{\partial \eta }{\partial x} +\frac{F_{x} }{\rho h} +\frac{F_{v,x} }{\rho h} } \\ {\frac{\partial v^{L} }{\partial t} +u^{L} \frac{\partial v^{L} }{\partial x} +v^{L} \frac{\partial v^{L} }{\partial y} +f\, u^{L} \, -\, \nu _{h} \left(\frac{\partial ^{2} v^{L} }{\partial x^{2} } +\frac{\partial ^{2} v^{L} }{\partial y^{2} } \right)=\frac{\tau _{sy} }{\rho h} -\frac{\tau _{by}^{E} }{\rho h} -g\frac{\partial \eta }{\partial y} +\frac{F_{y} }{\rho h} +\frac{F_{v,y} }{\rho h} } \\ {\frac{\partial \eta }{\partial t} +\frac{\partial hu^{L} }{\partial x} +\frac{\partial hv^{L} }{\partial y} =0} \end{array} 
\end{equation} 


\noindent where \textit{$\tau $${}_{sx}$}${}_{ }$and \textit{$\tau $${}_{sy}$}${}_{ }$are the wind shear stresses\textit{, $\tau $${}_{bx}$}${}_{ }$and \textit{$\tau $${}_{by}$}${}_{ }$are the bed shear stresses, \textit{$\eta $} is the water level, \textit{F${}_{x}$ }and \textit{F${}_{y}$} are the wave-induced stresses, \textit{F${}_{v,x}$, }and \textit{F${}_{v,y}$} are the stresses induced by vegetation, \textit{$\nu $${}_{h}$} is the horizontal viscosity and \textit{f} is the Coriolis coefficient. Note that the shear stress terms are calculated with the Eulerian velocities as experienced by the bed and not with the GLM velocities, as can be seen in \eqref{ZEqnNum488519}. 


\subsubsection{ Horizontal viscosity }

\noindent The horizontal viscosity (\textit{v${}_{h}$}) is by default computed using the Smagorinsky (1963) model to account for the exchange of horizontal momentum at spatial scales smaller than the computational grid size, which is given as:

\noindent 
\begin{equation} \label{ZEqnNum525463} 
v_{h} =c_{S} ^{2} 2^{\frac{1}{2} } \sqrt{\left(\frac{\delta u}{\delta x} \right)^{2} +\left(\frac{\delta v}{\delta y} \right)^{2} +\frac{1}{2} \left(\frac{\delta u}{\delta x} +\frac{\delta v}{\delta y} \right)^{2} } \Delta x\Delta y 
\end{equation} 


\noindent In \eqref{ZEqnNum525463} \textit{c${}_{S}$} is the Smagorinsky constant (keyword: \textit{nuh}), set at 0.1 in all model simulations. It is also possible to use a user-defined value for the horizontal viscosity by turning off the Smagorinsky model (keyword: \textit{smag = 0}) and specifying the value directly (also keyword: \textit{nuh}).


\subsubsection{ Bed shear stress}

\noindent The bed friction associated with mean currents and long waves is included via the formulation of the bed shear stress (\textit{$\tau $${}_{b}$}). Using the approach of Ruessink et al. (2001) the bed shear stress is calculated with:

\noindent 
\begin{equation} \label{2.52)} 
\begin{array}{l} {\tau _{bx}^{E} =c_{f} \rho u_{E} \sqrt{\left(1.16u_{rms} \right)^{2} +\left(u_{E} +v_{E} \right)^{2} } } \\ {\tau _{by}^{E} =c_{f} \rho v_{E} \sqrt{\left(1.16u_{rms} \right)^{2} +\left(u_{E} +v_{E} \right)^{2} } } \end{array} 
\end{equation} 


\noindent There are 5 different formulations in order to determine the dimensionless bed friction coefficient \textit{c${}_{f}$} (keyword: \textit{bedfriction}) implemented in XBeach (Table 2.2).

\noindent 

\noindent \textit{Table 2.2 Different bed friction formulations implemented}

\begin{tabular}{|p{1.6in}|p{1.1in}|p{1.3in}|} \hline 
\textbf{Bed friction formulation} & \textbf{Relevant coefficient } & \textbf{keyword} \\ \hline 
\textbf{Dimensionless friction coefficient} & \textbf{\textit{c${}_{f}$}} & \textbf{cf} \\ \hline 
\textbf{Ch�zy } & \textbf{C} & \textbf{chezy} \\ \hline 
\textbf{Manning} & \textbf{n} & \textbf{manning} \\ \hline 
\textbf{White-Colebrook} & \textbf{\textit{k${}_{s}$}} & \textbf{white-colebrook} \\ \hline 
\textbf{White-Colebrook grain size} & \textbf{D90} & \textbf{white-colebrook-grainsize} \\ \hline 
\end{tabular}



\noindent The dimensionless friction coefficient can be calculated from the Ch�zy value with equation \eqref{ZEqnNum120068}. A typical Ch�zy value is in the order of 55 m${}^{1/2}$/s.

\noindent 
\begin{equation} \label{ZEqnNum120068} 
c_{f} =\sqrt{\frac{g}{C^{2} } }  
\end{equation} 


\noindent In the Manning formulation the Manning coefficient (\textit{n}) must be specified. The dimensionless friction coefficient is calculated from equation \eqref{ZEqnNum752543}. Manning can be seen as a depth-dependent Ch�zy value and a typical Manning value would be in the order of 0.02 s/m${}^{1/3}$.

\noindent 
\begin{equation} \label{ZEqnNum752543} 
c_{f} =\sqrt{\frac{gn^{2} }{h^{1/12} } }  
\end{equation} 


\noindent In the White-Colebrook formulation the geometrical roughness of Nikuradse (\textit{k${}_{s}$}) must be specified. The dimensionless friction coefficient is calculated from \eqref{ZEqnNum849150} The White-Colebrook formulation has al log relation with the water depth and a typical \textit{k${}_{s}$} value would be in the order of 0.01 - 0.15 m. 

\noindent 
\begin{equation} \label{ZEqnNum849150} 
c_{f} =\sqrt{\frac{g}{\left(18\log \left(\frac{12h}{k_{s} } \right)\right)^{2} } }  
\end{equation} 


\noindent The option of White-Colebrook based on the grain size is somewhat different than the other four formulations. This formulation is based on the relation between the D${}_{90}$ of the top bed layer and the geometrical roughness of Nikuradse according to \eqref{ZEqnNum532913}.The user doesn't have to specify a value for the bed friction coefficient.

\noindent 
\begin{equation} \label{ZEqnNum532913} 
c_{f} =\sqrt{\frac{g}{\left(18\log \left(\frac{12h}{3D_{90} } \right)\right)^{2} } }  
\end{equation} 


\noindent Values of the drag coefficient for different seabed sediment grain sizes (flat beds) and similarly for bed form scenarios have been empirically derived from field and laboratory data in previous studies for different bed friction coefficients. The value of the friction coefficient (\textit{C, c${}_{f}$, n} or \textit{k${}_{s}$}) can be defined with one single value (keyword: \textit{bedfriccoef}) or for a separate value per grid cell (keyword: \textit{bedfricfile})


\subsubsection{ Damping by vegetation}

\noindent The presence of aquatic vegetation within the area of wave propagation or wave breaking may not only result in short wave dissipation (section 2.3.2.3), but also in damping of infragravity waves and/or mean flow. Since both long waves and mean flow are fully resolved with the nonlinear shallow water equations, the effect of vegetation can be modeled using a drag force (e.g. Dalrymple et al., 1984), which can be directly added to the momentum equations (Van Rooijen et al., 2015, equation \eqref{ZEqnNum488519}):

\noindent 
\begin{equation} \label{2.57)} 
F_{v} =F_{D} =\frac{1}{2} \rho C_{D} b_{v} Nu\left|u\right| 
\end{equation} 


\noindent Where \textit{C${}_{D}$} is a drag coefficient, \textit{b${}_{v}$} is the vegetation stem diameter, \textit{N} is the vegetation density and \textit{u} is the wave or current related velocity. To take into account the velocity due to mean flow and infragravity waves, we use the Lagrangian velocity (\textit{u${}^{L}$)}${}^{ }$here. The vegetation-induced time varying drag force is then calculated as the sum of the vegetation-induced drag force per vegetation layer:
\begin{equation} \label{2.58)} 
\begin{array}{l} {F_{v} (t)=\sum _{i=1}^{n_{v} }F_{v,i}  (t)} \\ {F_{v,i} (t)=\frac{1}{2} \rho \widetilde{C}_{D,i} b_{v,i} N_{v,i} h_{v,i} u^{L} (t)\left|u^{L} (t)\right|} \end{array} 
\end{equation} 


\noindent where $\widetilde{C}_{D,i} $is a (bulk) drag coefficient, \textit{b${}_{v,i}$} is the vegetation stem diameter, \textit{N${}_{v,i}$} is the vegetation density, and \textit{h${}_{v,i}$ }is the vegetation height for layer \textit{i}.


\subsubsection{ Wind}

\noindent The first term on the right hand side of the momentum equations \eqref{ZEqnNum488519} represents the forcing due to the wind stress. These forcing terms due to the wind are formulated as:

\noindent 

 $\tau _{sx} =\rho _{a} C_{d} W\left|W_{x} \right|$  \label{2.59)}          (
\[\tau _{sy} =\rho _{a} C_{d} W\left|W_{y} \right|\] 


\noindent where \textit{$\tau $${}_{w}$} is wind stress, \textit{$\rho $${}_{a}$} is density of air, \textit{C${}_{d}$} is the wind drag the coefficient, \textit{W} is the wind velocity. The wind stress is turned off by default, and can be turned on specifying a constant wind velocity (keyword: windv = \textit{value}) or by specifying a time varying wind file.


\subsection{ Non-hydrostatic pressure correction}

\noindent For non-hydrostatic XBeach calculations (keyword: \textit{waveform =} \textit{nonh}) depth-averaged flow due to waves and currents are computed using the non-linear shallow water equations, including a non-hydrostatic pressure. The non-hydrostatic model accounts for all wave motions (including short waves) within the shallow water equations, so the wave action balance should be turned off (keyword: \textit{swave} = \textit{0}). The depth-averaged normalized dynamic pressure (\textit{q}) is derived in a method similar to a one-layer version of the SWASH model (Zijlema et al. 2011). The depth averaged dynamic pressure is computed from the mean of the dynamic pressure at the surface and at the bed by assuming the dynamic pressure at the surface to be zero and a linear change over depth. In order to compute the normalized dynamic pressure at the bed, the contributions of advective and diffusive terms to the vertical momentum balance are assumed to be negligible.

\noindent 
\begin{equation} \label{ZEqnNum829942} 
\frac{\delta w}{\delta t} +\frac{\delta q}{\delta z} =0 
\end{equation} 


\noindent In \eqref{ZEqnNum829942} \textit{w} is the vertical velocity and \textit{z} is the vertical coordinate. The vertical velocity at the bed is set by the kinematic boundary condition:

\noindent 
\begin{equation} \label{ZEqnNum939704} 
w_{b} =u\frac{\delta (\eta -h)}{\delta x}  
\end{equation} 


\noindent Combining the Keller-box method (Lam and Simpson 1976), as applied by Stelling and Zijlema (2003) for the description of the pressure gradient in the vertical, the dynamic pressure at the bed can be described by:

\noindent 
\begin{equation} \label{ZEqnNum466261} 
q_{b} =-\frac{h}{2} \left(\left. \frac{\delta q}{\delta z} \right|_{s} +\left. \frac{\delta q}{\delta z} \right|_{b} \right) 
\end{equation} 


\noindent Substituting \eqref{ZEqnNum939704} in \eqref{ZEqnNum829942} allows the vertical momentum balance at the surface to be described by:

\noindent 
\begin{equation} \label{ZEqnNum801359} 
\frac{\delta w_{s} }{\delta t} =2\frac{q_{b} }{h} -\frac{\delta w_{b} }{\delta t}  
\end{equation} 


\noindent In \eqref{ZEqnNum801359} the subscript \textit{s} refers to the location at the surface. The dynamic pressure at the bed is subsequently solved by combining \eqref{ZEqnNum466261} and the local continuity equation:

\noindent 
\begin{equation} \label{2.64)} 
\frac{\delta u}{\delta x} +\frac{w_{s} -w_{b} }{h} =0 
\end{equation} 


\noindent In order to improve the computed location and magnitude of wave breaking, the hydrostatic front approximation (HFA) of Smit et al. (2013) is applied, in which the pressure distribution under breaking bores is assumed to be hydrostatic. Following the recommendations of Smit et al. (2013), we consider hydrostatic bores if $\frac{\delta \eta }{\delta t} >0.6$ and reform if $\frac{\delta \eta }{\delta t} <0.3$. The values can respectively be changed with the keywords \textit{maxbrsteep} and \textit{secbrsteep}. 

\noindent 

\noindent Although this method greatly oversimplifies the complex hydrodynamics of plunging waves, McCall et al. (2014) shows that the application of this method provides sufficient skill to describe dominant characteristics of the flow, without requiring computationally expensive high-resolution discretization of the vertical and surface tracking of overturning waves. 


\subsection{ Groundwater flow}

\noindent The groundwater module (keyword: \textit{gwflow = 1}) in XBeach utilizes the principle of Darcy flow for laminar flow conditions and a parameterization of the Forchheimer equations for turbulent groundwater flow. The module includes a vertical interaction flow between the surface water and groundwater. This flow is assumed to be a magnitude smaller than the horizontal flow and is not incorporated in the momentum balance.


\subsubsection{ Continuity}

\noindent In order to solve mass continuity in the groundwater model, the groundwater is assumed to be incompressible. Continuity is achieved by imposing a non-divergent flow field:

\noindent 
\begin{equation} \label{ZEqnNum113258} 
\nabla U=0 
\end{equation} 


\noindent where \textit{U} is the total specific discharge velocity vector, with components in the horizontal (\textit{u${}_{gw}$}, \textit{v${}_{gw}$}) and vertical (\textit{w${}_{gw}$}) direction:

\noindent 
\begin{equation} \label{2.66)} 
U=\left[\begin{array}{c} {u} \\ {v} \\ {w} \end{array}\right] 
\end{equation} 



\subsubsection{ Equation of motions}

\noindent Laminar flow of an incompressible fluid through a homogeneous medium can be described using the well-known Law of Darcy (1856), valid for laminar flow conditions (keyword: \textit{gwscheme = laminar})

\noindent 
\begin{equation} \label{ZEqnNum953353} 
\begin{array}{c} {u_{gw} =-K\frac{\partial H}{\partial x} } \\ {v_{gw} =-K\frac{\partial H}{\partial y} } \\ {w_{gw} =-K\frac{\partial H}{\partial z} } \end{array} 
\end{equation} 


\noindent in which \textit{K} is the hydraulic conductivity of the medium (keyword: \textit{kx, ky, kz}, for each horizontal and vertical direction) and \textit{H} is the hydraulic head.

\noindent 

\noindent In situations in which flow is not laminar, turbulent and inertial terms may become important. In these cases the user can specify XBeach to use a method (keyword: \textit{gwscheme = turbulent}) that is comparable with the USGS MODFLOW-2005 groundwater model (Harbaugh 2005), in which the turbulent hydraulic conductivity is estimated based on the laminar hydraulic conductivity (\textit{K${}_{lam}$}) and the Reynolds number at the start of turbulence (\textit{Re${}_{crit}$}) (Halford 2000):

\noindent 
\begin{equation} \label{ZEqnNum218388} 
\begin{array}{c} {u_{gw} =-K\left(Re\right)\frac{\partial H}{\partial x} {\rm \; \; \; in\; which\; }Re=\frac{\left|U\right|D_{50} }{n_{p} v} } \\ {K\left(Re\right)=\left\{\begin{array}{l} {K_{lam} \sqrt{\frac{Re_{crit} }{Re} } {\rm \; \; \; if\; \; }Re{\rm \; >\; }Re_{crit} {\rm \; }} \\ {K_{lam} {\rm \; \; \; \; \; \; \; \; \; \; \; \; \; \; \; \; if\; \; }Re\le {\rm \; }Re_{crit} {\rm \; }} \end{array}\right. } \end{array} 
\end{equation} 


\noindent In \eqref{ZEqnNum218388} the Reynolds number (\textit{Re}) is calculated using the median grain size (\textit{D${}_{5}$}${}_{0}$), the kinematic viscosity of water (\textit{$\nu $}) and the groundwater velocity in the pores (\textit{U/n${}_{p}$}), where \textit{n${}_{p}$} is the porosity. Similar expressions exist for the other two components of the groundwater flow. 

\noindent 

\noindent The critical Reynolds number for the start of turbulence (\textit{Re${}_{crit}$}) is specified by the user, based on in-situ or laboratory measurements, or expert judgment (keyword: \textit{gwReturb}). Since the hydraulic conductivity in the turbulent regime is dependent on the local velocity, an iterative approach is taken to find the correct hydraulic conductivity and velocity. 


\paragraph{ Determination of the groundwater head}

\noindent The XBeach groundwater model allows two methods to determine the groundwater head: a hydrostatic approach (keyword: \textit{gwnonh = 0}) and a non-hydrostatic approach (keyword: \textit{gwnonh = 1}).

\noindent 

\noindent \textit{Hydrostatic approach}

\noindent In the hydrostatic approach, the groundwater head is computed as follows:

\noindent 

\begin{enumerate}
\item  In cells where there is no surface water the groundwater head is set equal to the groundwater surface level \textit{$\eta $${}_{gw.}$}

\item  In cells where there is surface water, but the groundwater surface level \textit{$\eta $${}_{gw}$} is more than \textit{d${}_{wetlayer }$}(keyword: \textit{dwetlayer}) below the surface of the bed, the groundwater head is set equal to the groundwater surface level.

\item  In cells where there is surface water and the groundwater surface level \textit{$\eta $${}_{gw}$} is equal to the surface of the bed, the groundwater head is set equal to the surface water level.

\item  In cells where there is surface water and the groundwater surface level \textit{$\eta $${}_{gw}$} is equal to or less than \textit{d${}_{wetlayer}$} below the surface of the bed, the groundwater head is linearly weighted between that of the surface water level and the groundwater level, according to the distance from the groundwater surface to the surface of the bed. 
\end{enumerate}

\noindent 

\noindent It should be noted that the numerical parameter \textit{d${}_{wetlayer}$} is required to ensure numerical stability of the hydrostatic groundwater model. Larger values of \textit{d${}_{wetlayer}$} will increase numerical stability, at the expense of numerical accuracy. 

\noindent 

\noindent \textit{Non-hydrostatic approach}

\noindent Groundwater flow in the swash and surf zone has been shown to be non-hydrostatic (e.g., Li and Barry 2000; Lee et al. 2007). In order to capture this, it may be necessary in certain cases to reject the Dupuit--Forchheimer assumption of hydrostatic groundwater pressure. 

\noindent \textit{}

\noindent In the non-hydrostatic approach, the groundwater head is not assumed to be constant in the vertical. Since XBeach is depth-averaged, the model cannot compute true vertical profiles of the groundwater head and velocity. In order to estimate of the groundwater head variation over the vertical, a quasi-3D modeling approach is applied, which is set by two boundary conditions and one non-hydrostatic shape assumption:

\noindent 

\begin{enumerate}
\item  There is no exchange of groundwater between the aquifer and the impermeable layer below the aquifer.

\item  The groundwater head at the upper surface of the groundwater is continuous with the head applied at the groundwater surface.

\item  The shape of the non-hydrostatic head profile is parabolic (keyword: \textit{gwheadmodel = parabolic}), implying that the vertical velocity increases or decreases linearly from the bottom of the aquifer to the upper surface of the groundwater, or the non-hydrostatic head profile is hyperbolic (keyword: \textit{gwheadmodel = exponential}), cf., Raubenheimer et al. (1998).
\end{enumerate}

\noindent 

\noindent The vertical groundwater head approximation can be solved for the three imposed conditions by a vertical head function, shown here for the parabolic head assumption. The depth-average value of the groundwater head is used to calculate the horizontal groundwater flux and is found by integrating the groundwater head approximation over the vertical:

\noindent 
\begin{equation} \label{ZEqnNum576254} 
\overline{H}=\frac{1}{h_{gw} } \int _{0}^{h_{gw} }H(\sigma )d\sigma =H_{bc} -\frac{2}{3} \beta h_{gw}^{1}   
\end{equation} 


\noindent In \eqref{ZEqnNum576254} the mean vertical ground water head (\textit{H}) is calculated using the groundwater head imposed at the groundwater surface (\textit{H${}_{bc}$}), the groundwater head parabolic curvature coefficient (\textit{$\beta $}) and the height of the groundwater level above the bottom of the aquifer (\textit{h${}_{gw}$}). 

\noindent 

\noindent The unknown curvature coefficient (\textit{$\beta $}) in the vertical groundwater head approximation \eqref{ZEqnNum576254} is solved using the coupled equations for continuity \eqref{ZEqnNum113258} and motion \eqref{ZEqnNum953353}, thereby producing the depth-average horizontal groundwater head gradients and vertical head gradients at the groundwater surface.

\noindent 

\noindent Although the requirement for non-hydrostatic pressure has the benefit of being a more accurate representation of reality, and does not require the numerical smoothing parameter \textit{d${}_{wetlayer}$}, resolving the non-hydrostatic pressure field can be computationally expensive, particularly in 2DH applications.


\paragraph{ Exchange with surface water}

\noindent In the groundwater model there are three mechanisms for the vertical exchange of groundwater and surface water: 1) submarine exchange, 2) infiltration and 3) exfiltration. The rate of exchange between the groundwater and surface water (\textit{S}) is given in terms of surface water volume, and is defined positive when water is exchanged from the surface water to the groundwater. 

\noindent 

\noindent Infiltration and exfiltration can only occur in locations where the groundwater and surface water are not connected. Infiltration takes place when surface water covers an area in which the groundwater level is lower than the bed level. The flux of surface water into the bed is related to the pressure gradient across the wetting front.

\noindent 
\begin{equation} \label{ZEqnNum418429} 
\begin{array}{c} {S_{inf} =K\left(\frac{1}{\rho g} \frac{\left. p\right|^{z=\xi } }{\delta _{infill} } 1\right)} \\ {{\rm in\; which\; }\delta _{infill} (t)=\int \frac{S}{n_{p} } dt } \end{array} 
\end{equation} 
In \eqref{ZEqnNum418429} the surface water-groundwater exchange flow of infiltration (\textit{S${}_{inf}$}) is calculated using the effective hydraulic conductivity (\textit{K}), the surface water pressure at the bed ($\left. p\right|^{z=\xi } $) and the thickness of the wetting front (\textit{$\delta _{infill} $}).

\noindent 

\noindent Since the groundwater model is depth-averaged and cannot track multiple layers of groundwater infiltrating into the bed, the wetting front thickness is reset to zero when there is no available surface water, the groundwater exceeds the surface of the bed, or the groundwater and the surface water become connected. In addition, all infiltrating surface water is instantaneously added to the groundwater volume, independent of the distance from the bed to the groundwater table. Since the groundwater model neglects the time lag between infiltration at the beach surface and connection with the groundwater table a phase error may occur in the groundwater response to swash dynamics

\noindent 

\noindent Exfiltration (\textit{S${}_{exf}$}) occurs where the groundwater and surface water are not connected and the groundwater level exceeds the bed level. The rate of exfiltration is related to the rate of the groundwater level exceeding the bed level.

\noindent 
\begin{equation} \label{2.71)} 
S_{exf} =n_{p} \frac{\delta (z_{b} -\eta _{gw} )}{\delta t}  
\end{equation} 


\noindent Submarine exchange (\textit{S${}_{sub}$}) represents the high and low frequency infiltration and exfiltration through the bed due pressure gradients across the saturated bed. This process only takes place where the groundwater and surface water are connected. In the case of the non-hydrostatic groundwater model, the rate of submarine exchange is determined by the vertical specific discharge velocity at the interface between the groundwater and surface water. The value of this velocity can be found using the vertical derivative of the approximated groundwater head at the groundwater-surface water interface (shown for the parabolic head approximation).

\noindent 
\begin{equation} \label{2.72)} 
S_{sub} =2\beta h_{gw} K 
\end{equation} 


\noindent In the case of the hydrostatic groundwater model, the difference between the surface water head and the groundwater head is used to drive submarine discharge when the groundwater level is less than \textit{d${}_{wetlayer}$} from the bed surface. 

\noindent 

\noindent While most beach systems can acceptably described through vertical exchange of surface water and groundwater, in cases of very steep permeable slopes (e.g., porous breakwaters), it is necessary to include the horizontal exchange of groundwater and surface water between neighboring cells (keyword: \textit{gwhorinfil = 1}). In this case the horizontal head gradient between the surface water and groundwater across vertical interface between the cells is used to determine the horizontal exchange flux:

\noindent 
\begin{equation} \label{2.73)} 
S_{hor} =-K\frac{\partial H_{s} }{\partial s} A 
\end{equation} 


\noindent where \textit{$\delta $H${}_{s}$} is the head gradient between the surface water and groundwater in neighboring cell, \textit{$\delta $s} is the gradient distance, defined as the numerical grid size, and \textit{A} is the surface area through which the exchange takes place, defined as the difference in bed level between the neighboring cells.


\paragraph{ Calculation of groundwater and surface water levels}

\noindent Groundwater levels are updated through the continuity relation:

\noindent 
\begin{equation} \label{2.74)} 
n_{p} \frac{\delta \eta _{gw} }{\delta t} =-\frac{\partial h_{gw} u_{gw} }{\partial x} -\frac{\partial h_{gw} v_{gw} }{\partial y} +S_{inf} +S_{exf} +S_{sub} +S_{hor}  
\end{equation} 


\noindent In these same areas the surface water level is modified to account for exchange fluxes:

\noindent 
\begin{equation} \label{2.75)} 
\frac{\delta \eta }{\delta t} =-S_{inf} -S_{exf} -S_{sub} -S_{hor}  
\end{equation} 


\paragraph{ Boundary conditions}

\noindent Since the groundwater dynamics are described by a parabolic equation, the system of equations requires boundary conditions at all horizontal and vertical boundaries, as well as an initial condition:

\begin{enumerate}
\item  A zero flux condition is imposed at the horizontal boundaries and bottom of the aquifer. 

\item  The initial condition for the solution is specified by the model user in terms of the initial groundwater head (keyword: \textit{gw0}, or \textit{gw0file}).
\end{enumerate}


\subsection{ Sediment transport}


\subsubsection{ Advection-diffusion equation}

\noindent Sediment concentrations in the water column are modeled using a depth-averaged advection-diffusion scheme with a source-sink term based on equilibrium sediment concentrations (Galappatti and Vreugdenhil, 1985):

\noindent 
\begin{equation} \label{ZEqnNum934820} 
\frac{\partial hC}{\partial t} +\frac{\partial hCu^{E} }{\partial x} +\frac{\partial hCv^{E} }{\partial y} +\frac{\partial }{\partial x} \left[D_{h} h\frac{\partial C}{\partial x} \right]+\frac{\partial }{\partial y} \left[D_{h} h\frac{\partial C}{\partial y} \right]=\frac{hC_{eq} -hC}{T_{s} }  
\end{equation} 


\noindent In \eqref{ZEqnNum934820} \textit{C} represents the depth-averaged sediment concentration which varies on the wave-group time scale and \textit{D${}_{h}$} is the sediment diffusion coefficient. The entrainment of the sediment is represented by an adaptation time \textit{T${}_{s}$}, given by a simple approximation based on the local water depth \textit{h} and sediment fall velocity \textit{w${}_{s}$}. A small value of \textit{T${}_{s}$} corresponds to nearly instantaneous sediment response (keyword: \textit{Tsmin}). The factor \textit{f${}_{Ts}$} is a correction and calibration factor to take into account the fact that \textit{w${}_{s}$} is determined on depth-averaged data (keyword: \textit{tsfac}).

\noindent 
\begin{equation} \label{2.77)} 
T_{s} =\max \left(f_{Ts} \frac{h}{w_{s} } ,T_{s,\min } \right)\;  
\end{equation} 


\noindent The entrainment or deposition of sediment is determined by the mismatch between the actual sediment concentration \textit{C} and the equilibrium concentration \textit{C${}_{eq}$ }thus representing the source term in the sediment transport equation. 


\subsubsection{ General parameters}

\noindent In the sediment transport formulations, the equilibrium sediment concentration \textit{C${}_{eq}$}${}_{ }$(for both the bed load and the suspended load) is related to the velocity magnitude (\textit{v${}_{mg}$}), the orbital velocity (\textit{u${}_{rms}$}) and the fall velocity (\textit{w${}_{s}$}). This section elaborates how these are calculated.   Important to note: XBeach calculates the equilibrium concentration for the bed and suspended load \textit{separately.}  

\noindent 

\noindent First of all the Eulerian \underbar{velocity magnitude}, if long wave stirring is turned on (keyword: \textit{lws = 1}), the velocity magnitude \textit{v${}_{mg}$} is equal to the magnitude of the Eulerian velocity, as can be seen in \eqref{ZEqnNum872304}.

\noindent 
\begin{equation} \label{ZEqnNum872304} 
v_{mg} =\sqrt{\left(u^{E} \right)^{2} +\left(v^{E} \right)^{2} }  
\end{equation} 


\noindent If wave stirring is turned off (\textit{keyword: lws = 0}), the velocity magnitude will be determined by two terms: first of all a factor of the velocity magnitude of the previous time step (v${}_{mg}$${}^{n-1}$) and secondly a current-averaged part. Averaging will be carried out based on a certain factor \textit{f${}_{cats}$} (keyword: \textit{cats}) of the representative wave period \textit{T${}_{rep}$.}

\noindent 
\begin{equation} \label{2.79)} 
v_{mg} =\left(1-\frac{dt}{f_{cats} T_{rep} } \right)v_{mg}^{n-1} _{} +\frac{dt}{f_{cats} T_{rep} } \sqrt{\left(u^{E} \right)^{2} +\left(v^{E} \right)^{2} }  
\end{equation} 


\noindent Secondly, the \underbar{root-mean-squared velocity}, the \textit{u${}_{rms}$} is obtained from the wave group varying wave energy using linear wave theory. In this formulation \textit{T${}_{rep}$} is the representative wave period and the \textit{H${}_{rms}$} is the root-mean-square wave height. In this equation the water depth is enhanced with a certain factor of the wave height (keyword: \textit{delta}). 

\noindent 
\begin{equation} \label{2.80)} 
u_{rms} =\frac{\pi H_{rms} }{T_{rep} \sqrt{2} \sinh (k(h+\delta H_{rms} )}  
\end{equation} 


\noindent To account for wave breaking induced turbulence due to short waves, the orbital velocity is adjusted (van Thiel de Vries, 2009). In this formulation \textit{k${}_{b}$} is the wave breaking induced turbulence due to short waves. The turbulence is approximated with an empirical formulation in XBeach.

\noindent 
\begin{equation} \label{2.81)} 
u_{rms,2}^{2} =u_{rms}^{2} +1.45k_{b}  
\end{equation} 


\noindent Thirdly, the \underbar{fall velocity}, the \textit{w${}_{s}$} is calculated using the formulations of Ahrens (2000) which are derived based on a relationship suggested by Hallermeier (1981):

\noindent 
\begin{equation} \label{2.82)} 
w_{s} =\alpha _{1} \sqrt{\Delta gD_{50} } +\alpha _{2} \frac{\Delta gD_{50}^{2} }{\nu }  
\end{equation} 
\begin{equation} \label{2.83)} 
\alpha _{1} =1.06\tanh \left(0.016A^{0.50} \exp \left(-120/A\right)\right) 
\end{equation} 
\begin{equation} \label{2.84)} 
\alpha _{2} =0.055\tanh \left(12A^{-0.59} \exp \left(-0.0004A\right)\right) 
\end{equation} 
For high sediment concentrations, the fall velocity is reduced (keyword: \textit{fallvelred = 1}) using the expression of Richardson and Zaki (1954):

\noindent 
\begin{equation} \label{2.85)} 
w_{s,reduced} =\left(1-C\right)^{\alpha } w_{s}  
\end{equation} 


\noindent The exponent a is estimated using the equation of Rowe (1987), which depends purely on the Reynolds particle number R:

\noindent 
\begin{equation} \label{2.86)} 
\alpha =2.35\frac{2+0.175{\rm R}^{3/4} }{1+0.175{\rm R}^{3/4} }  
\end{equation} 
\begin{equation} \label{2.87)} 
{\rm R}=\frac{w_{s} D_{50} }{\nu }  
\end{equation} 


\subsubsection{ Transport formulations}

\noindent In the present version of XBeach, two sediment transport formulations are available. The formulae of the two formulations are presented in the following sections. For both methods the total equilibrium sediment concentration is calculated with \eqref{ZEqnNum890759}. In this equation the minimum value of the equilibrium sediment concentration (for both bed load en suspended load) compared to the maximum allowed sediment concentration (keyword: \textit{cmax}).  

\noindent 
\begin{equation} \label{ZEqnNum890759} 
C_{eq} ={\rm max\; }({\rm min\; }(C_{eq,b} ,\frac{1}{2} C_{\max } )+{\rm min\; }(C_{eq,s} ,\frac{1}{2} C_{\max } ),{\rm \; }0) 
\end{equation} 


\noindent The transport formulations implemented into XBeach distinguishes bed load and suspended load transport. It is possible to in- and exclude these transports components (keywords: \textit{bed} \& \textit{sus}, with bed = 1 one will include bed load transport). There is also a possibility to compute the total bulk transport rather than bed and suspended load separately (keyword: \textit{bulk = 1}). The bed load will be calculated if it is suspended transport. On top of that this switch will have impact on how the bed slope effect (see Section 2.7.6) will be calculated 


\paragraph{ Soulsby-Van Rijn}

\noindent The first possible sediment transport formulation are the Soulsby-Van Rijn equations (keyword: \textit{form = soulsby\_vanrijn}) (Soulsby, 1997; van Rijn, 1984). The equilibrium sediment concentrations are calculated according to:

\noindent 
\begin{equation} \label{2.89)} 
\begin{array}{l} {C_{eq,b} =\frac{A_{sb} }{h} \left(\sqrt{v_{mg} ^{2} +0.018\frac{u_{rms,2} ^{2} }{C_{d} } } -U_{cr} \right)^{2.4} } \\ {C_{eq,s} =\frac{A_{ss} }{h} \left(\sqrt{v_{mg} ^{2} +0.018\frac{u_{rms,2} ^{2} }{C_{d} } } -U_{cr} \right)^{2.4} } \end{array} 
\end{equation} 


\noindent For which the bed load and suspended load coefficients are calculated with:

\noindent 
\begin{equation} \label{2.90)} 
A_{sb} =0.005h\left(\frac{D_{50} }{h\Delta gD_{50} } \right)^{1.2} ,{\rm \; }A_{ss} =0.012D_{50} \frac{D_{*}^{-0.6} }{(\Delta gD_{50} )^{1.2} }  
\end{equation} 
In which the dimensionless sediment diameter (D*) can be calculated with the following formulation. The \textit{v} is the kinematic viscosity based on the expression of Van Rijn and is a function of the water temperature. XBeach assumes a constant temperature of 20 degrees Celsius, this result in a constant kinematic viscosity of 10${}^{-6}$ m${}^{2}$/s.

\noindent 
\begin{equation} \label{2.91)} 
D_{*} =\left(\frac{\Delta g}{\nu ^{2} } \right)^{1/3} D_{50}  
\end{equation} 


\noindent The critical velocity (\textit{U${}_{cr}$}) defines at which depth averaged velocity sediment motion is initiated:

\noindent 
\begin{equation} \label{2.92)} 
U_{cr} =\left\{\begin{array}{l} {0.19D_{50}^{0.1} \log 10\left(\frac{4h}{D_{90} } \right){\rm \; for\; }D_{50} \le 0.0005} \\ {8.5D_{50}^{0.6} \log 10\left(\frac{4h}{D_{90} } \right){\rm \; for\; }D_{50} >0.05} \end{array}\right.  
\end{equation} 


\noindent Finally the drag coefficient (\textit{C${}_{d}$}) is calculated with equation \eqref{ZEqnNum280790}. A drag coefficient is used to determine the equilibrium sediment concentrations. On top of that Souslby (1997) gives a relation between the bed shear stress of the depth-averaged current speed. 

\noindent 
\begin{equation} \label{ZEqnNum280790} 
C_{d} =\left(\frac{0.40}{\ln \left(\frac{\max (h,10z_{0} )}{z_{0} } \right)-1} \right)^{2}  
\end{equation} 


\noindent In this equation z0 is used for the bed roughness length and is used as zero flow velocity level in the formulation of the sediment concentration. In XBeach this is a fixed value (keyword: z0), but Soulsby (1997) argues there is a relation between the Nikuradse and kinematic viscosity.  $\; $  


\paragraph{ Van Thiel-Van Rijn}

\noindent The second possible sediment transport formulation are the Van Thiel-Van Rijn transport equations (keyword: \textit{form = vanthiel\_vanrijn}) (van Rijn, 2007; van Thiel de Vries, 2009). The major difference between the Soulsby -- Van Rijn equations is twofold. First of all, there is no drag coefficient calculated anymore and secondly the critical velocity is determined by calculating separately the critical velocity for currents (\textit{U${}_{crc}$}) according to Shields (1936) and for waves (\textit{U${}_{crw}$}) according to Komen and Miller (1975).

\noindent 

\noindent The equilibrium sediment concentrations are calculated according to

\noindent 
\begin{equation} \label{2.94)} 
\begin{array}{l} {C_{eq,b} =\frac{A_{sb} }{h} \left(\sqrt{v_{mg} ^{2} +0.64u_{rms,2}^{2} } -U_{cr} \right)^{1.5} } \\ {C_{eq,s} =\frac{A_{ss} }{h} \left(\sqrt{v_{mg} ^{2} +0.64u_{rms,2}^{2} } -U_{cr} \right)^{2.4} } \end{array} 
\end{equation} 


\noindent For which the bed-load and suspended load coefficient are calculated with:

\noindent 
\begin{equation} \label{2.95)} 
A_{sb} =0.015h\frac{\left(D_{50} /h\right)^{1.2} }{\left(\Delta gD_{50} \right)^{0.75} } ,{\rm \; }A_{ss} =0.012D_{50} \frac{D_{*}^{-0.6} }{(\Delta gD_{50} )^{1.2} }  
\end{equation} 


\noindent The critical velocity is computed as weighted summation of the separate contributions by currents and waves (Van Rijn, 2007):

\noindent 
\begin{equation} \label{2.96)} 
U_{cr} =\beta U_{crc} +(1-\beta )U_{crw} {\rm \; \; in\; which\; \; \; }\beta =\frac{v_{mg} }{v_{mg} +u_{rms} }  
\end{equation} 


\noindent The critical velocity for currents is based on Shields (1936):

\noindent 
\begin{equation} \label{2.97)} 
U_{crc} =\left\{\begin{array}{l} {0.19D_{50}^{0.1} \log 10\left(\frac{4h}{D_{90} } \right){\rm \; \; \; \; \; for\; }D_{50} \le 0.0005} \\ {8.5D_{50}^{0.6} \log 10\left(\frac{4h}{D_{90} } \right){\rm \; \; \; \; \; \; \; for\; }D_{50} \le 0.002} \\ {1.3\sqrt{\Delta gD_{50} } \left(\frac{h}{D_{50} } \right)^{1/6} {\rm \; \; \; \; \; \; \; for\; }D_{50} >0.0005} \end{array}\right.  
\end{equation} 


\noindent The critical velocity for waves is based on Komer and Miller (1975):

\noindent 
\begin{equation} \label{2.98)} 
U_{crw} =\left\{\begin{array}{l} {0.24(\Delta g)^{2/3} {\rm \; }\left(D_{50} T_{rep} \right)^{1/3} {\rm \; \; \; \; \; \; \; \; for\; \; }D_{50} <=0.0005} \\ {0.95(\Delta g)^{0.57} {\rm \; }\left(D_{50} \right)^{0.43} T_{rep} ^{0.14} {\rm \; \; for\; \; }D_{50} >0.0005} \end{array}\right.  
\end{equation} 


\subsubsection{ Effects of wave nonlinearity }

\noindent Effects of wave skewness and asymmetry are accounted for in the advection-diffusion equation, repeated here:

\noindent 
\begin{equation} \label{2.99)} 
\begin{array}{c} {\frac{\partial hC}{\partial t} +\frac{\partial hC(u^{E} -u_{a} \sin \theta _{m} )}{\partial x} +\frac{\partial hC(v^{E} -u_{a} \cos \theta _{m} )}{\partial y} } \\ {+\frac{\partial }{\partial x} \left[D_{h} h\frac{\partial C}{\partial x} \right]+\frac{\partial }{\partial y} \left[D_{h} h\frac{\partial C}{\partial y} \right]=\frac{hC_{eq} -hC}{T_{s} } } \end{array} 
\end{equation} 


\noindent XBeach considers the wave energy of short waves as averaged over their length, and hence does not simulate the wave shape. A discretization of the wave skewness and asymmetry was introduced by Van Thiel de Vries (2009), to affect the sediment advection velocity. In this equation \textit{u${}_{a}$} is calculated as function of wave skewness (\textit{S${}_{k}$}), wave asymmetry parameter (\textit{A${}_{s}$}), root-mean square velocity \textit{u${}_{rms}$} and two calibration factor \textit{f${}_{Sk}$} and\textit{ f${}_{As}$} (keyword: \textit{facSk \& facAs}), see equation \eqref{ZEqnNum326040}. To set both values one can use the keyword: \textit{facua}. The method to determine the skewness and asymmetry is described in section 2.3.4. A higher value for \textit{u${}_{a}$} will simulate a stronger onshore sediment transport component. 

\noindent 
\begin{equation} \label{ZEqnNum326040} 
u_{a} =(f_{Sk} S_{k} -f_{As} A_{s} )u_{rms}  
\end{equation} 


\subsubsection{ Hindered erosion by dilatancy}

\noindent Under overwash and breaching conditions (high flow velocities and large bed level variations in time), dilatancy might hinder the erosion rates (De Vet, 2014). To account for this effect, the theory of Van Rhee (2010) could be applied (keyword: \textit{dilatancy = 1}), reducing the critical Shields parameter at high flow velocities:

\noindent 
\begin{equation} \label{2.101)} 
\theta _{cr}^{adjusted} =\theta _{cr} \left(1+\frac{v_{e} }{k_{l} } \frac{n_{l} -n_{0} }{1-n_{l} } \frac{A}{\Delta } \right) 
\end{equation} 


\noindent In this equation, \textit{v${}_{e}$} refers to the erosion velocity, \textit{k${}_{l}$} is the permeability, \textit{n0} is the porosity prior, \textit{n${}_{l}$} is the porosity in the sheared zone (keyword \textit{pormax}) and the parameter \textit{A} (keyword \textit{rheeA}) is equal to 3/4 for single particles and approximately 1.7 for a continuum.

\noindent 

\noindent The larger the permeability of the bed, the smaller the dilatancy effect. Van Rhee (2010) suggests using the equation proposed by Den Adel (1987):

\noindent 
\begin{equation} \label{2.102)} 
k_{l} =\frac{g}{160\nu } D_{15}^{2} \frac{n_{0}^{3} }{\left(1-n_{0}^{2} \right)}  
\end{equation} 


\noindent Finally, the erosion velocity \textit{v${}_{e}$}, is the velocity at which the bottom level decreases:

\noindent 
\begin{equation} \label{2.103)} 
v_{e} =\left\{\begin{array}{l} {-\frac{dz_{b} }{dt} {\rm \; \; \; \; \; \; \; \; \; \; if\; \; \; \; \; \; \; \; \; \; \; }\frac{dz_{b} }{dt} <0} \\ {{\rm \; \; \; \; }0{\rm \; \; \; \; \; \; \; \; \; \; \; else}\; \; \; \; \; \; \; \; \; \; \; } \end{array}\right.  
\end{equation} 


\subsubsection{ Bed slope effect}

\noindent The bed slope affects the sediment transport in various ways (Walstra, 2007):

\noindent 

\begin{enumerate}
\item  The bed slope influences the local near-bed flow velocity;

\item  The bed slope may change the transport rate once the sediment is in motion;

\item  The bed slope may change the transport direction once the sediment is in motion;

\item  The bed slope will change the threshold conditions for initiation of motion.
\end{enumerate}

\noindent 

\noindent The influence of the bed slope on the local hydrodynamics is not considered in XBeach.

\noindent 

\noindent Two possible expressions are implemented to change the magnitude of the sediment transport. The first method is the default one in XBeach:

\noindent 
\begin{equation} \label{2.104)} 
\begin{array}{l} {q_{x,slope} =q_{x} -\alpha hC\sqrt{\left(u^{L} \right)^{2} +\left(v^{L} \right)^{2} } \frac{\partial z_{b} }{\partial x} } \\ {q_{y,slope} =q_{y} -\alpha hC\sqrt{\left(u^{L} \right)^{2} +\left(v^{L} \right)^{2} } \frac{\partial z_{b} }{\partial y} \; \; \; \; \; \; \; } \end{array} 
\end{equation} 


\noindent This method could be applied on either the total sediment transport (keyword: \textit{bdslpeffmag = roelvink\_total}) or only on the bed load transport (keyword: \textit{bdslpeffmag = roelvink\_bed}). The second method is based on the engineering formula of Soulsby (1997):

\noindent 
\begin{equation} \label{2.105)} 
q_{slope} =q\left(1-\alpha \frac{\partial z_{b} }{\partial s} \right)\;  
\end{equation} 


\noindent Also this method could be applied on the total transport (keyword: \textit{bdslpeffmag = soulsby\_total}) or on the bed load transport only (keyword: \textit{bdslpeffmag = soulsby\_total). }To change the direction of the bed load transport, the expressions of Van Bendegom (1947) and Talmon et al. (1995) could be used (keyword: \textit{bdslpeffdir = talmon):}

\noindent 
\begin{equation} \label{2.106)} 
\tan \left(\alpha _{\psi ,new} \right)=\frac{\sin \left(\alpha _{\psi } \right)-f\left(\theta \right)\frac{dz_{b} }{dy} }{\cos \left(\alpha _{\psi } \right)-f\left(\theta \right)\frac{dz_{b} }{dx} }  
\end{equation} 
\begin{equation} \label{2.107)} 
f\left(\theta \right)=\frac{1}{9\left(D_{50} /h\right)^{0.3} \theta ^{0.5} }  
\end{equation} 
\begin{equation} \label{2.108)} 
\begin{array}{l} {q_{b,x} =\left|q_{b} \right|\cos \left(\alpha _{\psi ,new} \right)} \\ {q_{b,y} =\left|q_{b} \right|\sin \left(\alpha _{\psi ,new} \right)} \end{array} 
\end{equation} 


\noindent Finally, it is possible to adjust the initiation of motion criteria for the total transport (keyword \textit{bdslpeffini = total}) or the bed load transport only (keyword \textit{bdslpeffini = bed}) through (Soulsby, 1997):

\noindent 
\begin{equation} \label{2.109)} 
\theta _{cr}^{adjusted} =\theta _{cr} \frac{\cos \left(\psi \right)\sin \left(\beta \right)+\sqrt{\cos ^{2} \left(\beta \right)\tan ^{2} \left(\phi _{i} \right)-\sin ^{2} \left(\psi \right)\sin ^{2} \left(\beta \right)} }{\tan \left(\phi _{i} \right)}  
\end{equation} 


\noindent In this equation is \textit{$\psi $ }the difference in angle between the flow direction and the on-slope directed vector, \textit{$\beta $} the bed slope and \textit{?${}_{i}$} the angle of repose. \textit{}

\noindent \textit{}

\noindent De Vet (2014) provides a detailed overview on how the bed slope and flow direction are calculated and how the bed slope effect is combined with the dilatancy concept if the adjustment to the initiation of motion is considered. 


\subsection{ Bottom updating}


\subsubsection{ Due to sediment fluxes}

\noindent Based on the gradients in the sediment transport the bed level changes according to:

\noindent 
\begin{equation} \label{ZEqnNum202334} 
\frac{\partial z_{b} }{\partial t} +\frac{f_{mor} }{\left(1-p\right)} \left(\frac{\partial q_{x} }{\partial x} +\frac{\partial q_{y} }{\partial y} \right)=0 
\end{equation} 


\noindent In \eqref{ZEqnNum202334} \textit{$\rho $} is the porosity, \textit{f${}_{mor}$} (keyword: \textit{morfac}) is a morphological acceleration factor of O(1-10) (Reniers et al., 2004) and \textit{q${}_{x}$} and \textit{q${}_{y}$} represent the sediment transport rates in x- and y-direction respectively. Sediment transport can be activated with the keyword: \textit{sedtrans}. 

\noindent 

\noindent The morphological acceleration factor speeds up the morphological time scale relative to the hydrodynamic timescale. It means that if you have a simulation of 10 minutes with a \textit{morfac} of 6 you effectively simulate the morphological evolution over one hour. There are now two ways in which you can input the time-varying parameters in combination with \textit{morfac}:

\noindent 

\begin{enumerate}
\item  All times are prescribed on input in morphological time. If you apply a \textit{morfac} all input time series and other time parameters are divided internally by \textit{morfac}. This is determined with keyword\textit{ morfacopt = 1}. If you now specify a \textit{morfac} of 6, the model just runs for 10 (hydrodynamic) minutes each hour, during which the bottom changes per step are multiplied by a factor 6. This of course saves a factor of 6 in computation time.
\end{enumerate}

\noindent 

\noindent This method is appropriate for short-term simulations with extreme events. This approach is only valid as long as the water level changes that are now accelerated by \textit{morfac} do not modify the hydrodynamics too much. This is the case if the tide is perpendicular to the coast and the vertical variations do not lead to significant currents. If you have an alongshore tidal current, as is the case in shallow seas, you cannot apply this method because you would affect the inertia terms and thus modify the tidal currents.

\noindent 

\begin{enumerate}
\item  Alternatively you run the model over, say, over a tidal cycle, and apply the \textit{morfac} without modifying the time parameters. This means you leave all the hydrodynamic parameters unchanged and just exaggerate what happens within a tidal cycle. As long as the evolution over a single tidal cycle is limited, the mean evolution over a tidal cycle using a morfac is very similar to running \textit{morfac} tidal cycles without \textit{morfac}. See Roelvink (2006) for a more detailed description of this approach. This option is enabled with keyword:\textit{ morfacopt = 0}. 
\end{enumerate}

\noindent 

\noindent This method is more appropriate for longer-term simulations with not too extreme events.


\subsubsection{ Avalanching}

\noindent To account for the slumping of sandy material from the dune face to the foreshore during storm-induced dune erosion avalanching (keyword: \textit{avalanching}) is introduced to update the bed evolution. Avalanching is introduced via the use of a critical bed slope for both the dry and wet area (keyword: \textit{wetslp} and \textit{dryslp}). It is considered that inundated areas are much more prone to slumping and therefore two separate critical slopes for dry and wet points are used. The default values are 1 and 0.3 respectively. When this critical slope is exceeded, material is exchanged between the adjacent cells to the amount needed to bring the slope back to the critical slope. 

\noindent 
\begin{equation} \label{2.110)} 
\left|\frac{\partial z_{b} }{\partial x} \right|>m_{cr}  
\end{equation} 


\noindent To prevent the generation of large shockwaves due to sudden changes of the bottom level, bottom updating due to avalanching has been limited to a maximum speed of v\textit{${}_{av,}$}${}_{max}$ (keyword: \textit{dzmax}). Equation \eqref{ZEqnNum441314} shows the resulting bed level change within one time step. 

\noindent 
\begin{equation} \label{ZEqnNum441314} 
\begin{array}{l} {\Delta z_{b} =\min \left(\, \, \, \, \left(\left|\frac{\partial z_{b} }{\partial x} \right|-m_{cr} \right)\Delta x,\, \, \, \, v_{av,\max } \Delta t\right)\quad ,\frac{\partial z_{b} }{\partial x} >0} \\ {\Delta z_{b} =\max \left(-\left(\left|\frac{\partial z_{b} }{\partial x} \right|-m_{cr} \right)\Delta x,-v_{av,\max } \Delta t\right)\quad ,\frac{\partial z_{b} }{\partial x} <0} \end{array} 
\end{equation} 


\subsubsection{ Bed composition}

\noindent If the effect of different sediment fractions, sorting and armoring are of importance, a bed composition constituting multiple sediment fractions can be defined. Each sediment fraction is characterized by a median grain size (\textit{D${}_{50}$}) and possible a \textit{D${}_{15}$} and \textit{D${}_{90}$} as well. When using multiple sediment fractions, multiple bed layers are needed as well to describe the vertical distribution of the sediment fractions in the bed.

\noindent By specifying multiple bed layers, XBeach can keep track of the different sediment fractions both in the horizontal and in the vertical. Coarse sediments may be deposited on top of fine sediment after which erosion of the coarse sediment is needed to expose the fine sediment again, effectively armoring the bed. Three types of bed layers are distinguished: 1) the top layers 2) the variable or ``breathing'' layer and 3) the bottom layers. The top layer is the only layer that interacts with the water column and can be eroded, but preserves its thickness. The bottom layers are layers of constant thickness that move with the top layer. A single variable or ``breathing'' layer is defined that adapts its thickness to the erosion and sedimentation of the bed. For example: if a grid cell is eroded, particular fractions of sediment are removed from the top layer, but the top layer preserves its thickness and thus it takes the same volume of sediment, likely of different composition than the eroded sediment, from the layer below. If this layer is a top layer as well, the thickness is preserved and again the same volume of sediment is taken form a lower bed layer. This continues until the variable or ``breathing'' layer is reached. This layer adapts its thickness to the amount of erosion. If the thickness of the layer becomes too small, the variable layer is merged with an adjacent bottom layer and a new bottom layer is defined underneath the existing ones to ensure a constant number of bed layers. Reversely, if a grid cell is accreting, the thickness of the variable layer will be increased and with sufficient increase the variable layer will be split in two effectively creating a new bottom layer. The lowest existing bottom layer is then discarded to ensure a constant number of bed layers. The ``breathing'' layer can be the upper or bottom layer in which case the top layer or bottom layer class does not exist. The thickness of the different layer classes can be set separately (keyword: \textit{dzg1, dzg2} and \textit{dzg3}) or at once (keyword: \textit{dzg}).

\noindent Each grid cell in XBeach holds its own sediment distribution and the sediment transport formulations are used differentiate between fractions. Therefore the distribution of sediment may change over time and processes like armoring and sorting can be simulated. Due to the shifting of sediment between bed layers numerical mixing of sediment occurs. Choosing bed layer thicknesses that are in balance with the expected erosion and deposition during the simulation should keep the numerical mixing to a minimum. A bed layer thickness that is too large will result in relatively uniform behavior, while a bed layer thickness that is too small will result in a lot of shifting and thus numerical mixing (Figure 2.8).

\noindent \includegraphics*[width=6.05in, height=3.27in, keepaspectratio=false]{image19}

\noindent \textit{Figure 2.8 Visualization of the diffusion that occurs when XBeach calculates sediment compositions. After sedimentation of fine sediment on top of coarser material it is uniformly mixed over the whole layer. Subsequent erosion erodes both the fines as the coarser material. To avoid this phenomenon, layers should not be too thick (van der Zwaag, 2014).}


\subsection{ Ship-induced wave motions}

\noindent A relatively new application field for XBeach is the generation and propagation of waves induced by sailing vessels. This functionality has been implemented recently (Zhou, 2013), and has currently been used in several studies (e.g. Zhou et al., 2014, De Jong et al., 2013), showing very good results.

\noindent 

\noindent For computing ship-induced waves the non-hydrostatic version of XBeach is used. A moving ship is represented as a pressure head that moves along a pre-defined track through the model domain. The ship is defined on a separate grid, where the ship draft is specified per grid point. Each computational time step the ship draft is interpolated from the ship grid to the global grid, where the ship volume is kept constant. Then the water pressure head in each global grid cell is updated based on the interpolated ship draft. By moving the pressure fields, the waves are generated and will propagate further through the global domain.

\noindent 

\noindent In Figure 2.9 an example XBeach setup for ship waves is shown. The ship track is user defined and can, for instance, be obtained from the Automatic Identification System (AIS) for marine traffic. In this example, the model results were compared with measurements taken at Bath, The Netherlands. A filtered time series of the measured and computed water level is shown in Figure 2.10. The time series was filtered to focus on the computation of the primary ship wave.

\noindent 

\noindent \includegraphics*[width=2.93in, height=2.58in, keepaspectratio=false, trim=0.48in 0.00in 0.27in 0.18in]{image20} \includegraphics*[width=2.97in, height=2.57in, keepaspectratio=false, trim=0.44in 0.01in 0.28in 0.18in]{image21}

\noindent \textit{Figure 2.9 Example XBeach setup (left) and result (right) for a ship wave simulation in the Scheldt Estuary (The Netherlands). The ship track (red dashed line) is user-defined, and the measurement location is indicated (magenta dot).}

\noindent 

\noindent \includegraphics*[width=4.27in, height=1.78in, keepaspectratio=false]{image22}

\noindent \textit{Figure 2.10 Example XBeach result for ship-induced waves. Measurements are taken at Bath, in the Scheldt Estuary, The Netherlands (Schroevers et al., 2011).}

\noindent 

\noindent In addition to the propagation of ship-induced waves, XBeach computes the forces and moments acting on the ship body. With this functionality, passing ship effects can be analyzed (e.g. Zhou, 2013).  

\noindent \eject 


\section{ Boundary conditions}


\subsection{ Waves}

\noindent XBeach allows users to include two different options for wave boundary conditions in the model. These wave boundary conditions can be applied only at the seaward boundary (keyword: \textit{wbctype}). First of all, in 3.1.1 the method to specify wave spectra is discussed. Secondly, in 3.1.2 the method to apply non-spectra, such as stationary wave conditions or time-series is elaborated. In 3.1.4 the lateral boundary conditions for waves are discussed. There is currently not a possibility to force waves on the landward boundary of a model. 


\subsubsection{ Spectral conditions}

\noindent The most-used wave boundary condition in XBeach is a spectral type. The input description of spectral wave boundary conditions can be found in 4.4.1.  XBeach allows the user to define these with three possibilities:

\noindent 

\begin{enumerate}
\item  \underbar{Parameterized spectrum}: With this option you define the boundary condition as parametric spectral input. The parameters (i.e. the spectral shape, the wave period and the directional spreading) can be specified. The option is especially handy when there is no nested model or measured spectrum. Here are two options:

\begin{enumerate}
\item  Specify a single parametric spectrum (keyword \textit{wbctype = jons}).

\item  Specify a series of parametric spectra (keyword \textit{wbctype = jons\_table}).
\end{enumerate}
\end{enumerate}

\noindent 

\begin{enumerate}
\item  \underbar{SWAN spectrum input}: In this case the two-dimensional (frequency-direction) output by the spectral wave model SWAN (.sp2 files) can be specified. (keyword \textit{wbctype = swan}). This option is especially convenient when nesting XBeach into a SWAN model.
\end{enumerate}

\noindent 

\begin{enumerate}
\item  \underbar{Formatted variance density spectrum}: In this case a more general type spectrum can be specified.  (keyword \textit{wbctype =vardens}). This option is often used when a measured spectrum is available.
\end{enumerate}


\subsubsection{ Non-spectral conditions}

\noindent XBeach also allows the user to define non-spectral wave boundary conditions. This is a variation of both wave conditions without wave groups and time series. The input description of non-spectral wave boundary conditions can be found in 4.4.2. XBeach allows the user to define these with two possibilities:

\noindent 

\begin{enumerate}
\item  \underbar{Stationary wave boundary condition.} This means that a uniform and constant wave energy is specified, based on the given values of H${}_{rms}$, T${}_{m01}$, direction and power of the directional distribution function. The station boundary condition will \underbar{not} contain wave groups. Here there are two options:

\begin{enumerate}
\item  Specify a single sea state (keyword \textit{wbctype = stat})

\item  Specify a series of sea states (keyword \textit{wbctype = stat\_table})
\end{enumerate}
\end{enumerate}

\noindent 

\begin{enumerate}
\item  \underbar{Time series of waves}. The user can also specify the variation in time of the wave energy. There are three options:

\begin{enumerate}
\item  First-order time series of waves (keyword \textit{wbctype = ts\_1}). XBeach will calculate the bound long wave based on the theory of Longuet-Higgins and Stewart (1964).

\item  Second-order time series of waves (keyword \textit{wbctype = ts\_2}). The bound long wave is specified by the user via a long wave elevation.

\item  It is also possible to specify a variation in time of the horizontal velocity, vertical velocity and the free surface elevation (keyword: \textit{wbctype = ts\_nonh}). Last two terms are optional in this boundary conditions type.  
\end{enumerate}
\end{enumerate}


\subsubsection{ Special conditions}

\noindent Besides clear spectral or non-spectral wave boundary conditions, there are also three special boundary condition types implemented in XBeach.

\noindent 

\begin{enumerate}
\item  \underbar{Bichromatic (two short-wave components) waves} (keyword \textit{wbctype = bichrom}). In this case, XBeach will be forced with regular wave groups as the two short-wave components force one difference (infragravity) wave period. The user needs to specify not only variables of the stationary situation but also a wave period for the long wave. This wave period will be used to calculate the long wave based on the theory of Longuet-Higgins and Stewart (1964). The bichromatic boundary condition is the most simplified form of a wave spectrum.

\item  \underbar{No wave boundary conditions} (keyword \textit{wbctype = off}). This is a simple no wave action boundary condition. It still allows for a tidal record to be specified, however this trough the zs0file parameter.

\item  \underbar{Reuse previous boundary conditions} (keyword: \textit{wbctype = reuse}). If the user does not wish to recalculate boundary condition files or specifically wants to reuse the boundary condition files of another XBeach simulation should be used. No further wave boundary condition data need be given. Obviously, the calculation grid should remain the same between runs, as the angles and number of grid points are embedded in the boundary condition files.
\end{enumerate}


\subsubsection{ Lateral boundary conditions}

\noindent There are two options to set the lateral boundaries for the wave model:

\begin{enumerate}
\item  Neumann boundaries (keyword: \textit{lateralwave = neumann)}: here the longshore gradient is set to zero.

\item  Wave crest boundaries (keyword: \textit{lateralwave = wavecrest)}. here the gradient in the wave energy along the wave crest is set to zero.
\end{enumerate}

\noindent 

\noindent For the stationary wave mode (keyword: \textit{wavemodel = stationary)} this is the only option. It allows a correct representation of the wave propagation near the lateral boundaries, without the usual shadow zones in e.g. SWAN. By neglecting the longshore gradients, the model automatically computes a consistent 1D solution. 

\noindent 

\noindent For the surfbeat mode (keyword: \textit{wavemodel = surfbeat}), Neumann leads to shadow zones, not so much in the wave height, but in the groupiness; the 'blobs' propagating in the mean wave direction turn into elongated, longshore uniform patches. To reduce this effect, the gradient along the wave crests of the wave energy can be set to zero, instead of the longshore gradient (keyword: \textit{lateralwave = wavecrest)}. This way the crests of the wave groups have approximately the right orientation, though the along-crest groupiness also disappears. In the \textit{wavecrest} case, the wave refraction may be overestimated leading to somewhat too large longshore currents. The effects of both boundary conditions are shown in Figure 3.1. 

\noindent \includegraphics*[width=5.86in, height=4.78in, keepaspectratio=false, trim=0.11in 0.00in 0.06in 0.00in]{image23}

\noindent \textit{Figure 3.1 Effect of the lateral wave boundary conditions on root-mean square wave height patterns (top) and longshore velocity (bottom) for the Delilah test case.  In this figure the left panels are used for simulations with Neumann boundaries and the right panel with the wavecrest boundary. }


\subsection{ Shallow water equations}


\subsubsection{ Offshore boundary}

\noindent Typically, an offshore or lateral boundary is an artificial boundary which has no physical meaning. On the offshore boundary wave and flow conditions are imposed. In the domain waves and currents will be generated which need to pass through the offshore boundary to the deep sea with minimal reflection. One way to do this is to impose a weakly reflective-type boundary condition (absorbing-generating), but there are also other possibilities implemented in XBeach (keyword: \textit{front}). This method can be applied in 1D or 2D, is recommended and therefore the default value for XBeach.

\noindent 

\noindent In XBeach, there are two options with regard to the offshore absorbing-generating boundary condition. With the parameter setting \textit{front = abs1d }a simple one-dimensional absorbing-generating boundary condition is activated. This option allows for a time-varying water level (surge and/or infragravity waves) to be specified at the boundary while allowing any waves propagating perpendicularly towards the boundary to be absorbed (i.e., passed through the boundary with a minimum of reflection. It is therefore only useful for 1D (flume like) simulations.

\noindent 

\noindent With option \textit{front = abs2d }(default value) the formulation by Van Dongeren and Svendsen (1997) is activated which in turn is based on Verboom et al. (1981) and is based on the `Method of Characteristics'. This boundary condition allows for obliquely-incident and obliquely-reflected waves to pass through the boundary. It is possible to account for situations with boundary-perpendicular and boundary-parallel currents. In order to differentiate between the particle velocities, the keyword \textit{epsi} must be set. This parameter control a simple Kalman-update filter which controls which part of the particle velocity is assumed to be part of the current and which part is wave-related. By default XBeach computes the value for epsi automatically using offshore boundary conditions (keyword: \textit{epsi = -1}).

\noindent 

\noindent There are three other possibilities implemented besides the absorbing-generating boundary conditions:

\begin{enumerate}
\item  No flux wall (keyword: \textit{front = wall}). This boundary condition type is a simple no flux boundary condition.

\item  Water level specification (keyword: \textit{front = wlevel}). This boundary sets the water level at a prescribed value. This can be constant or time-varying. With this option the outgoing long waves are not absorbed.

\item  Boundary condition for the non-hydrostatic option (keyword\textit{: front = nonh\_1d}). The user needs to provide a file containing time series for the velocity at the boundary.

\item  Radiation boundary condition (keyword: \textit{front = waveflume}). This boundary uses a continuity relation at the front boundary. This means that no net water can come into the model domain. The wave flume boundary condition is especially useful in lab experiments with a large set-up (e.g. coral reefs).
\end{enumerate}


\subsubsection{ Lateral boundaries}

\noindent Lateral boundaries are the boundaries perpendicular to the coastline. Usually these are artificial, because the model domain is limited but the physical coast will continue. At these boundaries (keywords: \textit{left} \& \textit{right}) we need to prescribe information about the area beyond the numerical model domain in such a way that the boundary condition does not influence the results in an adverse way. One way to do this is to prescribe a so-called ``no-gradient'' or Neumann boundaries (XBeach default), which state that there is locally no change in surface elevation and velocity, but there are also other possibilities implemented into XBeach. This method is recommended and is therefore the default value for XBeach. Each lateral boundary is a separate condition, so it is possible to mix different type of lateral boundary per side.

\noindent 

\noindent Neumann boundary conditions are activated where the longshore water level gradient is prescribed. The alongshore gradient is prescribed by the difference in specified water levels at the offshore corner points, divided by the alongshore length of the domain. This type of Neumann boundary condition has been shown to work quite well with (quasi-) stationary situations, where the coast can be assumed to be uniform alongshore outside the model domain. So far we have found that also in case of obliquely incident wave groups this kind of boundary conditions appears to give reasonable results when a shadow zone is taken into account. This means that regions where the boundary conditions are not fully enforced the results are not taken into account. Neumann boundaries can be individually defined (keyword: \textit{left = neumann).} 

\noindent 

\noindent There are three other possibilities implemented besides the absorbing-generating boundary conditions:

\begin{enumerate}
\item  Simple no-flux boundary conditions can also be applied (keyword: \textit{left = wall}). Wall boundary conditions will result in a zero velocity at the lateral boundary.

\item  Velocity at the boundary will be calculated from NLSWE, but only include the advective terms (keywords: \textit{left = no\_advec}). The effect is that only terms that decrease the velocity will be taken into account. The result is an intermediate form between a full Neumann boundary and a wall boundary. 

\item  Velocity at the boundary will simply be copied from the adjacent cell in the model domain (keyword: \textit{left = neumann\_v}). 
\end{enumerate}

\noindent 


\subsubsection{ Tide and surge}

\noindent XBeach can take in up to four time-vary tidal signals to be applied to the four boundaries (offshore-left, backshore-left, backshore-right, offshore-right).  A time-varying water level signal is read into XBeach by reading the specified file in zs0file. The input signal will be interpolated to the local time step of the simulation; therefore the signals only need to be long enough and temporally-fine enough to resolve the water level phenomenon of interest (i.e. tide variations, surge event).

\noindent 

\noindent There are now four options for handling the tidal and/or surge contribution to the boundaries:

\begin{enumerate}
\item  Uniform water level (keyword: \textit{tideloc = 0})

\item  One time-varying water level signal (keyword: \textit{tideloc = 1})

\item  Two time-varying water level signals, which requires point of application indication. (keyword: \textit{tideloc = 2})

\item  Four time-varying water level signals (keyword: \textit{tideloc = 4})
\end{enumerate}

\noindent 

\noindent For the option with a uniform water level the value specified in the params.txt is applied in the complete model domain (keyword: \textit{zs0}). For the option with one time-varying water level signal the specified water level is applied (keyword: \textit{zs0file = name\_of\_your\_time\_serie}) to the offshore boundary and a fixed value is applied at the backshore boundary (keyword: \textit{zs0=value}). For the option with two time-varying water level signals two water level signals are read from the zs0file. Note: one tidal record is applied to both sea corners and one tidal record to both land corners. This means there is no alongshore variation. An alongshore variation can be applied when applying four time-varying water level signals. 


\subsubsection{ River and point discharge}

\noindent The effect of a river outflow or other discharges can be simulated with XBeach. Multiple discharge locations can be designated. At a discharge location the discharge orifice is defined as well as the discharge time series in m${}^{3}$/s. The discharge orifice always constitutes an uninterrupted series of full grid abreast cell borders. It is not possible to define a discharge over half a grid cell nor is it possible to define a single discharge through grid cell borders that are either not adjacent or are not abreast.

\noindent At each time step the model sets the discharge and velocities at the grid cell borders that constitute the discharge orifice, which can be computed given the size of the discharge orifice and discharge time series. The discharge is positive in positive x or y direction. An exception is made when discharges are defined at the domain border. In that case the discharge is positive towards the domain (influx).

\noindent When a discharge is defined with a zero size orifice the discharge is assumed to be in vertical direction where a positive discharge is into the domain (influx). In these cases the discharge is linked to the closest grid cell center and at each time step mass according to the discharge time series is added. No momentum is added in case of a vertical discharge.


\subsection{ Sediment transport}

\noindent The boundary conditions for sediment transport are Neumann boundaries everywhere, implying that the cross-boundary gradients in the advection-diffusion equation are set to zero, as well as the gradients of the bed load transports in that direction. Cross-shore profile changes due to cross-shore transport gradients are possible, allowing the boundary to smoothly follow the rest of the model. Still, it is good modeling practice to have the boundaries away from the area of interest. \eject 


\section{ Input description}


\subsection{ General}

\noindent Upon running the XBeach executable \textit{xbeach.exe}, the file \textit{params.txt} in the current working directory will be read. The \textit{params.txt} file contains grid and bathymetry info, wave input, flow input, morphological input, etc. in the form of keyword/value pairs. Each keyword/value pair may contain an actual model parameter or refer to another file with additional information on the model setup. If a \textit{params.txt} file cannot be found then XBeach will not run.

\noindent In the \textit{params.txt} file there can be a single keyword/value pair per line. The keywords can be specified in any order. A keyword/value pair is separated by an equal sign (=). Each line containing an equal sign is interpreted as a keyword/value pair. Reversely, any lines without an equal sign are ignored and may be used for comments. Only a few keywords are required for the model to run, others have default values that are used in case the keyword is not mentioned in the \textit{params.txt} file. The essential parameters for a simulation with a JONSWAP spectrum are listed below:

\begin{enumerate}
\item  A \underbar{grid with values for x and y}. This can both in XBeach format (separate x and y files; keyword: \textit{xfile} and \textit{yfile}) or Delft3D (one single xy file; keyword: \textit{xyfile}). On top of that the user needs to specify the width of each domain (keyword: \textit{nx} and \textit{ny})

\item  A \underbar{bathymetry} file (keyword: \textit{depfile}) that matches with the grid you specified at 1)

\item  A \underbar{simulation time} (keyword: \textit{tstop}) in seconds

\item  A \underbar{directional grid} for short waves and rollers. The grid is determined by a minimum and maximum angle and width per bin (keywords: \textit{thetamin}, \textit{thetamax} and \textit{dtheta}).

\item  A \underbar{wave boundary condition type} (keyword: \textit{wbctype = jons}). With a separate file containing the variables of the parametric (keyword: \textit{bcfile}).
\end{enumerate}

\noindent 

\noindent It is strongly recommended to specify as few parameters explicitly as possible and rely on the defaults for the other parameters. When running XBeach, a file called xbeach.log is created, which lists all the parameters set through the params.txt but also all parameters not set, for which the defaults are used. When the user starts the model, it generates a file named XBlog.txt. In this file all the different keyword available are determined. When no keyword is defined the default value will be applied.  

\noindent 

\noindent This chapter describes the possibilities of the \textit{params.}txt file and any auxiliary information files that are called from the \textit{params.txt} file. The tables in this chapter contain a description of the keywords, the default values, its units and recommended value ranges, while the formats for additional input files are described in the relevant sections. Keyword marked with an astrix (*) are essential for XBeach to run. Keywords marked with a plus (+) are considered advanced expert options and should not be used for regular applications of XBeach. 

\noindent In this chapter, any references to keywords refer to keywords that can be used in the \textit{params.txt} file. Also any references to time indications are in seconds unless stated otherwise. A typical \textit{params.txt} file for a 1D XBeach model is:

\noindent \textbf{params.txt}

\noindent \%\%\%\%\%\%\%\%\%\%\%\%\%\%\%\%\%\%\%\%\%\%\%\%\%\%\%\%\%\%\%\%\%\%\%\%\%\%\%\%\%\%\%\%\%\%\%\%\%\%\%\%\%\%\%\%\%\%\%\%\%\%\%\%\%\%\%\%\%\%\%\%\%\%\%

\noindent \%\%\% XBeach parameter settings input file                                \%\%\%

\noindent \%\%\%                                                                     \%\%\%

\noindent \%\%\% date:     01-Jan-2015 12:00                                         \%\%\%

\noindent \%\%\% function: xb\_write\_params                                           \%\%\%

\noindent \%\%\%\%\%\%\%\%\%\%\%\%\%\%\%\%\%\%\%\%\%\%\%\%\%\%\%\%\%\%\%\%\%\%\%\%\%\%\%\%\%\%\%\%\%\%\%\%\%\%\%\%\%\%\%\%\%\%\%\%\%\%\%\%\%\%\%\%\%\%\%\%\%\%\%

\noindent 

\noindent \%\%\% Grid parameters \%\%\%\%\%\%\%\%\%\%\%\%\%\%\%\%\%\%\%\%\%\%\%\%\%\%\%\%\%\%\%\%\%\%\%\%\%\%\%\%\%\%\%\%\%\%\%\%\%\%\%\%\%\%\%

\noindent 

\noindent depfile      = bed.dep

\noindent posdwn       = 0

\noindent nx           = 265

\noindent ny           = 0

\noindent alfa         = 0

\noindent vardx        = 1

\noindent xfile        = x.grd

\noindent yfile        = y.grd

\noindent thetamin     = -90

\noindent thetamax     = 90

\noindent dtheta       = 15

\noindent thetanaut    = 0

\noindent 

\noindent \%\%\% Model time \%\%\%\%\%\%\%\%\%\%\%\%\%\%\%\%\%\%\%\%\%\%\%\%\%\%\%\%\%\%\%\%\%\%\%\%\%\%\%\%\%\%\%\%\%\%\%\%\%\%\%\%\%\%\%\%\%\%\%\%

\noindent 

\noindent tstop        = 3600

\noindent 

\noindent \%\%\% Physical constants \%\%\%\%\%\%\%\%\%\%\%\%\%\%\%\%\%\%\%\%\%\%\%\%\%\%\%\%\%\%\%\%\%\%\%\%\%\%\%\%\%\%\%\%\%\%\%\%\%\%\%\%

\noindent 

\noindent rho          = 1025

\noindent 

\noindent \%\%\% Tide boundary conditions \%\%\%\%\%\%\%\%\%\%\%\%\%\%\%\%\%\%\%\%\%\%\%\%\%\%\%\%\%\%\%\%\%\%\%\%\%\%\%\%\%\%\%\%\%\%

\noindent 

\noindent tideloc      = 2

\noindent zs0file      = tide.txt

\noindent 

\noindent \%\%\% Wave boundary condition parameters \%\%\%\%\%\%\%\%\%\%\%\%\%\%\%\%\%\%\%\%\%\%\%\%\%\%\%\%\%\%\%\%\%\%\%\%

\noindent 

\noindent wbctype      = jons

\noindent bcfile       = filelist.txt

\noindent 

\noindent \%\%\% Output variables \%\%\%\%\%\%\%\%\%\%\%\%\%\%\%\%\%\%\%\%\%\%\%\%\%\%\%\%\%\%\%\%\%\%\%\%\%\%\%\%\%\%\%\%\%\%\%\%\%\%\%\%\%\%

\noindent 

\noindent outputformat = netcdf

\noindent tint         = 3600

\noindent tstart       = 0

\noindent 

\noindent nglobalvar   = 3

\noindent zb

\noindent zs

\noindent H


\subsection{ Physical processes}

\noindent XBeach supports a variety of physical processes from generic, like waves and flow, to very specific, like ship motions and point discharge. Each process can be switched on or off. The commonly used processes are turned on by default. The table below lists the keywords used to switch on or off physical processes in XBeach.

\noindent \textit{Table 4.1 Input parameters for physical processes supported by XBeach}

\begin{tabular}{|p{0.8in}|p{1.1in}|p{0.6in}|p{0.7in}|p{0.4in}|p{0.6in}|} \hline 
keyword & description & default & range & units & remark \\ \hline 
avalanching & Turn on avalanching & 1 & 0 - 1 & - &  \\ \hline 
flow & Turn on flow calculation & 1 & 0 - 1 & - &  \\ \hline 
gwflow+ & Turn on groundwater flow & 0 & 0 - 1 & - &  \\ \hline 
lwave & Turn on short wave forcing on NLSW equations and boundary conditions & 1 & 0 - 1 & - &  \\ \hline 
morphology & Turn on morphology & 1 & 0 - 1 & - &  \\ \hline 
nonh+ & Turn on non-hydrostatic pressure: 0 = NSWE, 1 = NSW + non-hydrostatic pressure compensation Stelling \& Zijlema, 2003 & 0 & 0 - 1 & - &  \\ \hline 
sedtrans & Turn on sediment transport & 1 & 0 - 1 & - &  \\ \hline 
setbathy & Turn on time series of prescribed bathy input & 0 & 0 - 1 & - &  \\ \hline 
ships+ & Turn on ship waves & 0 & 0 - 1 & - &  \\ \hline 
single\_dir+ & Turn on stationary model for refraction, surfbeat based on mean direction & 0 & 0 - 1 & - &  \\ \hline 
snells+ & Turn on Snell's law for wave refraction & 0 & 0 - 1 & - &  \\ \hline 
swave & Turn on short waves & 1 & 0 - 1 & - &  \\ \hline 
swrunup+ & Turn on short wave runup & 0 & 0 - 1 & - &  \\ \hline 
vegetation+ & Turn on interaction of waves and flow with vegetation & 0 & 0 - 1 & - &  \\ \hline 
\end{tabular}


\subsection{ Grid and bathymetry}

\noindent XBeach' spatial grid size is defined by the keywords \textit{nx} and \textit{ny}. Here nx are the number of grid points in the cross-shore direction and ny the number in the alongshore direction. The size of the computational grid will be \textit{nx+1} by \textit{ny+1} cells large. The initial bathymetry is provided using a separate file that is referred to by the \textit{depfile} keyword, which has to have a size of [nx+1, ny+1]. This file contains an initial bed level for each grid cell where each line corresponds to a transect in x-direction (cross-shore). The values are positive down by default (so e.g. a value of `10' is 10 meters depth), but this can be changed using the \textit{posdwn} keyword.

\noindent Three main types of XBeach grids are supported: fast 1D, 1D and 2DH. Fast 1D grids have a single alongshore grid cell and thus a value \textit{ny=0 }and thus a single row (ny+1=1) in the \textit{depfile}. The 1D grids have 3 alongshore grid cells and thus a value \textit{ny=2 }and three rows in the \textit{depfile}. The 2DH grids have more than 3 alongshore grid cells, a value \textit{ny$>$2} and as many rows in the \textit{depfile}. In general, the bathymetry file has the following space-separated format:

\noindent \textbf{bed.dep}

\noindent $<$z 1,1$>$ $<$z 2,1$>$ $<$z 3,1$>$ ... $<$z nx,1$>$ $<$z nx+1,1$>$

\noindent $<$z 1,2$>$ $<$z 2,2$>$ $<$z 3,2$>$ ... $<$z nx,2$>$ $<$z nx+1,2$>$

\noindent $<$z 1,3$>$ $<$z 2,3$>$ $<$z 3,3$>$ ... $<$z nx,3$>$ $<$z nx+1,3$>$

\noindent ...

\noindent $<$z 1,ny$>$ $<$z 2,ny$>$ $<$z 3,ny$>$ ... $<$z nx,ny$>$ $<$z nx+1,ny$>$

\noindent $<$z 1,ny+1$>$ $<$z 2,ny+1$>$ $<$z 3,ny+1$>$ ... $<$z nx,ny+1$>$ $<$z nx+1,ny+1$>$

\noindent XBeach spatial grids can be equidistant or non-equidistant. In the former case the grid size is defined by the keywords \textit{dx} and \textit{dy}. In the latter case the keyword \textit{vardx} should be set to \textit{1} and x- and y-coordinates of the grid cells should be provided through the files referenced by the \textit{xfile }and \textit{yfile} keywords. These files take exactly the same format as the \textit{depfile} file where all coordinates along the x-direction are in one row and each row represents a cell in y-direction. XBeach grids are defined in a coordinate system of choice and can be either rectangular or curvilinear grids. This section can be found in 2.1.

\noindent Delft3D grids created with tools like RFGRID are also supported. To use Delft3D grids, choose \textit{gridform=delft3d} and provide a grid file via the keyword \textit{xyfile}. The format of Delft3D grids is not described here, but can be found in the Delft3D manual (Deltares, 2011). Also forced updating of bathymetries is supported as described in Section 6.2.11.

\noindent Apart for the spatial grid, XBeach also uses a directional grid for short waves and rollers. The grid is determined by a minimum and maximum angle and a directional bin size using the keywords \textit{thetamin}, \textit{thetamax} and \textit{dtheta} respectively. The \textit{thetamin} and \textit{thetamax} angles are either defined according to the Cartesian convention (angle w.r.t. the computational x-axis) or according to the nautical convention (angle w.r.t. deg. N, so from W is 270 deg. N). The convention is chosen using the keyword \textit{thetanaut} (thetanaut=0 for Cartesian and thetanaut=1 for Nautical)

\noindent Examples of typical input for a non-equidistant, fast 1D XBeach model, together with the \textit{params.txt} example at the start of this chapter, are:

\noindent \textbf{depfile = bed.dep}

\noindent -20.00 -20.00 -19.90 -19.80 -19.70 ... 14 14 15 15 15 

\noindent \textbf{xfile = x.grd}

\noindent 0.00 10.00 20.00 30.00 40.00 ... 1992.00 1994.00 1996.00 1998.00 2000.00

\noindent \textbf{yfile = y.grd}

\noindent 0.00 0.00 0.00 0.00 0.00 0.00 0.00 ... 0.00 0.00 0.00 0.00 0.00 0.00 0.00

\noindent All keywords related to grid and bathymetry input are listed in the following table:

\noindent \textit{Table 4.2 Input parameters for grid and bathymetry supported by XBeach}

\begin{tabular}{|p{0.6in}|p{1.0in}|p{0.5in}|p{1.0in}|p{0.5in}|p{0.5in}|} \hline 
keyword & description\textbf{} & default & range & units & remark \\ \hline 
alfa & Angle of computational x-axis relative to ``East'' & 0.0 & 0.0 - 360.0 & deg &  \\ \hline 
depfile & Name of the input bathymetry file &  &  & $<$file$>$ &  \\ \hline 
dtheta & Directional resolution & 10.0 & 0.1 - 20.0 & deg &  \\ \hline 
dtheta\_s & Directional resolution in case of stationary refraction  & 10.0 & 0.1 - 20.0 & deg &  \\ \hline 
dx & Regular grid spacing in x-direction & -1.0 & 0.0 - 1000000000.0 & m &  \\ \hline 
dy & Regular grid spacing in y-direction & -1.0 & 0.0 - 1000000000.0 & m &  \\ \hline 
gridform & Grid definition format & xbeach & xbeach, delft3d &  &  \\ \hline 
nx & Number of computational cell corners in x-direction & 50 & 2 - 10000 & - &  \\ \hline 
ny & Number of computational cell corners in y-direction & 2 & 0 - 10000 & - &  \\ \hline 
posdwn & Bathymetry is specified positive down \eqref{GrindEQ__1_} or positive up (-1) & 1.0 & -1.0 - 1.0 & - &  \\ \hline 
thetamax & Higher directional limit (angle w.r.t computational x-axis) & 90.0 & -180.0 - 180.0 & deg &  \\ \hline 
thetamin & Lower directional limit (angle w.r.t computational x-axis) & -90.0 & -180.0 - 180.0 & deg &  \\ \hline 
thetanaut & specify thetamin and thetamax in cartesian (0) or nautical \eqref{GrindEQ__1_} convention  & 0 & 0 - 1 & - &  \\ \hline 
vardx & Switch for variable grid spacing & 0 & 0 - 1 & - &  \\ \hline 
xfile* & Name of the file containing x-coordinates of the calculation grid &  &  & $<$file$>$ &  \\ \hline 
xori & X-coordinate of origin of axis & 0.0 & -100000000.0 - 1000000000.0 & m &  \\ \hline 
xyfile & Name of the file containing Delft3D xy-coordinates of the calculation grid &  &  & $<$file$>$ &  \\ \hline 
yfile* & Name of the file containing y-coordinates of the calculation grid &  &  & $<$file$>$ &  \\ \hline 
yori & Y-coordinate of origin of axis & 0.0 & -1000000000.0 - 1000000000.0 & m &  \\ \hline 
\end{tabular}


\subsection{ Waves input}

\noindent An XBeach model is generally forced by waves on its offshore boundary. These waves are described by the wave boundary conditions discussed in this section. The details of the wave motions within the model are described by the wave numerics in terms of the wave action balance (see 2.3.1), wave dissipation model (see 0) and wave roller model (see 2.3.6)

\noindent XBeach supports a variety of wave boundary condition types that are divided in two main groups: stationary and spectral boundary conditions. The \textit{wbctype} keyword can be used to select one particular type of wave boundary conditions. 3.1 gives an overview of all types of wave boundary conditions available for XBeach. Figure 4.1 can be used to help determine what type of wave boundary conditions is appropriate for your case. Each wave boundary condition type is explained in the following subsections. Note that most spectral wave boundary conditions can vary both in space and time using a \textit{FILELIST} and/or \textit{LOCLIST} construction as described in 4.4.4.

\noindent \textit{Table 4.3 Overview of wave boundary conditions supported by XBeach}

\begin{tabular}{|p{0.7in}|p{0.8in}|p{2.6in}|} \hline 
\textit{wbctype} & Type & description \\ \hline 
off & Special & no wave boundary condition \\ \hline 
stat & Non-spectral & stationary wave boundary condition (sea state) \\ \hline 
bichrom & Special & bichromatic (two wave component) waves \\ \hline 
ts\_1 & Non-spectral & first-order time series of waves (generated outside XBeach), this option specifies free surface elevation and short wave energy \\ \hline 
ts\_2 & Non-spectral & second-order time series of waves (generated outside XBeach), this option specifies free surface elevation and short wave energy \\ \hline 
jons & Spectral & wave groups generated using a parametric (JONSWAP) spectrum \\ \hline 
swan & Spectral & wave groups generated using a SWAN 2D output file \\ \hline 
vardens & Spectral & wave groups generated using a formatted file \\ \hline 
ts\_nonh & Non-spectral & boundary conditions initially meant for the non-hydrostatic wave type, this option specifies free surface elevation and velocity \\ \hline 
reuse & Special & reuse of wave conditions \\ \hline 
stat\_table & Non-spectral & a sequence of stationary conditions (sea states) \\ \hline 
jons\_table & Spectral & a sequence of time-varying wave groups \\ \hline 
\end{tabular}



\noindent \eject 

\noindent \includegraphics*[width=5.40in, height=8.92in, keepaspectratio=false]{image24}

\noindent \textit{Figure 4.1 Decision tree for selecting the appropriate type of wave boundary conditions}


\subsubsection{ Spectral wave boundary conditions}

\noindent Spectral wave boundary conditions are enabled using \textit{wbctype} values \textit{jons}, \textit{swan}, \textit{vardens} or \textit{jons\_table}. The conditions are defined in separate files referenced from the \textit{params.txt} file using the \textit{bcfile} keyword. A spectral wave boundary condition describes a spectrum \textit{shape} that XBeach uses to generate a (random) wave time series. The length and resolution of the generated time series is determined by the keywords \textit{rt} and \textit{dtbc} respectively. XBeach will reuse the generated time series until the simulation is completed. The resolution of the time series should be enough to accurately represent the bound long wave, but need not be as small as the time step used in XBeach.

\noindent An overview of all keywords relevant for spectral wave boundary conditions is given in the table below. The necessary file formats for each type of spectral wave boundary condition is explained in the following subsections.

\noindent \textit{Table 4.4 Input parameters for spectral wave boundary conditions supported by XBeach}

\begin{tabular}{|p{0.9in}|p{1.0in}|p{0.5in}|p{0.7in}|p{0.5in}|p{0.5in}|} \hline 
keyword & description & default & range & units & remark \\ \hline 
Tm01switch+ & Switch to enable Tm01 rather than Tm-10 & 0 & 0 - 1 & - &  \\ \hline 
bcfile & Name of spectrum file &  &  & $<$file$>$ &  \\ \hline 
correctHm0+ & Switch to enable Hm0 correction & 1 & 0 - 1 & - &  \\ \hline 
dtbc+ & Time step used to describe time series of wave energy and long wave flux at offshore boundary (not affected by morfac) & 1.0 & 0.1 - 2.0 & s &  \\ \hline 
dthetaS\_XB+ & The (counter-clockwise) angle in the degrees needed to rotate from the x-axis in SWAN to the x-axis pointing East & 0.0 & -360.0 - 360.0 & deg &  \\ \hline 
fcutoff+ & Low-freq cutoff frequency for wbctype = jons, swan or vardens boundary conditions & 0.0 & 0.0 - 40.0 & Hz &  \\ \hline 
wbctype & Wave boundary condition type & bichrom & stat, bichrom, ts\_1, ts\_2, jons, swan, vardens, reuse, ts\_nonh, off, stat\_table, jons\_table &  &  \\ \hline 
nonhspectrum+ & Spectrum format for wave action balance of non-hydrostatic waves & 0 & 0 - 1 & - &  \\ \hline 
nspectrumloc+ & Number of input spectrum locations & 1 & 1 -- ny+1 & - &  \\ \hline 
nspr+ & Switch to enable long wave direction forced into centers of short wave bins & 0 & 0 - 1 & - &  \\ \hline 
oldnyq+ & Switch to enable old nyquist switch & 0 & 0 - 1 & - &  \\ \hline 
random+ & Switch to enable random seed for wbctype = jons, swan or vardens boundary conditions & 1 & 0 - 1 & - &  \\ \hline 
rt & Duration of wave spectrum at offshore boundary, in morphological time & 3600 & 1200-7200 & s &  \\ \hline 
sprdthr+ & Threshold ratio to maximum value of S above which spectrum densities are read in & 0.08 & 0.0 - 1.0 & - &  \\ \hline 
trepfac+ & Compute mean wave period over energy band: for wbctype jons, swan or vardens; converges to Tm01 for trepfac = 0.0 and & 0.01 & 0.0 - 1.0 & - &  \\ \hline 
wbcversion+ & Version of wave boundary conditions & 3 & 1 - 3 & - &  \\ \hline 
\end{tabular}


\paragraph{ JONSWAP wave spectra}

\noindent JONSWAP spectrum input is enabled using \textit{wbctype} \textit{= jons}. A JONSWAP wave spectrum is parametrically defined in a file that is referenced using the \textit{bcfile} keyword. This file contains a single parameter per line in arbitrary order. The parameters that can be defined are listed in Table 4.5. All variables are optional. If no value is given, the default value as specified in the table is used. It is advised not to specify the keyword \textit{dfj} and allow XBeach to calculate the default value.

\noindent A typical JONSWAP definition file looks as follows:

\noindent \textbf{jonswap.txt}

\noindent Hm0  = 0.8

\noindent Tp  = 8

\noindent mainang = 285.

\noindent gammajsp = 3.3

\noindent s  = 10.

\noindent fnyq  = 0.3

\noindent For the definitions see the table below.

\noindent It is possible to use an alternative file format for time-varying JONSWAP spectra. To enable this option use the \textit{wbctype} value \textit{jons\_table}. In this case, each line in the spectrum definition file contains a parametric definition of a spectrum, like in a regular JONSWAP definition file, plus the duration for which that spectrum is used during the simulation. XBeach does not reuse time-varying spectrum files. Therefore the total duration of all spectra should at least match the duration of the simulation. The name of the file can be chosen freely, but the file format is fixed as follows and all parameters should be present in all lines:

\noindent \textbf{jonswap.txt}

\noindent $<$Hm0$>$ $<$Tp$>$ $<$mainang$>$ $<$gammajsp$>$ $<$s$>$ $<$duration$>$ $<$dtbc$>$

\noindent Note that we refer to the keywords used in a regular JONSWAP definition file in this example, with three differences: 1) the peak period rather than the peak frequency is defined 2) the duration is added (similar to \textit{rt }in \textit{params.txt}) 3) the time resolution is added (similar to \textit{dtbc} in \textit{params.txt}). The duration and boundary condition time step in this file overrules \textit{rt} and \textit{dtbf} in \textit{params.txt}. This format is also used for time-varying stationary wave boundary conditions as described in 

\noindent . As an example, the JONSWAP spectrum definition file presented above would look as follows if the significant wave height should be increased with 0.2~m every hour:

\noindent \textbf{jonswap.txt}

\noindent 0.8 8. 285. 3.3 10. 0.3 3600. 1

\noindent 1.0 8. 285. 3.3 10. 0.3 3600. 1

\noindent 1.2 8. 285. 3.3 10. 0.3 3600. 1

\noindent A more generic way of providing time-varying spectral wave boundary conditions is using a FILELIST construction as described in 4.4.4 Temporally and/or spatially varying wave boundary conditions. This approach is compatible with all spectral wave boundary condition types as well as spatially varying boundary conditions as described in the same section.

\noindent The parameter s in the JONSWAP spectrum definition is related to the directional spreading (in deg.) through the following relation $\sigma =\sqrt{\frac{2}{s+1} } \, ,\, \, \, \, \, s=\frac{2}{\sigma ^{2} } -1$. Here\textit{ $\sigma $} is the directional spreading in radians and s the JONSWAP spreading parameter.

\noindent \textbf{\includegraphics*[width=3.42in, height=2.66in, keepaspectratio=false, trim=0.22in 0.15in 0.29in 0.13in]{image25}}

\noindent \textit{Figure 4.2 Effect a variation in s for the direction spreading of wave energy}

\noindent 

\noindent \textit{Table 4.5 Overview of available keywords in JONSWAP definition file}

\begin{tabular}{|p{0.6in}|p{1.7in}|p{0.6in}|p{0.6in}|p{0.6in}|} \hline 
keyword & description & default & minimum & maximum \\ \hline 
Hm0 & Hm0 of the wave spectrum, significant wave height [m] & 0.0 & 0.0 & 5.0 \\ \hline 
fp & Peak frequency of the wave spectrum [s-1] & 0.08 & 0.0625 & 0.4 \\ \hline 
gammajsp & Peak enhancement factor in the JONSWAP expression [-] & 3.3 & 1.0 & 5.0 \\ \hline 
s & Directional spreading coefficient, cos${}^{2s}$ law [-] & 10. & 1.0 & 1000. \\ \hline 
mainang & Main wave angle (nautical convention) [${}^\circ$] & 270. & 180. & 360. \\ \hline 
fnyq & Highest frequency used to create JONSWAP spectrum [s-1] & 0.3 & 0.2 & 1.0 \\ \hline 
dfj & Step size frequency used to create JONSWAP spectrum [s-1] & fnyq/200 & fnyq/1000 & fnyq/20 \\ \hline 
\end{tabular}


\paragraph{ SWAN wave spectra}

\noindent XBeach can read standard SWAN 2D variance density or energy density output files (+.sp2 files) as specified in the SWAN v40.51 manual. This option is enabled using \textit{wbctype} \textit{= swan} in \textit{params.txt }and a reference to the spectrum file via the keyword \textit{bcfile}. XBeach assumes the directional information in the SWAN file is according to the nautical convention. If the file uses the Cartesian convention for directions, the user must specify the angle in degrees to rotate the x-axis in SWAN to the x-axis in XBeach (by the Cartesian convention). This value is specified in \textit{params.txt} using the keyword \textit{dthetaS\_XB}.

\noindent Note that time-varying and spatially varying SWAN spectra can be provided using the FILELIST and LOCLIST constructions as described in 4.4.4.

\noindent An example of a 2D SWAN spectrum is given below:

\noindent \textbf{swan.txt}

\noindent SWAN   1                                Swan standard spectral file

\noindent \$ Data produced by SWAN version 40.51

\noindent \$ Project:'projname'     ;   run number:'runnum'

\noindent LOCATIONS                               locations in x-y-space

\noindent 1                                       number of locations

\noindent 22222.22        0.00

\noindent RFREQ                                   relative frequencies in Hz

\noindent 23                                      number of frequencies
\[0.0545\] 
\[0.0622\] 
\[0.0710\] 
\[0.0810\] 
\[0.0924\] 
\[0.1055\] 
\[0.1204\] 
\[0.1375\] 
\[0.1569\] 
\[0.1791\] 
\[0.2045\] 
\[0.2334\] 
\[0.2664\] 
\[0.3040\] 
\[0.3470\] 
\[0.3961\] 
\[0.4522\] 
\[0.5161\] 
\[0.5891\] 
\[0.6724\] 
\[0.7675\] 
\[0.8761\] 
\[1.0000\] 
CDIR                                    spectral Cartesian directions in degr

\noindent   12                                    number of directions

\noindent  30.0000

\noindent  60.0000

\noindent  90.0000
\[120.0000\] 
\[150.0000\] 
\[180.0000\] 
\[210.0000\] 
\[240.0000\] 
\[270.0000\] 
\[300.0000\] 
\[330.0000\] 
\[360.0000\] 
QUANT

\noindent 1                                       number of quantities in table

\noindent VaDens                                  variance densities in m2/Hz/degr

\noindent m2/Hz/degr                              unit

\noindent -0.9900E+02                             exception value

\noindent FACTOR

\noindent 0.675611E-06

\noindent   51   242   574   956   1288  1482  1481   1286   957    579    244    51

\noindent  129  610   1443  2402  3238  3725  3724   3234   2406   1454   613    128

\noindent  273  1287  3054  5084  6846  7872  7869   6837   5091   3076   1295   271

\noindent  665  3152  7463  12402 16712 19229 19221  16690  12419  7518   3172   662

\noindent 1302 6159  14608 24275 32688 37618 37603  32644  24309  14716  6198   1296

\noindent 2328 10989 26020 43341 58358 67109 67080  58281  43401  26213  11058  2317

\noindent 3365 15922 37712 62733 84492 97150 97110  84380  62820  37991  16021  3349

\noindent 3426 16230 38440 63939 86109 99010 98969  85995  64027  38724  16331  3410

\noindent 2027 9612  22730 37790 50909 58529 58505  50841  37843  22898  9672   2018

\noindent  672  3178  7538  12535 16892 19440 19432  16870  12552  7594   3198   669

\noindent  101  479   1135  1890  2542  2924  2923   2539   1892   1144   482    101

\noindent    2    11    26    43    57    66    66     57     43     26     11     2

\noindent    0    0     0     0     0     0     0      0      0      0      0      0

\noindent    0    0     0     0     0     0     0      0      0      0      0      0

\noindent    0    0     0     0     0     0     0      0      0      0      0      0

\noindent    0    0     0     0     0     0     0      0      0      0      0      0

\noindent    0    0     0     0     0     0     0      0      0      0      0      0

\noindent    0    0     0     0     0     0     0      0      0      0      0      0

\noindent    0    0     0     0     0     0     0      0      0      0      0      0

\noindent    0    0     0     0     0     0     0      0      0      0      0      0

\noindent    0    0     0     0     0     0     0      0      0      0      0      0

\noindent    0    0     0     0     0     0     0      0      0      0      0      0

\noindent    0    0     0     0     0     0     0      0      0      0      0      0

\noindent    0    0     0     0     0     0     0      0      0      0      0      0


\paragraph{ Variance density spectra}

\noindent 2D spectral information that is not in SWAN format can be provided using a formatted variance density spectrum file and \textit{wbctype = vardens}. The spectrum file itself is again referenced using the keyword \textit{bcfile}. The contents of the file must adhere to a specific format:

\noindent \textbf{vardens.txt}

\noindent $<$number of frequencies (n)$>$

\noindent $<$frequency 1$>$

\noindent $<$frequency 2$>$

\noindent $<$frequency 3$>$

\noindent ...

\noindent $<$frequency n-1$>$

\noindent $<$frequency n$>$

\noindent $<$number of directions (m)$>$

\noindent $<$directions 1$>$

\noindent $<$directions 2$>$

\noindent $<$directions 3$>$

\noindent ...

\noindent $<$directions m-1$>$

\noindent $<$directions m$>$

\noindent $<$variance density 1,1$>$ $<$variance density 2,1$>$ ... $<$variance density m,1$>$

\noindent $<$variance density 1,2$>$ $<$variance density 2,2$>$ ... $<$variance density m,2$>$

\noindent ...

\noindent $<$variance density 1,n$>$ $<$variance density 2,n$>$ ... $<$variance density m,n$>$

\noindent Note that the directions must be defined according to the Cartesian convention and in the coordinate system used by XBeach. In this coordinate system 0${}^\circ$ corresponds to waves travelling in the direction of the x-axis, while 90${}^\circ$ corresponds to the direction of the y-axis. Also, the directions must be defined in increasing order. Time-varying and spatially varying variance density spectra can be provided using the FILELIST and LOCLIST constructions as described in 4.4.4 Temporally and/or spatially varying wave boundary conditions.

\noindent An example of a formatted variance density file is given below:

\noindent \textbf{vardens.txt}

\noindent 15 
\[0.0418\] 
\[0.0477\] 
\[0.0545\] 
\[0.0622\] 
\[0.0710\] 
\[0.0810\] 
\[0.0924\] 
\[0.1055\] 
\[0.1204\] 
\[0.1375\] 
\[0.1569\] 
\[0.1791\] 
\[0.2045\] 
\[0.2334\] 
\[0.2664\] 
13 
\[-180.0000\] 
\[-150.0000\] 
\[-120.0000\] 
\[-90.0000\] 
\[-60.0000\] 
\[-30.0000\] 
\[0.0000\] 
\[30.0000\] 
\[60.0000\] 
\[90.0000\] 
\[120.0000\] 
\[150.0000\] 
\[180.0000\] 
   0     0     0     0     0     0     0     0     0     0     0     0

\noindent   51   242   574   956  1288  1482  1481  1286   957   579   244    51

\noindent  129   610  1443  2402  3238  3725  3724  3234  2406  1454   613   128

\noindent  273  1287  3054  5084  6846  7872  7869  6837  5091  3076  1295   271

\noindent  665  3152  7463 12402 16712 19229 19221 16690 12419  7518  3172   662

\noindent 1302  6159 14608 24275 32688 37618 37603 32644 24309 14716  6198  1296

\noindent 2328 10989 26020 43341 58358 67109 67080 58281 43401 26213 11058  2317

\noindent 3365 15922 37712 62733 84492 97150 97110 84380 62820 37991 16021  3349

\noindent 3426 16230 38440 63939 86109 99010 98969 85995 64027 38724 16331  3410

\noindent 2027  9612 22730 37790 50909 58529 58505 50841 37843 22898  9672  2018

\noindent  672  3178  7538 12535 16892 19440 19432 16870 12552  7594  3198   669

\noindent  101   479  1135  1890  2542  2924  2923  2539  1892  1144   482   101

\noindent    2    11    26    43    57    66    66    57    43    26    11     2

\noindent    0     0     0     1     1     1     1     1     1     0     0     0

\noindent    0     0     0     0     0     0     0     0     0     0     0     0

\noindent    0     0     0     0     0     0     0     0     0     0     0     0


\subsubsection{ Non-spectral wave boundary conditions}

\noindent Stationary wave boundary conditions are enabled using \textit{wbctype} values \textit{stat}, \textit{ts\_1}, \textit{ts\_2, ts\_nonh }or \textit{stat\_table}. The conditions are\textit{ }generally defined within the \textit{params.txt} file directly using the keywords described in the table below. In addition, in case of \textit{wbctype} values \textit{ts\_1} or \textit{ts\_2} the file \textit{bc/gen.ezs} should be present that describes the infragravity wave forcing generated outside of XBeach. For the \textit{wbctype} \textit{ts\_nonh }also a separate file is needed. More information about these files in 4.4.2.2. 


\paragraph{ Stationary wave boundary conditions}

\noindent Only in case of \textit{wbctype = stat\_table} the time-varying stationary wave boundary conditions are fully described in an external file referenced by the \textit{bcfile} keyword. The \textit{bcfile }keyword is part of the spectral wave boundary condition input and also the referenced file is designed for time-varying spectral input in the form of JONSWAP spectra (see 4.4.1.1). In stationary mode only the relevant data from this file is used and irrelevant data like \textit{gamma} and \textit{dfj} are discarded. 

\noindent \textit{Table 4.6 Overview of available keywords for stationary boundary conditions}

\begin{tabular}{|p{0.8in}|p{1.1in}|p{0.6in}|p{0.8in}|p{0.4in}|p{0.6in}|} \hline 
keyword & description & default & range & units & remark \\ \hline 
Hrms & Hrms wave height for wbctype = stat, bichrom, ts\_1 or ts\_2 & 1.0 & 0.0 - 10.0 & m &  \\ \hline 
Tlong & Wave group period for case wbctype = bichrom & 80.0 & 20.0 - 300.0 & s &  \\ \hline 
Trep & Representative wave period for wbctype = stat, bichrom, ts\_1 or ts\_2 & 10.0 & 1.0 - 20.0 & s &  \\ \hline 
dir0 & Mean wave direction for wbctype = stat, bichrom, ts\_1 or ts\_2 (nautical convention) & 270.0 & 180.0 - 360.0 & deg &  \\ \hline 
wbctype & Wave boundary condition type & bichrom & stat, bichrom, ts\_1, ts\_2, jons, swan, vardens, reuse, ts\_nonh, off, stat\_table, jons\_table &  &  \\ \hline 
lateralwave & Switch for lateral boundary at left & neumann & neumann, wavecrest, cyclic &  &  \\ \hline 
m & Power in cos\^{}m directional distribution for wbctype = stat, bichrom, ts\_1 or ts\_2 & 10 & 2 - 128 & - &  \\ \hline 
nmax+ & Maximum ratio of cg/c for computing long wave boundary conditions & 0.8 & 0.5 - 1.0 & - &  \\ \hline 
taper & Spin-up time of wave boundary conditions, in morphological time & 100.0 & 0.0 - 1000.0 & s &  \\ \hline 
\end{tabular}


\paragraph{ Time series}

\noindent The wave boundary condition types of ts\_1 and ts\_2 need a separate file containing short wave energy and free surface elevation (including long wave motions). The format of this file is as follows:

\noindent 

\noindent \textbf{bc/gen.ezs}

\noindent $<$time 1$>$ $<$zs 1$>$ $<$E 1$>$

\noindent $<$time 1$>$ $<$zs 2$>$ $<$E 2$>$

\noindent $<$time 2$>$ $<$zs 3$>$ $<$E 3$>$

\noindent ...

\noindent The wave boundary condition type of ts\_\textit{nonh} also needs a separate file in order to run the simulation. This file, however, needs to contain free surface elevations and velocities (both in u and v).

\noindent \textbf{Boun\_u.bcf}

\begin{tabular}{|p{4.0in}|} \hline 
$<$ scalar/vector       $>$\newline $<$ number of variables $>$\newline $<$ variables: t,U,Zs,W $>$\newline $\left. \begin{array}{cccccccccc} {t^{0} } & {U_{1}^{0} } & {\cdots } & {U_{J}^{0} } & {\eta _{1}^{0} } & {\cdots } & {\eta _{J}^{0} } & {W_{1}^{0} } & {\cdots } & {W_{J}^{0} } \\ {\vdots } & {\vdots } & {} & {\vdots } & {\vdots } & {} & {\vdots } & {\vdots } & {} & {\vdots } \\ {t^{n} } & {U_{1}^{N} } & {\cdots } & {U_{J}^{N} } & {\eta _{1}^{N} } & {\cdots } & {\eta _{J}^{N} } & {W_{1}^{N} } & {\cdots } & {W_{J}^{N} } \end{array}\right\}{\rm [Data]}$ \\ \hline 
\end{tabular}




\subsubsection{ Special types of wave boundary conditions}

\noindent Two special types of wave boundary conditions are available that makes XBeach skip the generation of new wave time series. The first is \textit{wbctype} \textit{= off} which simply does not provide any wave forcing on the model and hence no wave action in the model.

\noindent The second is \textit{wbctype} \textit{= reuse} which makes XBeach reuse wave time series that were generated during a previous simulation. This can be a simulation using the same or a different model as long as the computational grids are identical. In order to reuse boundary conditions, all relevant files should be copied to the current working directory of the model (where the \textit{params.txt }file is located). Relevant files are the \textit{ebcflist.bcf} and \textit{qbcflist.bcf} files and all files referenced therein. Generally, the referenced files have \textit{E\_} and \textit{q\_} prefixes. No further wave boundary condition data need be given in \textit{params.txt}.

\noindent On top of that bichromatic waves are also supported. Currently the same input parameters as non-spectral waves (see 4.4.2) are required; however there are planes to elaborate the input parameters of the bichromatic wave in order to specify individual frequency.


\subsubsection{ Temporally and/or spatially varying wave boundary conditions}

\noindent Time-varying spectral wave boundary conditions can be defined by feeding in multiple spectrum definition files rather than a single definition file. In addition, the duration for which these spectra should occur needs to be defined.

\noindent To make use of this option, the user must specify a regular \textit{wbctype} value for spectral wave boundary conditions (\textit{jons}, \textit{swan} or \textit{vardens}), but instead of referencing a single spectrum definition file using the \textit{bcfile} keyword, an extra file listing all spectrum definition files is now referenced.

\noindent The first word in this extra file must be the keyword \textit{FILELIST}. In the following lines, each line contains the duration of this wave spectrum condition in seconds (similar to \textit{rt} in \textit{params.txt}), the required time step in this boundary condition file in seconds (similar to \textit{dtbf} in \textit{params.txt}) and the name of the spectral definition file used to generate these boundary conditions. The duration and boundary condition time step in this file overrules \textit{rt} and \textit{dtbf} in \textit{params.txt}. XBeach does not reuse time-varying spectrum files. Therefore the total duration of all spectra should at least match the duration of the simulation.

\noindent A typical input file contains the following:

\noindent \textbf{filelist.txt}

\noindent FILELIST

\noindent 1800 0.2 jonswap1.inp

\noindent 1800 0.2 jonswap1.inp

\noindent 1350 0.2 jonswap2.inp

\noindent 1500 0.2 jonswap3.inp

\noindent 1200 0.2 jonswap2.inp

\noindent 3600 0.2 jonswap4.inp

\noindent Similar to time-varying spectral wave boundary conditions, also spatially varying wave boundary conditions can be defined using a similar construction. In order to apply spatially varying spectra on the offshore boundary, the user must specify set the keywords \textit{wbcversion =3} and \textit{nspectrumloc=ns} in \textit{params.txt} where \textit{ns} is the number of locations in which a spectrum is defined. By default the number of defined spectra is one.

\noindent Similar to time-varying spectral wave boundary conditions, its spatially varying sibling uses an extra file listing all relevant spectrum definition files. The first word in this extra file must be the keyword \textit{LOCLIST}. This line should be followed by one line per spectrum definition location containing the world x-coordinate and world y-coordinate of the location that the input spectrum should apply, and the name of the file containing spectral wave information.

\noindent  A typical input file for a run with three JONSWAP spectra contains the following:

\noindent \textbf{loclist.txt}

\noindent LOCLIST

\noindent 0. 0.    jonswap1.inp

\noindent 0. 100.  jonswap2.inp

\noindent 0. 200.  jonswap3.inp

\noindent Note that it is not possible to use a mix of JONSWAP, SWAN and variance density files in either a \textit{FILELIST} or a \textit{LOCLIST} construction. It is also not possible to vary \textit{dthetaS\_XB} between files in case of non-nautical SWAN spectra. However, it is possible to combine \textit{FILELIST} and \textit{LOCLIST} files by referencing \textit{FILELIST} files from the \textit{LOCLIST} file. In this case all \textit{FILELIST} files should adhere to the same time discretization, so the duration and time step values should be constant over al \textit{FILELIST} files as well as the number of wave spectra definitions.

\noindent The user is reminded that along the offshore boundary of the model, the wave energy, rather than the wave height, is interpolated linearly between input spectra without consideration of the physical aspects of the intermediate bathymetry. In cases with large gradients in wave energy, direction or period, the user should specify sufficient wave spectra for the model to accurately represent changes in offshore wave conditions.


\subsection{ Flow, tide and surge input}

\noindent An XBeach model needs flow boundary conditions on all boundaries of the model domain. Moreover, on each boundary  tidal elevations and/ or surges may be imposed. The flow boundary conditions and time-varying tide or surge input are discussed in this section. The details on how the flow is computed within the model are described in the sections on bed friction and viscosity parameters (see 2.4).


\subsubsection{ Flow boundary conditions}

\noindent Flow boundary conditions need to be specified on all sides of the domain. We will differentiate between the offshore, lateral and landward boundaries that are set using the keywords \textit{front}, \textit{back} and \textit{left}/\textit{right,} respectively. Table 4.7 to Table 4.10 give an overview of the available flow boundary condition types for each of these boundaries.

\noindent The keyword \textit{freewave} can be used to switch from bound to free long waves, which can be useful when time series of the free long wave incident on the offshore boundary need to be specified. The file \textit{bc/gen.ezs} can be used to describe the free long waves at the offshore boundary as discussed in 3.1.

\noindent \textit{Table 4.7 Overview of available offshore flow boundary condition types}

\begin{tabular}{|p{0.9in}|p{3.2in}|} \hline 
\textit{front} & description \\ \hline 
abs1d & absorbing-generating (weakly-reflective) boundary in 1D \\ \hline 
abs2d & absorbing-generating (weakly-reflective) boundary in 2D \\ \hline 
wall & no flux wall \\ \hline 
wlevel & water level specification (from file) \\ \hline 
nonh\_1d & boundary condition for non-hydrostatic option \\ \hline 
waveflume & boundary condition for flume experiments based on a continuity relation  \\ \hline 
\end{tabular}

\textit{}

\noindent \textit{Table 4.8 Overview of available landward flow boundary condition types}

\begin{tabular}{|p{0.9in}|p{3.2in}|} \hline 
\textit{back} & description \\ \hline 
wall & no flux wall \\ \hline 
abs1d & absorbing-generating (weakly-reflective) boundary in 1D \\ \hline 
abs2d & absorbing-generating (weakly-reflective) boundary in 2D \\ \hline 
wlevel & water level specification (from file) \\ \hline 
\end{tabular}

\textit{}

\noindent \textit{Table 4.9 Overview of available lateral flow boundary condition types}

\begin{tabular}{|p{0.9in}|p{3.2in}|} \hline 
\textit{back} & description \\ \hline 
wall & no flux wall \\ \hline 
neumann & Neumann boundary condition (constant water level gradient) \\ \hline 
neumann\_v & velocity is determined by the adjacent cell \\ \hline 
no\_advec & Neumann boundary condition, but only the advective terms are taken into account. Intermediate form between wall and neumann \\ \hline 
\end{tabular}

\textit{}

\noindent \textit{Table 4.10 Preview of all keywords related to the flow boundary conditions}

\begin{tabular}{|p{0.7in}|p{1.1in}|p{0.6in}|p{0.8in}|p{0.4in}|p{0.6in}|} \hline 
keyword & description & default & range & units & remark \\ \hline 
ARC+ & Switch for active reflection compensation at seaward boundary & 1 & 0 - 1 & - &  \\ \hline 
back & Switch for boundary at bay side & abs\_2d & wall, abs\_1d, abs\_2d, wlevel &  &  \\ \hline 
epsi+ & Ratio of mean current to time varying current through offshore boundary & -1.0 & -1.0 - 0.2 & - &  \\ \hline 
freewave+ & Switch for free wave propagation 0 = use cg (default); 1 = use sqrt(gh) in wbctype = ts\_2 & 0 & 0 - 1 & - &  \\ \hline 
front & Switch for seaward flow boundary & abs\_2d & abs\_1d, abs\_2d, wall, wlevel, nonh\_1d, waveflume &  &  \\ \hline 
left & Switch for lateral boundary at ny+1 & neumann & neumann, wall, no\_advec, neumann\_v &  &  \\ \hline 
nc+ & Smoothing distance for estimating umean (defined as nr of cells) & ny+1 & 1 - ny+1 & - &  \\ \hline 
order+ & Switch for order of wave steering, 1 = first order wave steering (short wave energy only), 2 = second order wave steering (bound long wave corresponding to short wave forcing is added) & 2.0 & 1.0 - 2.0 & - &  \\ \hline 
right & Switch for lateral boundary at 0 & neumann & neumann, wall, no\_advec, neumann\_v &  &  \\ \hline 
tidetype+ & Switch for offshore boundary, velocity boundary or instant water level boundary & velocity & instant, velocity &  &  \\ \hline 
\end{tabular}


\subsubsection{ Time-varying tide/surge}

\noindent Time-varying tidal (or surge) signals can be applied all four boundaries in a number of ways. 

\noindent The number of tidal signals is determined by the keyword \textit{tideloc} that can take the values 0, 1, 2 or 4. Specifying three tidal signals is not an option. Setting \textit{tideloc=0} disables the time-varying tide/surge option. In this case a constant and uniform water level is used specified by the keyword \textit{zs0}. With \textit{tideloc =1} the specified tidal record is specified on all four corners of the domain and interpolated along the boundaries.

\noindent Using \textit{tideloc = 2}, two tidal signals are specified and there are two options available: 1) the first signal is imposed on the offshore boundary and the second on the landward boundary or 2) the first signal is imposed on the left lateral boundary and the second on the right lateral boundary. The choice between the two options is made using the keyword \textit{paulrevere} where a value \textit{0} indicates the first option and a value \textit{1} indicates the second option. Also in the case of two tidal signals the signals are spatially interpolated along the boundaries.

\noindent Using \textit{tideloc=4, }four tide/surge signals are to be specified\textit{ }on each corner of the model domain and spatially interpolated along the boundaries. The first signal is imposed to the left offshore boundary seen from sea (x=1,y=1) and the others according to a clockwise rotation. Therefore the columns in the \textit{zs0file }must follow the order of: (x=1,y=1), (x=1,y=N), (x=N,y=N), (x=N,y=1).

\noindent The length of the tidal signals is determined by the keyword \textit{tidelen}. This is the number of water levels specified in the file referenced with the \textit{zs0file} keyword. The tidal signal will be interpolated to the local time step of the XBeach simulation; therefore the resolution of the signals only needs to be enough to resolve the water level phenomenon of interest (i.e. tide variations, surge event). The tidal signals are not re-used, therefore the signal should be at least as long as the simulation time.

\noindent The \textit{zs0file} file must adhere to the following format where the last three columns are optional depending on the value of \textit{tideloc} and \textit{tlen} represents the value of \textit{tidelen}:

\noindent \textbf{tide.txt}

\noindent $<$time 1$>$ $<$zs 1,1$>$ [$<$zs 2,1$>$ [$<$zs 3,1$>$ $<$zs 4,1$>$]]

\noindent $<$time 2$>$ $<$zs 1,2$>$ [$<$zs 2,2$>$ [$<$zs 3,2$>$ $<$zs 4,2$>$]]

\noindent $<$time 3$>$ $<$zs 1,3$>$ [$<$zs 2,3$>$ [$<$zs 3,3$>$ $<$zs 4,3$>$]]

\noindent ...

\noindent $<$time tlen$>$ $<$zs 1,tlen$>$ [$<$zs 2,tlen$>$ [$<$zs 3,tlen$>$ $<$zs 4,tlen$>$]]

\noindent In case of a single tidal signal, the signal is imposed on both offshore corners of the domain, while a constant water level defined by the keyword \textit{zs0} is imposed on the landward corners. 

\noindent \textit{Table 4.11 Overview of all keywords related to the tide boundary conditions}

\begin{tabular}{|p{0.8in}|p{1.0in}|p{0.6in}|p{0.7in}|p{0.5in}|p{0.6in}|} \hline 
keyword & description & default & range & units & remark \\ \hline 
paulrevere & Specifies tide on sea and land or two sea points if tideloc = 2 & land & land, sea &  &  \\ \hline 
tideloc & Number of corner points on which a tide time series is specified & 0 & 0 - 4 & - &  \\ \hline 
zs0 & Initial water level & 0.0 & -5.0 - 5.0 & m &  \\ \hline 
zs0file & Name of tide boundary condition series &  &  & $<$file$>$ &  \\ \hline 
\end{tabular}


\subsection{ Water level (dam break)}

\noindent Water levels can be imposed on the model boundaries as explained in 4.5.2 Time-varying tide/surge after which the shallow water equations force the water body in the model domain. Specific applications may require the initialization of the entire water body in the model domain at the start of the simulation. For example, an initial significant gradient in the water level that ``collapses'' at the start of the simulation may simulate a dam break. The initialization of the water level in the model domain is governed by the keywords listed in the table below.

\noindent The keyword \textit{zsinitfile} references an external file describing the initial water levels in the entire model domain. The file should have the same format as the bathymetry input files described in 4.3 (Grid and bathymetry).

\noindent \eject \textit{}

\noindent \textit{Table 4.12 Overview of all keywords related to the water levels}

\begin{tabular}{|p{0.8in}|p{1.0in}|p{0.6in}|p{0.6in}|p{0.5in}|p{0.6in}|} \hline 
keyword & description & default & range & units & remark \\ \hline 
hotstartflow+ & Switch for hotstart flow conditions with pressure gradient balanced by wind and bed stress & 0 & 0 - 1 & - &  \\ \hline 
zs0 & Initial water level & 0.0 & -5.0 - 5.0 & m &  \\ \hline 
zsinitfile & Name of initial water level file &  &  & $<$file$>$ &  \\ \hline 
\end{tabular}


\subsection{ Wind input}

\noindent Spatially-uniform winds can parametrically defined using the keywords \textit{windv} and \textit{width} that represent the wind velocity and direction (nautical convention) respectively. Time-varying winds can be defined in an external file referenced by the \textit{windfile} keyword. The file should adhere to the format indicated below. The total length of the time series is automatically determined and should be at least as long as the simulation time.

\noindent \textbf{wind.txt}

\noindent $<$time 1$>$ $<$windv 1$>$ $<$windth 1$>$

\noindent $<$time 2$>$ $<$windv 2$>$ $<$windth 2$>$

\noindent $<$time 3$>$ $<$windv 3$>$ $<$windth 3$>$

\noindent ...

\noindent The table below gives an overview of all keywords related to the wind:

\noindent \textit{Table 4.13 Overview of all keywords related to the wind input}

\begin{tabular}{|p{0.7in}|p{1.0in}|p{0.6in}|p{0.7in}|p{0.5in}|p{0.6in}|} \hline 
keyword & description & default & range & units & remark \\ \hline 
Cd+ & Wind drag coefficient & 0.002 & 0.0001 - 0.01 & - &  \\ \hline 
rhoa+ & Air density & 1.25 & 1.0 - 2.0 & kgm\^{}-3 &  \\ \hline 
windfile & Name of file with non-stationary wind data &  &  & $<$file$>$ &  \\ \hline 
windth & Nautical wind direction, in case of stationary wind & 270.0 & -360.0 - 360.0 & deg &  \\ \hline 
windv & Wind velocity, in case of stationary wind & 0.0 & 0.0 - 200.0 & ms\^{}-1 &  \\ \hline 
\end{tabular}


\subsection{ Sediment input}

\noindent The sediment input determines the (initial) composition of the bed and the detail in which processes related to sediment sorting are resolved. This is different from how the sediment transport processes are handled in the model itself and that are described in 2.7 and 2.8

\noindent The simplest situation is an XBeach simulation with uniform sediment. In this case it is sufficient to specify the uniform grain size using the keyword \textit{D50} indicating the median grain size. The effects of a specific sediment distribution can be parametrically defined by additionally specifying values for \textit{D15 }and \textit{D90} and optionally the bed composition can be fine-tuned by specifying the porosity and sediment density using the keywords \textit{por} and \textit{rhos }respectively. In this case no sorting of sediment will be simulated.

\noindent If the effect of different sediment fractions, sorting and armoring are of importance, multiple sediment fractions can be defined. The number of sediment fraction is determined by the keyword \textit{ngd}. For each sediment fraction a value for \textit{D50,} and optionally \textit{D15} and \textit{D90}, should be defined separated by a space. Moreover, when using multiple sediment fractions, multiple bed layers are needed as well. The number of bed layers can be defined using the keyword \textit{nd}. 

\noindent Three types of bed layers are distinguished: 1) the top layer 2) the variable or ``breathing'' layer and 3) the bottom layers. At least one of each type of bed layer is needed, which makes that at least three bed layers are required (see 2.8.3). Each bed layer has a thickness. Choosing bed layer thicknesses that are in balance with the expected erosion and deposition during the simulation should keep the numerical mixing to a minimum. A bed layer thickness that is too large will result in relatively uniform behavior, while a bed layer thickness that is too small will result in a lot of shifting and thus numerical mixing. The bed layer thicknesses are determined by the three keywords \textit{dzg1}, \textit{dzg2} and \textit{dzg3} for the top, variable and bottom layers respectively.

\noindent Apart from the discretization of the grain size distribution and the vertical structure of the bed, the initial bed composition needs to be defined. The bed composition is defined using external files that are not explicitly referenced from \textit{params.txt}, but are assumed to be located in the working directory of the model (next to \textit{params.txt}). There is one file for each sediment fraction specified by \textit{ngd}. The file corresponding to the first sediment fraction is named \textit{gdist1.inp}, the second \textit{gdist2}.inp, et cetera.

\noindent The bed composition files hold information on how much sediment of a specific fraction is in each grid cell and bed layer at the start of the simulation. The values are a volumetric fraction that implies that they should add up to unity over all fractions. For example, if a specific grid cell is filled with the first sediment fraction only, the value corresponding to this grid cell will be one in the \textit{gdist1.inp} file and zero in all others. Alternatively, if we defined five sediment fractions and a specific grid cell is filled equally with all fractions, the value corresponding to this grid cell will be 1/5 = 0.2 in all files. The \textit{gidst$<$N$>$.inp} files are formatted comparable to the bathymetry files (see 4.3 Grid and bathymetry), but now holds values over the three dimensions x (nx+1), y (ny+1) and the bed layers (nd). The file format is as follows:

\noindent \textbf{gdist1.inp}

\noindent $<$p 1,1,1$>$ $<$p 1,2,1$>$ $<$p 1,3,1$>$ ... $<$p 1,nx,1$>$ $<$p 1,nx+1,1$>$

\noindent $<$p 1,1,2$>$ $<$p 1,2,2$>$ $<$p 1,3,2$>$ ... $<$p 1,nx,2$>$ $<$p 1,nx+1,2$>$

\noindent $<$p 1,1,3$>$ $<$p 1,2,3$>$ $<$p 1,3,3$>$ ... $<$p 1,nx,3$>$ $<$p 1,nx+1,3$>$

\noindent ...

\noindent $<$p 1,1,ny$>$ $<$p 1,2,ny$>$ $<$p 1,3,ny$>$ ... $<$p 1,nx,ny$>$ $<$p 1,nx+1,ny$>$

\noindent $<$p 1,1,ny+1$>$ $<$p 1,2,ny+1$>$ $<$p 1,3,ny+1$>$ ... $<$p 1,nx,ny+1$>$ $<$p 1,nx+1,ny+1$>$

\noindent 

\noindent ...

\noindent 

\noindent $<$p 2,1,1$>$ $<$p 2,2,1$>$ $<$p 2,3,1$>$ ... $<$p 2,nx,1$>$ $<$p 2,nx+1,1$>$

\noindent $<$p 2,1,2$>$ $<$p 2,2,2$>$ $<$p 2,3,2$>$ ... $<$p 2,nx,2$>$ $<$p 2,nx+1,2$>$

\noindent $<$p 2,1,3$>$ $<$p 2,2,3$>$ $<$p 2,3,3$>$ ... $<$p 2,nx,3$>$ $<$p 2,nx+1,3$>$

\noindent ...

\noindent $<$p 2,1,ny$>$ $<$p 2,2,ny$>$ $<$p 2,3,ny$>$ ... $<$p 2,nx,ny$>$ $<$p 2,nx+1,ny$>$

\noindent $<$p 2,1,ny+1$>$ $<$p 2,2,ny+1$>$ $<$p 2,3,ny+1$>$ ... $<$p 2,nx,ny+1$>$ $<$p 2,nx+1,ny+1$>$

\noindent 

\noindent ...

\noindent 

\noindent $<$p nd,1,1$>$ $<$p nd,2,1$>$ $<$p nd,3,1$>$ ... $<$p nd,nx,1$>$ $<$p nd,nx+1,1$>$

\noindent $<$p nd,1,2$>$ $<$p nd,2,2$>$ $<$p nd,3,2$>$ ... $<$p nd,nx,2$>$ $<$p nd,nx+1,2$>$

\noindent $<$p nd,1,3$>$ $<$p nd,2,3$>$ $<$p nd,3,3$>$ ... $<$p nd,nx,3$>$ $<$p nd,nx+1,3$>$

\noindent ...

\noindent $<$p nd,1,ny$>$ $<$p nd,2,ny$>$ $<$p nd,3,ny$>$ ... $<$p nd,nx,ny$>$ $<$p nd,nx+1,ny$>$

\noindent $<$p nd,1,ny+1$>$ $<$p nd,2,ny+1$>$ ... $<$p nd,nx,ny+1$>$ $<$p nd,nx+1,ny+1$>$

\noindent The table below gives an overview of all keywords related to working with multiple sediment fractions and bed layers:

\noindent \textit{Table 4.14 Overview of all keywords related to the sediment input}

\begin{tabular}{|p{0.6in}|p{1.0in}|p{0.6in}|p{1.0in}|p{0.4in}|p{0.6in}|} \hline 
keyword & description & default & range & units & remark \\ \hline 
D15 & D15 grain size per grain type & 0.00015 & 0.0001-0.0008 & m &  \\ \hline 
D50 & D50 grain size per grain type & 0.0002 & 0.0001-0.0008 & m &  \\ \hline 
D90 & D90 grain size per grain type & 0.0003 & 0.0001-0.0015 & m &  \\ \hline 
dzg+ & Thickness of top sediment class layers & 0.1 & 0.01 - 1.0 & m &  \\ \hline 
dzg2+ & Nominal thickness of variable sediment class layer & 0.1 & 0.01 - 1.0 & m &  \\ \hline 
dzg3+ & Thickness of bottom sediment class layers & 0.1 & 0.01 - 1.0 & m &  \\ \hline 
nd+ & Number of computational layers in the bed & 3 & 3 - 1000 & - &  \\ \hline 
ngd & Number of sediment classes & 1 & 1 - 20 & - &  \\ \hline 
por & Porosity & 0.4 & 0.3 - 0.5 & - &  \\ \hline 
rhos & Solid sediment density (no pores) & 2650.0 & 2400.0 - 2800.0 & kgm\^{}-3 &  \\ \hline 
sedcal+ & Sediment transport calibration coefficient per grain type & 1 & None & - &  \\ \hline 
ucrcal+ & Critical velocity calibration coefficient per grain type & 1 & None & - &  \\ \hline 
\end{tabular}


\subsection{ Vegetation input}

\noindent Short wave dissipation, long wave dissipation and flow interaction due to vegetation is supported. The user can define multiple vegetation species. The number of vegetation species is set by the keyword \textit{nveg}. Furthermore, two files should be created and specified in the params.txt-file: a vegetation characteristics file (keyword \textit{veggiefile}) and a vegetation location file (keyword \textit{veggiemapfile}). 

\noindent The veggiefile is a text file listing the names of the vegetation characteristics files that should be created for every individual vegetation species that should be accounted for. These property files contain the vegetation parameters \textit{nsec}, \textit{ah}, \textit{C${}_{d}$}, \textit{b${}_{v}$} and \textit{N} that represent the number of vertical sections, height of vegetation section relative to the bed , the drag coefficient, stem diameter and vegetation density per vegetation section, respectively. An example of a set of files describing two different vegetation species is given below.

\noindent \textbf{veggiefile.txt}

\noindent seagrass.txt

\noindent mangrove.txt

\noindent \textbf{seagrass.txt}

\noindent ah = 0.2

\noindent Cd = 1.0

\noindent bv = 0.02

\noindent N = 1200

\noindent \textbf{mangrove.txt}

\noindent nsec = 3

\noindent ah = 0.5 0.8 1.3

\noindent Cd = 2.0 1.0 2.0

\noindent bv = 0.05 0.15 0.1

\noindent N = 1000 50 500

\noindent The \textit{nsec} keyword in the species property file allows the user to define multiple height segments of the species with different properties. The height per vegetation section is defined relative to the bed level. For all properties, the values are given from bottom to top. A definition sketch is given in Figure XX.

\noindent \includegraphics*[width=4.01in, height=2.22in, keepaspectratio=false]{image26}

\noindent \textit{Figure 4.3 Definition sketch of vegetation specification in XBeach (example for mangrove type vegetation schematized in three vertical sections).}

\noindent 

\noindent Finally,  the \textit{veggiemapfile} indicates in what grid cell which vegetation species can be found. The format of this file is similar to the bathymetry files described in 4.3 Grid and bathymetry, but the values are integers referring to a species where 1 refers to the first listed species, 2 to the second, et cetera. A zero indicates no vegetation at that particular location.

\noindent In summary, the following files should be created when the effect of vegetation is modeled:

\begin{enumerate}
\item  1 x Veggiemapfile: file similar to bathymetry file containing 0, 1 etc. (up to \textit{nveg})

\item  1 x Veggiefile: list of file names of vegetation property files per species

\item  \textit{Nveg }x vegetation property file(s): describing vegetation properties per species
\end{enumerate}

\noindent 

\noindent Below the relevant keywords in the params.txt are given. In addition, the keyword \textit{vegetation} should be set to 1.

\noindent \textit{Table 4.15 Overview of all keywords related to the vegetation module}

\begin{tabular}{|p{0.8in}|p{1.0in}|p{0.6in}|p{0.6in}|p{0.5in}|p{0.6in}|} \hline 
keyword & description & default & range & units & remark \\ \hline 
nveg & Number of vegetation species & -123 &  & - &  \\ \hline 
veggiefile & Name of vegetation species list file &  &  & $<$file$>$ &  \\ \hline 
veggiemapfile & Name of vegetation species map file &  &  & $<$file$>$ &  \\ \hline 
\end{tabular}


\subsection{ Discharge input}

\noindent Discharge of water at the model boundaries or directly in the model domain is defined along specific grid sections. The keywords \textit{ndischarge} and \textit{ntdischarge} define the number of discharge sections and the length of the discharge time series respectively. The \textit{disch\_loc\_file }keyword references a file that defines the discharge sections. Each line in this file corresponds to a grid section and each line contains four numbers being the start and end coordinates of the section. The file is formatted as follows, where \textit{ndisch} refers to the keyword \textit{ndischarge}:

\noindent \textbf{disch\_loc.txt}

\noindent $<$x\_start 1$>$ $<$y\_start 1$>$ $<$x\_end 1$>$ $<$y\_end 1$>$

\noindent $<$x\_start 2$>$ $<$y\_start 2$>$ $<$x\_end 2$>$ $<$y\_end 2$>$

\noindent $<$x\_start 3$>$ $<$y\_start 3$>$ $<$x\_end 3$>$ $<$y\_end 3$>$

\noindent ...

\noindent $<$x\_start ndisch$>$ $<$y\_start ndisch$>$ $<$x\_end ndisch$>$ $<$y\_end ndisch$>$

\noindent The world coordinates specified in this file must be chosen such that they are close to the desired grid cell borders, since the grid cell borders are eventually used as discharge section. Discharge sections can be located along grid cell borders that are either oriented cross-shore or alongshore, but not a combination of the two. In a regular grid this implies that either the start or end x-coordinates are equal, or the start and end y-coordinates are equal. Alternatively, both are equal. In this case a vertical discharge from above is assumed, rather than a horizontal discharge. Vertical discharges only add mass and no momentum to the water body.

\noindent The keyword \textit{disch\_time series\_file} references a file defining the time series imposed on the discharge locations. The file lists the timings in the first column and a discharge value in m${}^{3}$/s for each discharge section as follows, where \textit{ntdisch} refers to the keyword \textit{ntdischarge}:

\noindent \textbf{disch\_time series.txt}

\noindent $<$t 1$>$ $<$Q 1,1$>$ $<$Q 2,1$>$ ... $<$Q ndisch,1$>$

\noindent $<$t 2$>$ $<$Q 1,2$>$ $<$Q 2,2$>$ ... $<$Q ndisch,2$>$

\noindent $<$t 3$>$ $<$Q 1,3$>$ $<$Q 2,3$>$ ... $<$Q ndisch,3$>$

\noindent ...

\noindent $<$t ntdisch$>$ $<$Q 1,ntdisch$>$ $<$Q 2,ntdisch$>$ ... $<$Q ndisch,ntdisch$>$

\noindent Discharges defined at the domain borders are positive in direction towards the domain (influx). Discharges defined in the domain itself are positive in direction of the positive x or y direction. Vertical discharges are positive into the domain (influx).

\noindent \includegraphics*[width=5.09in, height=3.04in, keepaspectratio=false]{image27}

\noindent \textit{Figure 4.4 Possible discharge orifices. A discharge orifice is defined as line in between two points (red). The resulting discharge orifice constitutes out of full abreast grid cells (green). The discharge direction is in positive grid direction (s or n), except at the domain border where the discharge is an influx in the domain (green arrows).}

\noindent The table below gives an overview of all keywords related to discharges:

\noindent \textit{Table 4.16 Overview of all keywords related to the discharge input}

\begin{tabular}{|p{0.9in}|p{1.0in}|p{0.6in}|p{0.6in}|p{0.5in}|p{0.6in}|} \hline 
keyword & description & default & range & units & remark \\ \hline 
disch\_loc\_file+ & Name of discharge locations file &  &  & $<$file$>$ &  \\ \hline 
disch\_time series\_file+ & Name of discharge time series file &  &  & $<$file$>$ &  \\ \hline 
ndischarge+ & Number of discharge locations &   & 0 - 100 & - &  \\ \hline 
ntdischarge+ & Length of discharge time series &   & 0 - 100 & - &  \\ \hline 
\end{tabular}


\subsection{ Drifters input}

\noindent Drifters can be deployed during the model simulation by specifying the number of drifters using the keyword \textit{ndrifter} and the location, start and end time of the drifter deployment in a separate file referenced by the \textit{drifterfile} keyword. The file format is as follows:

\noindent \textbf{drifter.txt}

\noindent $<$x 1$>$ $<$y 1$>$ $<$t\_start 1$>$ $<$t\_end 1$>$

\noindent $<$x 2$>$ $<$y 2$>$ $<$t\_start 2$>$ $<$t\_end 2$>$

\noindent $<$x 3$>$ $<$y 3$>$ $<$t\_start 3$>$ $<$t\_end 3$>$

\noindent ...

\noindent $<$x ndrifter$>$ $<$y ndrifter$>$ $<$t\_start ndrifter$>$ $<$t\_end ndrifter$>$

\noindent The table below gives an overview of all keywords related to drifters:

\noindent \textit{Table 4.17 Overview of all keywords related to drifters}

\begin{tabular}{|p{0.8in}|p{1.0in}|p{0.8in}|p{0.6in}|p{0.5in}|p{0.6in}|} \hline 
keyword & description\textbf{} & default & range & units & remark \\ \hline 
drifterfile & Name of drifter data file &  &  & $<$file$>$ &  \\ \hline 
ndrifter & Number of drifters & ndrifter & 0 - 50 & - &  \\ \hline 
\end{tabular}


\subsection{ Ship-induced wave motions}

\noindent Ship waves can be simulated by defining the ships' geometries and trajectories in a collection of files. The user can define multiple ships. The number of ships is set by the keyword \textit{nship}. In the file referenced by the keyword \textit{shipfile} each ship is given a name. The properties of each ship are summarized in another textfile with the name of the ship (\textit{shipname.txt}). This properties file defines the parameters for the discretization of the ships geometry. The ship grid is determined by the keywords \textit{dx, dy, nx}, and \textit{ny}, and the ship geometry itself is given in a separate file, referenced bythe keyword \textit{shipgeom}. This file contains the ship draft per ship grid point, and should have a size of \textit{nx}+1 by \textit{ny}+1. The center of gravity of the ship is also defined in the ship properties file using the keywords \textit{xCG, yCG} and \textit{zCG}. The ships trajectory is defined in a file referenced from the ship properties file by the keyword\textit{ shiptrack}. Each row in this file contains a time, x- and y-coordinate indicating the ships  position as function of time.  

\noindent To avoid numerical problems, the full ship track should be within the model domain (i.e. a vessel cannot sail through the model boundary). Furthermore, it is advised to start the ship track with a very low velocity and gradually increase the sailing speed. When not taking into account such spin up period, the impact of the ship on the water level can be too abrupt, resulting in unrealistic wave patterns. In addition, it is advised to maintain a relatively deep edge of a few grid cells at both model boundaries (i.e. front and back side).

\noindent Another way to avoid numerical issues at the initialization of a ship simulation is to use the 'flying' option, which can be specified in the ship file (\textit{flying }= 1). In case the option \textit{flying} is enabled, also a z-coordinate is defined in the shiptrack-file indicating the vertical position of the ship. In this way, the vessel can `land' on the water with its correct sailing speed, thereby avoiding unwanted disturbance to the water level. Also, the ship can `fly out' of the model domain before reaching the back boundary. By using this method the spin-up time can be reduced considerably and both inflow and outflow boundaries are unaffected by the ship

\noindent The two keywords\textit{ compute\_force} and\textit{ compute\_motion }enable the computation of forces on the ship and the ships motions due to wave forcing respectively; the latter has not been implemented yet. Forces on a ship in motion may be unreliable due to near-field effects; forces on a ship at rest are much more reliable. An example of ship definition files is:

\noindent \textbf{shipfile.txt}

\noindent containership.txt

\noindent oiltanker.txt

\noindent \textbf{containership.txt}

\noindent dx = 10

\noindent dy = 10

\noindent nx = 30

\noindent ny = 10

\noindent shipgeom = container\_geom.dep

\noindent xCG = 120

\noindent yCG = 50

\noindent zCG = 30

\noindent shiptrack = container\_track.txt

\noindent flying = 1

\noindent compute\_force = 1

\noindent compute\_motion = 1

\noindent \textbf{pannamax\_geom.txt}

\noindent $<$z 0,0$>$ $<$z 1,0$>$ $<$z 2,0$>$ $<$z 3,0$>$ ... $<$z nx,0$>$ $<$z nx+1,0$>$

\noindent $<$z 0,1$>$ $<$z 1,1$>$ $<$z 2,1$>$ $<$z 3,1$>$ ... $<$z nx,1$>$ $<$z nx+1,1$>$

\noindent ...

\noindent $<$z 0,ny$>$ $<$z 1,ny$>$ $<$z 2,ny$>$ $<$z 3,ny$>$ ... $<$z nx,ny$>$ $<$z nx+1,ny$>$

\noindent $<$z 0,ny+1$>$ $<$z 1,ny+1$>$ $<$z 2,ny+1$>$ $<$z 3,ny+1$>$ ... $<$z nx,ny+1$>$ $<$z nx+1,ny+1$>$

\noindent \textbf{pannamax\_track.txt}

\noindent $<$t 1$>$ $<$x 1$>$ $<$y 1$>$ $<$z 1$>$

\noindent $<$t 2$>$ $<$x 2$>$ $<$y 2$>$ $<$z 2$>$

\noindent $<$t 3$>$ $<$x 3$>$ $<$y 3$>$ $<$z 3$>$

\noindent ...

\noindent \textbf{oiltanker.txt}

\noindent dx = 2

\noindent dy = 2

\noindent nx = 20

\noindent ny = 4

\noindent shipgeom = tanker\_geom.dep

\noindent xCG = 20

\noindent yCG = 40

\noindent zCG = 1.5

\noindent shiptrack = tanker\_track.txt

\noindent flying = 0

\noindent \textbf{tanker\_track.txt}

\noindent $<$t 1$>$ $<$x 1$>$ $<$y 1$>$

\noindent $<$t 2$>$ $<$x 2$>$ $<$y 2$>$

\noindent $<$t 3$>$ $<$x 3$>$ $<$y 3$>$

\noindent ...

\noindent \textit{Table 4.18 Overview of all keywords related to the ship module}

\begin{tabular}{|p{0.7in}|p{1.1in}|p{0.6in}|p{0.7in}|p{0.5in}|p{0.6in}|} \hline 
Keyword & description & default & range & units & remark \\ \hline 
nship* & Number of ships & -123 &  & - &  \\ \hline 
shipfile & Name of ship data file &  &  & $<$file$>$ &  \\ \hline 
\end{tabular}


\subsection{ Output selection}

\noindent Output selection determines what data computed by XBeach is written to a file in terms of location and time and in what format. The output types, output times and output formats supported by XBeach are explained in more detail in the following subsections. The table below gives an overview of all keywords related to model output:

\noindent \textit{Table 4.19 Overview of all keywords related to the output definitions}

\begin{tabular}{|p{0.8in}|p{1.0in}|p{0.6in}|p{0.7in}|p{0.5in}|p{0.6in}|} \hline 
keyword & description\textbf{} & default & range & units & remark \\ \hline 
globalvars+ & Mnems of global output variables, not per se the same size as nglobalvar (invalid variables, defaults) & 'abc' &  & - &  \\ \hline 
meanvars+ & Mnems of mean output variables (by variables) & 'abc' &  & - &  \\ \hline 
ncfilename+ & Xbeach netcdf output file name &  &  & $<$file$>$ &  \\ \hline 
ncross+ & Number of output cross sections & 0 & 0 - 50 & - &  \\ \hline 
nglobalvar & Number of global output variables (as specified by user) & -1 & -1 - 20 & - &  \\ \hline 
nmeanvar & Number of mean, min, max, var output variables & 0 & 0 - 15 & - &  \\ \hline 
npoints & Number of output point locations & 0 & 0 - 50 & - &  \\ \hline 
npointvar & Number of point output variables & 0 & 0 - 50 & - &  \\ \hline 
nrugauge & Number of output runup gauge locations & 0 & 0 - 50 & - &  \\ \hline 
nrugdepth+ & Number of depths to compute runup in runup gauge & 1 & 1 - 10 & - &  \\ \hline 
outputformat+ & Output file format & fortran & fortran, netcdf, debug &  &  \\ \hline 
pointtypes+ & Point types (0 = point, 1 = rugauge) & $>$ NULL() &  & - &  \\ \hline 
pointvars+ & Mnems of point output variables (by variables) & 'abc' &  & - &  \\ \hline 
rugdepth+ & Minimum depth for determination of last wet point in runup gauge & 0 & 0.0.1 & m &  \\ \hline 
timings+ & Switch enable progress output to screen & 1 & 0 - 1 & - &  \\ \hline 
tintc+ & Interval time of cross section output & 1.0 & 0.01 - 100000.0 & s &  \\ \hline 
tintg & Interval time of global output & 1.0 & 0.01 - 100000.0 & s &  \\ \hline 
tintm & Interval time of mean, var, max, min output & tstop- tstart & 1 -- [tstop -- start] & s &  \\ \hline 
tintp & Interval time of point and runup gauge output & 1.0 & 0.01 - 100000.0 & s &  \\ \hline 
tscross+ & Name of file containing timings of cross section output & None & None - None & - &  \\ \hline 
tsglobal+ & Name of file containing timings of global output & None & None - None & - &  \\ \hline 
tsmean+ & Name of file containing timings of mean, max, min and var output & None & None - None & - &  \\ \hline 
tspoints+ & Name of file containing timings of point output & None & None - None & - &  \\ \hline 
tstart & Start time of output, in morphological time & 1.0 & 0.0 - 1000000.0 & s &  \\ \hline 
\end{tabular}


\subsubsection{ Output types}

\noindent XBeach supports four different types of output: 1) instantaneous spatial output 2) time-averaged spatial output 3) fixed point output or 4) run-up gauge output. In principle any variable in XBeach can be outputted as long as it is part of the \textit{spaceparams} structure defined in \textit{spaceparams.tmpl} in the XBeach source code. An overview of all currently supported parameters in this file is presented in Table 4.20.

\noindent The amount of output variables used for each type is determined by the keywords \textit{nglobalvar}, \textit{nmeanvar}, \textit{npoints} and \textit{nrugauge}. Each of these keywords takes a number indicating the number of parameters or locations that should be written to file. If any of the keywords is set to zero, the output type is effectively disabled. If \textit{nglovalvar} is set to \textit{-1} then a standard set of output variables is used, being \textit{H, zs, zs0, zb, hh, u, v, ue, ve, urms, Fc, Fy, ccg, ceqsg, ceqbg, Susg, Svsg, E, R, D} and \textit{DR}. If \textit{nglobalvar} is not set it defaults to \textit{-1}. The lines in the \textit{params.txt} file immediately following these keywords determine what parameters or locations are used, as will be explained in more detail in the following subsections.

\noindent \textit{Table 4.20 Possible output parameters}

\begin{tabular}{|p{0.7in}|p{2.5in}|p{0.9in}|} \hline 
Name & \textbf{Explanation} & \textbf{Unit} \\ \hline 
As & asymmetry of short waves             & [-] \\ \hline 
bi & incoming bound long wave             & [m] \\ \hline 
BR & maximum wave surface slope used in roller dissipation formulation        & [-] \\ \hline 
c & wave celerity               & [m/s] \\ \hline 
ccbg & depth-averaged bed concentration for each sediment fraction          & [m3/m3] \\ \hline 
ccg & depth-averaged suspended concentration for each sediment fraction          & [m3/m3] \\ \hline 
cctot & Sediment concentration integrated over bed load and suspended and for all sediment grains    & [m3/m3] \\ \hline 
ceqbg & depth-averaged bed equilibrium concentration for each sediment class         & [m3/m3] \\ \hline 
ceqsg & depth-averaged suspended equilibrium concentration for each sediment class         & [m3/m3] \\ \hline 
cf & Friction coefficient flow              & [-] \\ \hline 
cg & group velocity               & [m/s] \\ \hline 
cgx & group velocity, x-component              & [m/s] \\ \hline 
cgy & group velocity, y-component              & [m/s] \\ \hline 
costh & cos of wave angles relative to grid direction         & [-] \\ \hline 
ctheta & wave celerity theta-direction (refraction)             & [rad/s] \\ \hline 
cx & wave celerity, x-component              & [m/s] \\ \hline 
cy & wave celerity, y-component              & [m/s] \\ \hline 
D & dissipation                & [W/m2] \\ \hline 
D50 & D50 grain diameters for all sediment classes          & [m] \\ \hline 
D50top & Friction coefficient flow              & [-] \\ \hline 
D90 & D90 grain diameters for all sediment classes          & [m] \\ \hline 
D90top & Friction coefficient flow              & [-] \\ \hline 
Dc & diffusion coefficient               & [m2/s] \\ \hline 
dcbdx & bed concentration gradient x-dir.             & [kg/m3/m] \\ \hline 
dcbdy & bed concentration gradient y-dir.             & [kg/m3/m] \\ \hline 
dcsdx & suspended concentration gradient x-dir.             & [kg/m3/m] \\ \hline 
dcsdy & suspended concentration gradient y-dir.             & [kg/m3/m] \\ \hline 
depo\_ex & explicit bed deposition rate per fraction           & [m/s] \\ \hline 
depo\_im & implicit bed deposition rate per fraction           & [m/s] \\ \hline 
Df & dissipation rate due to bed friction           & [W/m\^{}2] \\ \hline 
dinfil & Infiltration layer depth used in quasi-vertical flow model for groundwater       & [m] \\ \hline 
dnc & grid distance in n-direction, centered around c-point          & [m] \\ \hline 
dnu & grid distance in n-direction, centered around u-point          & [m] \\ \hline 
dnv & grid distance in n-direction, centered around v-point          & [m] \\ \hline 
dnz & grid distance in n-direction, centered around z-point (=eta-point)         & [m] \\ \hline 
Dp & dissipation rate in the swash due to transformation of kinetic wave energy to potential wave energy & [W/m\^{}2] \\ \hline 
DR & roller energy dissipation              & [W/m2] \\ \hline 
dsc & grid distance in s-direction, centered around c-point          & [m] \\ \hline 
dsdnui & inverse of grid cell surface, centered around u-point         & [1/m2] \\ \hline 
dsdnvi & inverse of grid cell surface, centered around v-point         & [1/m2] \\ \hline 
dsdnzi & inverse of grid cell surface, centered around z-point         & [1/m2] \\ \hline 
dsu & grid distance in s-direction, centered around u-point          & [m] \\ \hline 
dsv & grid distance in s-direction, centered around v-point          & [m] \\ \hline 
dsz & grid distance in s-direction, centered around z-point (=eta-point)         & [m] \\ \hline 
dzav & total bed level change due to avalanching          & [m] \\ \hline 
dzbdt & rate of change bed level            & [m/s] \\ \hline 
dzbdx & bed level gradient in x-direction            & [-] \\ \hline 
dzbdy & bed level gradient in y-direction            & [-] \\ \hline 
dzbed & bed level gradient               & [-] \\ \hline 
dzsdt & rate of change water level            & [m/s] \\ \hline 
dzsdx & water surface gradient in x-direction            & [m/s] \\ \hline 
dzsdy & water surface gradient in y-direction            & [m/s] \\ \hline 
E & wave energy               & [Nm/m2] \\ \hline 
ee & directionally distributed wave energy             & [J/m2/rad] \\ \hline 
ero & bed erosion rate per fraction            & [m/s] \\ \hline 
Fx & wave force, x-component              & [N/m2] \\ \hline 
Fy & wave force, y-component              & [N/m2] \\ \hline 
gw0back & boundary condition back boundary for groundwater head          & [m] \\ \hline 
gwbottom & level of the bottom of the aquifer          & [m] \\ \hline 
gwhead & groundwater head (differs from gwlevel)            & [m] \\ \hline 
gwheight & vertical size of aquifer through which groundwater can flow        & [m] \\ \hline 
gwlevel & groundwater table  & [m] \\ \hline 
gwu & groundwater flow in x-direction             & [m/s] \\ \hline 
gwv & groundwater flow in y-direction             & [m/s] \\ \hline 
gww & groundwater flow in z-direction (interaction between surface and ground water)       & [m/s] \\ \hline 
H & Hrms wave height based on instantaneous wave energy         & [m] \\ \hline 
hh & water depth               & [m] \\ \hline 
hold & water depth previous time step            & [m] \\ \hline 
hu & water depth in u-points             & [m] \\ \hline 
hum & water depth in u-points             & [m] \\ \hline 
hv & water depth in v-points             & [m] \\ \hline 
hvm & water depth in v-points             & [m] \\ \hline 
idrift & Drifter x-coordinate in grid space            & [-] \\ \hline 
jdrift & Drifter y-coordinate in grid space            & [-] \\ \hline 
k & wave number               & [rad/m] \\ \hline 
kb & near bed turbulence intensity due to depth induces breaking        & [m\^{}2/s\^{}2] \\ \hline 
kturb & depth averaged turbulence intensity due to long wave breaking        & [m\^{}2/s\^{}2] \\ \hline 
L1 & wave length (used in dispersion relation)           & [m] \\ \hline 
maxzs & maximum elevation in simulation             & [m] \\ \hline 
minzs & minimum elevation in simulation             & [m] \\ \hline 
n & ratio group velocity/wave celerity             & [-] \\ \hline 
nd & number of bed layers (can be different for each computational cell)      & [-] \\ \hline 
ndist & cum. distance from right boundary along n-direction          & [m] \\ \hline 
nuh & horizontal viscosity coefficient              & [m2/s] \\ \hline 
pdisch & Discharge locations               & [-] \\ \hline 
ph & pressure head due to ship            & [m] \\ \hline 
pntdisch & Point discharge locations (no momentum)            & [-] \\ \hline 
pres & normalized dynamic pressure              & [m\^{}2/s\^{}2] \\ \hline 
Qb & fraction breaking waves              & [-] \\ \hline 
qdisch & Discharges                & [m\^{}2/s] \\ \hline 
qx & discharge in u-points, x-component             & [m2/s] \\ \hline 
qy & discharge in u-points, y-component             & [m2/s] \\ \hline 
R & roller energy               & [Nm/m2] \\ \hline 
rolthick & long wave roller thickness             & [m] \\ \hline 
rr & directionally distributed roller energy             & [J/m2/rad] \\ \hline 
sdist & cum. distance from offshore boundary along s-direction          & [m] \\ \hline 
sedero & cum. sedimentation/erosion               & [m] \\ \hline 
sigm & mean frequency               & [rad/s] \\ \hline 
sigt & relative frequency               & [rad/s] \\ \hline 
sinth & sin of wave angles relative to grid direction         & [-] \\ \hline 
Sk & skewness of short waves             & [-] \\ \hline 
structdepth & Depth of structure in relation to instantaneous bed level        & [m] \\ \hline 
Subg & bed sediment transport for each sediment class (excluding pores), x-component       & [m2/s] \\ \hline 
Susg & suspended sediment transport for each sediment class (excluding pores), x-component       & [m2/s] \\ \hline 
Sutot & Sediment transport integrated over bed load and suspended and for all sediment grains, x-component   & [m2/s] \\ \hline 
Svbg & bed sediment transport for each sediment class (excluding pores), y-component       & [m2/s] \\ \hline 
Svsg & suspended sediment transport for each sediment class (excluding pores), y-component       & [m2/s] \\ \hline 
Svtot & Sediment transport integrated over bed load and suspended and for all sediment grains, y-component   & [m2/s] \\ \hline 
Sxx & radiation stress, x-component              & [N/m] \\ \hline 
Sxy & radiation stress, y-component              & [N/m] \\ \hline 
Syy & radiation stress, y-component              & [N/m] \\ \hline 
taubx & bed shear stress, x-component             & [N/m\^{}2] \\ \hline 
tauby & bed shear stress, y-component             & [N/m\^{}2] \\ \hline 
Tbore & wave period interval associated with breaking induced turbulence         & [s] \\ \hline 
tdisch & Discharge time series              & [-] \\ \hline 
tdrifter & Drifter retrieval time              & [s] \\ \hline 
theta & wave angles               & [rad] \\ \hline 
theta & wave angles directional distribution w.r.t. comp. x-axis          & [rad] \\ \hline 
theta0 & mean incident wave angle             & [rad] \\ \hline 
thetamax & minimum angle of computational wave grid (cart. in rad)        & [rad] \\ \hline 
thetamean & mean wave angle              & [rad] \\ \hline 
thetamin & minimum angle of computational wave grid (cart. in rad)        & [rad] \\ \hline 
tideinpt & input time of input tidal signal          & [s] \\ \hline 
tideinpz & input tidal signal              & [m] \\ \hline 
tidelen & length of tide time series            & [-] \\ \hline 
tm & mean wave direction              & [rad] \\ \hline 
Tsg & sediment response time for each sediment class          & [s] \\ \hline 
u & GLM velocity in cell center, x-component           & [m/s] \\ \hline 
ua & time averaged flow velocity due to wave asymmetry         & [m/s] \\ \hline 
ucrcal & calibration factor for u critical for each sediment class        & [-] \\ \hline 
ue & Eulerian velocity in cell center, x-component           & [m/s] \\ \hline 
ueu & Eulerian velocity in u-points, x-component            & [m/s] \\ \hline 
ui & incident bound wave velocity in, x-component           & [m/s] \\ \hline 
umean & long-term mean velocity at bnds in u-points, x-component         & [m/s] \\ \hline 
umwci & velocity (time-averaged) for wci, x-component            & [m/s] \\ \hline 
ur & reflected velocity at bnds in u-points           & [m/s] \\ \hline 
urepb & representative flow velocity for sediment advection and diffusion, x-component        & [m/s] \\ \hline 
ureps & representative flow velocity for sediment advection and diffusion, x-component        & [m/s] \\ \hline 
urms & orbital velocity               & [m/s] \\ \hline 
usd & return flow due to roller after breaker delay         & [m/s] \\ \hline 
ust & Stokes drift               & [m/s] \\ \hline 
ustr & return flow due to roller            & [m/s] \\ \hline 
uu & GLM velocity in u-points, x-component            & [m/s] \\ \hline 
uv & GLM velocity in v-points, x-component            & [m/s] \\ \hline 
uwf & Stokes drift, x-component              & [m/s] \\ \hline 
v & GLM velocity in cell center, y-component           & [m/s] \\ \hline 
vardx & 0 = uniform grid size, 1 = variable grid size       & [-] \\ \hline 
ve & Eulerian velocity in cell center, y-component           & [m/s] \\ \hline 
vev & Eulerian velocity in u-points, y-component            & [m/s] \\ \hline 
vi & incident bound wave velocity in, y-component           & [m/s] \\ \hline 
vmag & velocity magnitude in cell center            & [m/s] \\ \hline 
vmageu & Eulerian velocity magnitude u-points             & [m/s] \\ \hline 
vmagev & Eulerian velocity magnitude v-points             & [m/s] \\ \hline 
vmagu & GLM velocity magnitude u-points             & [m/s] \\ \hline 
vmagv & GLM velocity magnitude v-points             & [m/s] \\ \hline 
vmean & long-term mean velocity at bnds in u-points, y-component         & [m/s] \\ \hline 
vmwci & velocity (time-averaged) for wci, y-component            & [m/s] \\ \hline 
vrepb & representative flow velocity for sediment advection and diffusion, y-component        & [m/s] \\ \hline 
vreps & representative flow velocity for sediment advection and diffusion, y-component        & [m/s] \\ \hline 
vu & GLM velocity in u-points, y-component            & [m/s] \\ \hline 
vv & GLM velocity in v-points, y-component            & [m/s] \\ \hline 
vwf & Stokes drift, y-component              & [m/s] \\ \hline 
wb & vertical velocity at the bottom            & [m/s] \\ \hline 
wetu & mask wet/dry u-points              & [-] \\ \hline 
wetv & mask wet/dry v-points              & [-] \\ \hline 
wetz & mask wet/dry eta-points              & [-] \\ \hline 
wi & Vertical velocity at boundary due to (short) waves         & [m/s] \\ \hline 
winddirts & input wind direction            & [deg nautical] \\ \hline 
windinpt & input time of input wind signal          & [s] \\ \hline 
windlen & length of tide time series            & [-] \\ \hline 
windnv & wind velocity in N direction in v point at current time step     & [m/s] \\ \hline 
windsu & wind velocity in S direction in u point at current time step     & [m/s] \\ \hline 
windvelts & input wind velocity             & [m/s] \\ \hline 
windxts & time series of input wind velocity (not S direction), x-component      & [m/s] \\ \hline 
windyts & time series of input wind velocity (not N direction), y-component      & [m/s] \\ \hline 
wm & mean abs frequency              & [rad/s] \\ \hline 
ws & vertical velocity at the free surface           & [m/s] \\ \hline 
zb & bed level               & [m] \\ \hline 
zb0 & initial bed level              & [m] \\ \hline 
zi & Surface elevation at boundary due to (short) waves         & [m] \\ \hline 
zs & water level               & [m] \\ \hline 
zswci & water level (time-averaged) for wci             & [m] \\ \hline 
\end{tabular}


\paragraph{ Instantaneous spatial output}

\noindent Instantaneous spatial output describes the instantaneous state of variables across the entire model domain at various points in time. To make use of this option the user must specify the number of output variables required using the \textit{nglobalvar} keyword in \textit{params.txt}, immediately followed by the names of the requested variables on a separate line each. The output of three instantaneous grids can look as follows:

\noindent \textbf{params.txt}

\noindent nglobalvar = 3

\noindent zs

\noindent zb

\noindent H


\paragraph{ Time-averaged spatial output}

\noindent Time-averaged spatial output describes the time-averaged state of variables across the entire model domain at various points in time. The user can define the averaging period in \textit{params.txt}. To make use of this option the user must specify the number of output variables required using the \textit{nmeanvar} keyword in \textit{params.txt}, immediately followed by the names of the requested variables on a separate line each. The output of two time-averaged grids may look as follows:

\noindent \textbf{params.txt}

\noindent nmeanvar = 2

\noindent u

\noindent v

\noindent \eject \textit{}


\paragraph{ Fixed point output}

\noindent Fixed point output allows the user to select one or more locations for which a time series of data is stored. This output describes a time-series of one or more variables at one point in the model domain. To make use of this option, the user must specify the number of output locations using the \textit{npoints} keyword in \textit{params.txt}, immediately followed by one line per output location describing the location coordinates given as the x-coordinate and y-coordinate and in world coordinates. XBeach will link the output location to the nearest computational point.

\noindent The user can specify the number and selection of output variables for all points (and run-up gauges, discussed in the following section) using the \textit{npointvar} keyword in \textit{params.txt }to specify the number of output variables, immediately followed by the names of the requested variables on a separate line each.

\noindent Fixed point output significantly reduces the amount of data written to file in each time step and is therefore particularly suitable for high temporal resolution output.

\noindent An example with two output locations is given below. The first point is located on the offshore boundary (x = 0.0) and somewhere in the middle of the model domain in y-direction (y = 800.0). The second point is located on the lateral boundary (y = 1600.0) and somewhere in the middle of the domain in x-direction (x = 2000.0). Both locations have four output variables: \textit{H, zs, zb }and\textit{ D}.

\noindent \textbf{params.txt}

\noindent npoints = 2

\noindent 0. 800. 

\noindent 2000. 1600. 

\noindent 

\noindent npointvar = 4

\noindent H

\noindent zs

\noindent zb

\noindent D


\paragraph{ Run-up gauge output}

\noindent Run-up gauge output describes a time-series of a number of variables at the (moving) waterline. In this case XBeach scans in an x-directional transect defined by the user for the location of the waterline. Output information is recorded for this moving point. This is particularly useful to keep track of run-up levels in cross-shore transects.

\noindent The definition of run-up gauges is similar to the definition of fixed point output. The user needs to specify the number of run-up gauges using the \textit{nrugauge} keyword in \textit{params.txt}, immediately followed by one line per run-up gauge location describing the coordinates of the initial location of the run-up gauge. XBeach will subsequently link the initial run-up gauge location to the nearest computational cross-shore transect rather than just the nearest computational point.

\noindent Run-up gauges share their selection of output variables with regular point output. However, in the case of run-up gauges, XBeach will automatically also include the variables \textit{xw}, \textit{yw} and \textit{zs} to the point output variables, if these variables were not specified using the \textit{npointvar} keyword in \textit{params.txt}. Note that the user should refer to the \textit{pointvars.idx} output file to check order of output variables for points and run-up gauges.

\noindent An example of a run-up gauge input is given below. The run-up gauge is initially located on the offshore boundary (x = 0.0) and somewhere in the middle of the model domain in y-direction (y = 800.0). The run-up gauge will display standard output variables (\textit{xw}, \textit{yw} and \textit{zs}, as well as any output variables specified by the \textit{npointvar} keyword).

\noindent \textbf{params.txt}

\noindent nrugauge = 1

\noindent 0. 800. 


\subsubsection{ Output times}

\noindent The user may determine the output times for regular spatial output variables, time averaged spatial variables and point location variables individually. Run-up gauge output and fixed point output are given at the same moments in time. For all three types of output the user may choose to either state a fixed interval time at which output is given or supply an external file containing times at which output should be given or a combination of both. 


\paragraph{ Output at fixed intervals}

\noindent The user should define a point in time after the start of the simulation at which the first output is generated for fixed interval output. The user can do this by using the \textit{tstart} keyword in \textit{params.txt}. All output that is being generated at fixed intervals uses \textit{tstart} as their base. The interval for instantaneous spatial output is given by the \textit{tintg} keyword. The keywords for the interval of time-averaged spatial output and point output are \textit{tintm} and \textit{tintp} respectively, where \textit{tintp} is used both for fixed point and run-up gauge output. Note that \textit{tintg}, \textit{tintm} and \textit{tintp} supersede the older \textit{tint} parameter that is valid for all types of output. The default value of \textit{tintg} is one second. If \textit{tintp} or \textit{tintm} is not stated, but output is declared (\textit{npoints}, \textit{nrugauge} or \textit{nmeanvar} is stated larger than zero), XBeach assumes the same output interval as \textit{tintg}. An example of the definition of fixed intervals is given below.

\noindent \textbf{params.txt}

\noindent tstart = 100.

\noindent tintg = 100.

\noindent tintp = 2.

\noindent tintm = 3600.

\noindent In the case of instantaneous spatial output and point output, the first output is given at \textit{tstart}. In the case of time-averaged spatial variables, the first output is given at \textit{tstart}+\textit{tintm}. This output represents the average condition over the interval between \textit{tstart} and \textit{tstart}+\textit{tintm}. 


\paragraph{ Output times defined by external file}

\noindent The user is given the option to have output at a set of points in time that are not separated by regular intervals. In this case the user must supply an additional file for each output type. The user specifies the name of the output time series file for instantaneous spatial output using the \textit{tsglobal} keyword. The keywords for time series files for time-averaged spatial output and point output are \textit{tsmean} and \textit{tspoint} respectively, where \textit{tspoint} is again used for both fixed point and run-up gauge output. All time series files must contain on the first line the number of output times followed by every output time on a new line. An example of such irregular output time definition is given below.

\noindent \textbf{params.txt}

\noindent tsglobal= time series1.txt

\noindent tspoints = time series2.txt

\noindent tsmean= time series3.txt

\noindent \textbf{time series1.txt}
\[18\] 
\[0.05\] 
\[0.15\] 
\[0.2\] 
\[0.8\] 
\[12.0\] 
\[12.5\] 
\[19.124\] 
\[30.\] 
\[60.\] 
\[90.\] 
\[120.\] 
\[150.\] 
\[160.\] 
\[170.\] 
\[177.\] 
\[178.\] 
\[179.\] 
\[180.\] 
In the case of instantaneous spatial output and point output, the first output is given at the first stated point in time. In the case of time-averaged spatial variables, the first output is given at the second stated point in time. This output represents the average condition over the interval between first and second stated point in time. Subsequent averaging is done over every interval.


\paragraph{ Combinations of fixed internal and external files}

\noindent The user is allowed to define certain types of output using fixed intervals and others using external files. The use of an external file supersedes the use of fixed intervals. Note that \textit{tstart} will only apply to output of fixed interval type. An example of mixing fixed and varying output time intervals is given below.

\noindent \textbf{param.txt}

\noindent tstart = 100.

\noindent tintg = 100.

\noindent tspoints = time series2.txt

\noindent tintm = 3600.


\subsubsection{ Output format}

\noindent XBeach supports two types of output: 1) Fortran binary and 2) netCDF. The output format used is determined by the keyword \textit{outputformat}. The use of netCDF output is more convenient since all output (and input) is stored in a single, easy accessible file. Also the netCDF file format is compatible with many programming languages (e.g. Matlab, Python) as well as many visualization tools (e.g. QuickPlot, Morphan). It should be noted that the support for output types in netCDF could be limited for recent functionalities of the XBeach model.


\paragraph{ Fortran binary}

\noindent Output files in Fortran binary format are bare matrix dumps of XBeach' computational matrices. At each output time, one such matrix block is added to the output file. These files can generally be read by binary read functions, like \textit{fread} in Matlab and the \textit{struct} package in Python.

\noindent Output files written in Fortran binary format are given the name \textit{$<$variable$>$.dat}, for example \textit{zs.dat}, for instantaneous spatial output. The only exception is that files containing information about the wave height of the short waves are called \textit{hrms.dat} instead of \textit{H.dat} to maintain backward compatibility. Time-averaged spatial output is stored similarly, but the file names have a suffix indicating the type of averaging \textit{$<$variable$>$\_mean.dat}. For time-averaged spatial output also the variance, minimum and maximum values are stored using the suffixes \_\textit{var}, \_\textit{min} and \_\textit{max} respectively.

\noindent All data corresponding to fixed point locations will be stored in files called \textit{point$<$NNN$>$.dat}. \textit{$<$NNN$>$} represents a number between 001 and 999 corresponding to the order in which the points are declared in \textit{params.txt}. The data files are plain text and contain one row for each output time step. The first position on each row is the time at which the output is given. The subsequent positions in the row are the instantaneous values of the variables at the given point. The order of the variables is equal to the order in which they are defined for that point in \textit{params.txt}. Data corresponding to run-up gauge locations are stored in the same format as fixed point output, but the files are named \textit{rugau$<$NNN$>$.dat}.

\noindent An extra file called \textit{dims.dat} is always written at the start of the simulation in Fortran binary output mode. This file contains the dimensions of the XBeach model. It simply states the following dimensions in order: \textit{nt }(number of output time steps)\textit{, nx }(number of grid cells in x-direction)\textit{, ny }(number of grid cells in y-direction)\textit{, ngd} (number of sediment fractions)\textit{, nd }(number of bed layers)\textit{, ntp }(number of point output time steps)\textit{, ntm }(number of time-averaged output time steps). Subsequently, the irregular time series are stored, if applicable:\textit{ tsglobal }(irregular output times)\textit{, tspoints }(irregular point output times)\textit{ }and\textit{ tsmean }(irregular time-averaged output times). Similarly, a file \textit{xy.dat} is written containing the x- and y- coordinates of the full computational grid.


\paragraph{ netCDF}

\noindent All data in netCDF output is stored in a single output file. By default this file is named \textit{xboutput.nc}, but this name can be chosen freely using the keyword \textit{ncfilename}. The netCDF file holds all output data, dimensions and input data in a single file. It should be noted that netCDF files can hold a multiple time axes. The temporal unit can be specified in the \textit{params.txt} file using the keyword \textit{tunits}. This unit does not affect calculations and is only used for output. An example of the layout of the netcdf file is given below:

\noindent \textbf{xboutput.nc (structure only, no real contents)}

\noindent netcdf xboutput \{

\noindent dimensions:

 x = 565 ;

 y = 101 ;

 wave\_angle = 9 ;

 bed\_layers = 3 ;

 sediment\_classes = 1 ;

 inout = 2 ;

 globaltime = 2 ;

 tidetime = 435 ;

 tidecorners = 2 ;

 windtime = 2 ;

\noindent variables:

 double x(x) ;

  x:units = "m" ;

  x:long\_name = "local x coordinate" ;

 double y(y) ;

  y:units = "m" ;

  y:long\_name = "local y coordinate" ;

 double globaltime(globaltime) ;

  globaltime:units = "s" ;

 double H(globaltime, y, x) ;

  H:units = "m" ;

  H:long\_name = "wave height" ;

 double zs(globaltime, y, x) ;

  zs:units = "m" ;

  zs:long\_name = "water level" ;

 double zb(globaltime, y, x) ;

  zb:units = "m" ;

  zb:long\_name = "bed level" ;

 double ue(globaltime, y, x) ;

  ue:units = "m/s" ;


\subsection{ Time parameters}

\noindent In all XBeach simulations the hydrodynamic simulation starts at time 0. Model output can be postponed until the time specified by the keyword \textit{tstart }(see 0 Table 4.18 Overview of all keywords related to the ship moduleKeyworddescriptiondefaultrangeunitsremarkW1520W3775W5042W6491W7464W8822nship*Number of ships-123-W1520W3775W5042W6491W7464W8822shipfileName of ship data file$<$file$>$W1520W3775W5042W6491W7464W8822Output selection). The simulation stops at the time specified by \textit{tstop}. The time step used in the hydrodynamic simulation is determined based on a given maximum Courant number using the keyword \textit{CFL}. The table below gives an overview of all keywords related to time management:

\noindent \textit{Table 4.21 Overview of all keywords related to time management}

\begin{tabular}{|p{0.7in}|p{1.1in}|p{0.6in}|p{0.8in}|p{0.4in}|p{0.6in}|} \hline 
keyword & description & default & range & units & remark \\ \hline 
CFL & Maximum Courant-Friedrichs-Lewy number & 0.7 & 0.1 - 0.9 & - &  \\ \hline 
tstop & Stop time of simulation, in morphological time & 2000.0 & 1.0 - 1000000.0 & s &  \\ \hline 
tunits+ & Time units in udunits format (seconds since 1970-01-01 00:00:00.00 +1:00) & 's' &  & - &  \\ \hline 
\end{tabular}

\eject \textbf{}


\section{ Bibliography}

\noindent Andrews, D. G., \& Mcintyre, M. E. (1978). An exact theory of nonlinear waves on a Lagrangian-mean flow. Journal of Fluid Mechanics, 89, 609. doi:10.1017/S0022112078002773

\noindent Baldock, T. E., Holmes, P., Bunker, S., \& van Weert, P. (1998). Cross-shore hydrodynamics within an unsaturated surfzone. Coastal Engineering, 34, 173--196.

\noindent Battjes, J. A. (1975). Modelling of turbulence in the surfzone. Symposium on Modelling Techniques, San Francisco, 1050--1061.

\noindent Daly, C., Roelvink, J. A., van Dongeren, A. R., \& McCall, R. T. (2010). Short wave breaking effects on law frequency waves. Proceedings 30th International Conference on Coastal Engineering, San Diego, 1--13.

\noindent Daly, C., Roelvink, J. A., van Dongeren, A. R., van Thiel de Vries, J. S. M., \& McCall, R. T. (2012). Validation of an advective-deterministic approach to short wave breaking in a surf-beat model. Coastal Engineering, 60, 69--83. doi:10.1016/j.coastaleng.2011.08.001

\noindent Darcy, H. (1856). Les fontaines publiques de la ville de dijon. Tech. Rep., Dalmont, Paris.

\noindent De Jong, M. P. C., Roelvink, J. A., \& Breederveld, C. (2013). Numerical modelling of passing-ship effects in complex geometries and on shallow water. Pianc Smart Rivers 2013.

\noindent De Vet, P. L. M. (2014). Modelling sediment transport and morphology during overwash and breaching events. MSc thesis, Delft University of Technology, Delft.

\noindent Deigaard, R. (1993). A note on the three-dimensional shear stress distribution in a surf zone. Coastal Engineering, 20, 157--171. doi:10.1016/0378-3839\eqref{GrindEQ__93_}90059-H

\noindent Deltares. (2011). Delft3D-FLOW: user manual.

\noindent Den Adel, H. (1987). Heranalyse doorlatendheidsmetingen door middel van de forchheimer relatie. Technical Report M 1795/H 195, CO 272550/56, Grondmechanica Delft, Waterloopkundig Laboratorium. Dutch.

\noindent Galappatti, R., \& Vreugdenhill, C. B. (1985). A depth integrated model for suspended transport. Journal for Hydraulic Research, 23\eqref{GrindEQ__4_}, 359--377.

\noindent Guza, R. T., \& Thornton, E. B. (1985). Velocity moments in the nearshore. Coastal Engineering, 111\eqref{GrindEQ__2_}, 235--256.

\noindent Halford, K. (2000). Simulation and interpretation of borehole flowmeter results under laminar and turbulent flow conditions. Proceedings of the Seventh International Symposium on Logging for Minerals and Geotechnical Applications, Golden, Colorado, The Minerals and Geotechnical Logging Society, 157--168.

\noindent Hallermeier, R. J. (1981). Terminal settling velocity of commonly occurring sand grains. Sedimentology, 28, 859--865.

\noindent Harbaugh, A. W. (2005). MODFLOW-2005 , The USGS Modular Ground-Water Model. U.S. Geological Survey Techniques and Methods, 253.

\noindent Holthuijsen, L. H., Booij, N., \& Herbers, T. H. C. (1989). A prediction model for stationary, short-crested waves in shallow water with ambient currents. Coastal Engineering, 13\eqref{GrindEQ__1_}, 23--54. doi:10.1016/0378-3839\eqref{GrindEQ__89_}90031-8

\noindent Janssen, T. T., \& Battjes, J. A. (2007). A note on wave energy dissipation over steep beaches. Coastal Engineering, 54, 711--716. doi:10.1016/j.coastaleng.2007.05.006

\noindent Komar, P. D., \& Miller, M. C. (1975). On the comparison between the threshold of sediment motion under waves under unidirectional currents with a discussion of the practical evaluation of the threshold. Journal of Sedimentary Research, 362--367.

\noindent Lam, D. C. L., \& Simpson, R. B. (1976). Centered differencing and the box scheme for diffusion convection problems. Journal of Computational Physics, 22, 486--500.

\noindent Lee, K. H., Mizutani, N., Hur, D. S., \& Kamiya, A. (2007). The effect of groundwater on topographic changes in a gravel beach. Ocean Engineering, 34, 605--615. doi:10.1016/j.oceaneng.2005.10.026

\noindent Li, L., \& Barry, D. A. (2000). Wave-induced beach groundwater flow. Advances in Water Resources, 23, 325--337. doi:10.1016/S0309-1708\eqref{GrindEQ__99_}00032-9

\noindent Longuet-Higgins, M. S., \& Stewart, R. W. (1962). Radiation stress and mass transport in gravity waves, with application to ``surf beats.'' Journal of Fluid Mechanics, 13, 481--504.

\noindent Longuet-Higgins, M. S., \& Stewart, R. W. (1964). Radiation stress in water waves: a physical discussion with applications. Deep-Sea Research, 529--562.

\noindent Longuet-Higgins, M. S., \& Turner, J. S. (1974). An ``entraining plume'' model of a spilling breaker. Journal of Fluid Mechanics, 63\eqref{GrindEQ__01_}, 1--20.

\noindent Lowe, R. J., Falter, J. L., Koseff, J. R., Monismith, S. G., \& Atkinson, M. J. (2007). Spectral wave flow attenuation within submerged canopies: Implications for wave energy dissipation. Journal of Geophysical Research: Oceans, 112, 1--14. doi:10.1029/2006JC003605

\noindent McCall, R. T., Masselink, G., Poate, T. G., Roelvink, J. a., Almeida, L. P., Davidson, M., \& Russell, P. E. (2014). Modelling storm hydrodynamics on gravel beaches with XBeach-G. Coastal Engineering, 91, 231--250. doi:10.1016/j.coastaleng.2014.06.007

\noindent Mendez, F. J., \& Losada, I. J. (2004). An empirical model to estimate the propagation of random breaking and nonbreaking waves over vegetation fields. Coastal Engineering, 51, 103--118. doi:10.1016/j.coastaleng.2003.11.003

\noindent Nairn, R. B., Roelvink, J. A., \& Southgate, H. N. (1990). Transition zone width and implications for modeling surfzone hydrodynamics. Proceedings 22th International Conference on Coastal Engineering, 68--81. doi:10.9753/icce.v22.

\noindent Nederhoff, C. M., Lodder, Q. J., Boers, M., Den Bieman, J. P., \& Miller, J. K. (2015). Modeling the effects of hard structures on dune erosion and overwash - a case study of the impact of Hurricane Sandy on the New Jersey coast. Proceedings Coastal Sediments, San Diego, CA.

\noindent Phan, L., van Thiel de Vries, J. S. M., \& Stive, M. J. F. (2014). Coastal Mangrove Squeeze in the Mekong Delta. Journal of Coastal Research.

\noindent Phillips, O. M. (1977). The dynamics of the upper ocean. Cambridge University Press, 366.

\noindent Raubenheimer, B., Guza, R. T., \& Elgar, S. (1999). Tidal water table fluctuations in a sandy ocean beach. Water Resources Research, 35\eqref{GrindEQ__8_}, 2313. doi:10.1029/1999WR900105

\noindent Reniers, A. J. H. M., MacMahan, J. H., Thornton, E. B., \& Stanton, T. P. (2007). Modeling of very low frequency motions during RIPEX. Journal of Geophysical Research: Oceans, 112(February), 1--14. doi:10.1029/2005JC003122

\noindent Reniers, A. J. H. M., Roelvink, J. A., \& Thornton, E. B. (2004). Morphodynamic modeling of an embayed beach under wave group forcing. Journal of Geophysical Research, 109, 1--22. doi:10.1029/2002JC001586

\noindent Rienecker, M. M., \& Fenton, J. D. (1981). A Fourier approximation method for steady water waves. Journal of Fluid Mechanics, 104, 119. doi:10.1017/S0022112081002851

\noindent Roelvink, J. A. (1993a). Dissipation in random wave group incident on a beach. Coastal Engineering, 19, 127--150.

\noindent Roelvink, J. A. (1993b). Surf beat and its effect on cross-shore profiles. PhD thesis, Delft Unversity of Technology, Delft.

\noindent Roelvink, J. A. (2006). Coastal morphodynamic evolution techniques. Coastal Engineering, 53, 277--287. doi:10.1016/j.coastaleng.2005.10.015

\noindent Roelvink, J. A., \& Stive, M. J. F. (1989). Bar-generating cross-shore flow mechanisms on a beach. Journal of Geophysical Research, 94, 4785--4800.

\noindent Rowe, P. N. (1987). A convenient empirical equation for estimation of the richardson-zaki exponent. Chemical Engineering Science, 42\eqref{GrindEQ__11_}, 2795 -- 2796.

\noindent Ruessink, B. G., Miles, J. R., Feddersen, F., Guza, R. T., \& Elgar, S. (2001). Modeling the alongshore current on barred beaches. Journal of Geophysical Research, 106\eqref{GrindEQ__22_}, 451--463.

\noindent Ruessink, B. G., Ramaekers, G., \& van Rijn, L. C. (2012). On the parameterization of the free-stream non-linear wave orbital motion in nearshore morphodynamic models. Coastal Engineering, 65, 56--63. doi:10.1016/j.coastaleng.2012.03.006

\noindent Schroevers, M., Huisman, B. J., van der Wal, A., \& Terwindt, J. (2011). Measuring ship induced waves and currents on a tidal flat in the Western Scheldt Estuary. Current, Waves and Turbulence Measurements (CWTM), 2011 IEEE/OES 10th, 123--129.

\noindent Shields, A. (1936). Anwendung der Aehnlichkeitsmechanik under der Turbulenzforschung auf die Geschiebebewegung. Preussischen Versuchsanstalt Fur Wasserbau and Schiffbau, 26, 524--526.

\noindent Smagorinsky, J. (1963). General circulation experiments wiht the primitive equations I. The basic experiment. Monthly Weather Review, 91, 99--164. doi:10.1126/science.27.693.594

\noindent Smit, P. B., Janssen, T. T., Holthuijsen, L. H., \& Smith, J. (2014). Non-hydrostatic modeling of surf zone wave dynamics. Coastal Engineering, 83, 36--48. doi:10.1016/j.coastaleng.2013.09.005

\noindent Soulsby, R. L. (1997). Dynamics of Marine Sands. London: Thomas Telford Publications.

\noindent Stelling, G. S., \& Zijlema, M. (2003). An accurate and efficient finite-difference algorithm for non-hydrostatic free-surface flow with application to wave propagation. International Journal for Numerical Methods in Fluids, 43\eqref{GrindEQ__1_}, 1--23.

\noindent Stive, M. J. F., \& De Vriend, H. J. (1994). Shear stresses and mean flow in shoaling and breaking waves. Proceedings 24th International Conference on Coastal Engineering, 594--608. doi:10.9753/icce.v24.

\noindent Suzuki, T., Zijlema, M., Burger, B., Meijer, M. C., \& Narayan, S. (2012). Wave dissipation by vegetation with layer schematization in SWAN. Coastal Engineering, 59\eqref{GrindEQ__1_}, 64--71. doi:10.1016/j.coastaleng.2011.07.006

\noindent Svendsen, I. A. (1984a). Mass flux and undertow in a surf zone. Coastal Engineering, 8, 347--365.

\noindent Svendsen, I. A. (1984b). Wave heights and set-up in a surf zone. Coastal Engineering, 8, 303--329. doi:10.1016/0378-3839\eqref{GrindEQ__84_}90028-0

\noindent Talmon, A. M., van Mierlo, M. C., \& Struiksma, N. (1995). Laboratory measurements of the direction of sediment transport on transverse alluvial-bed slope. Journal of Hydraulic Research, 33\eqref{GrindEQ__4_}, 495--517.

\noindent Van Bendegom, L. (1947). Enige beschouwingen over riviermorphologie en rivierverbetering.

\noindent Van der Zwaag, J. (2014). Modelling sediment sorting near the large scale nourishment ' The Sand Motor '. MSc thesis, Delft University of Technology, Delft.

\noindent Van Dongeren, A. R., \& Svendsen, I. A. (1997). Absorbing-generating boundary condition for shallow water models. Journal of Waterway, Port, Coastal and Ocean Engeering, \eqref{GrindEQ__123_}, 303--313.

\noindent Van Rhee, C. (2010). Sediment entrainment at high flow velocity. Journal of Hydraulic Engineering, 136, 572--582.

\noindent Van Rijn, L. C. (1985). Sediment transport, part III: bed forms and alluvial roughness. Journal of Hydraulic Engineering, 110\eqref{GrindEQ__12_}, 1733--1754.

\noindent Van Rijn, L. C. (2007). Unified View of Sediment Transport by Currents and Waves: part I and II. Journal of Hydraulic Engineering, (June), 649--667.

\noindent Van Rooijen, A. A., Van Thiel de Vries, J. S. M., McCall, R. T., van Dongeren, A. R., Roelvink, J. A., \& Reniers, A. J. H. M. (2015). Modeling of wave attenuation by vegetation with XBeach. E-Proceedings of the 36th IAHR World Congress 28 June -- 3 July, 2015, The Hague, The Netherlands.

\noindent Van Thiel de Vries, J. S. M. (2009). Dune erosion during storm surges. PhD thesis, Delft Unversity of Technology, Delft.

\noindent Verboom, G. K., Stelling, G. S., \& Officer, M. J. (1981). Boundary conditions for the shallow water equations. Engineering Applications of Computational Hydraulics, 230--262.

\noindent Walstra, D. J. R., Roelvink, J. A., \& Groeneweg, J. (2000). Calculation of wave-driven currents in a 3D mean flow model. In Proceedings 27th International Conference on Coastal Engineering (pp. 1050--1063).

\noindent Walstra, D. J. R., van Rijn, L. C., Van Ormondt, M., Briere, C., \& Talmon, A. M. (2007). The Effects of Bed Slope and Wave Skewness on Sediment Transport and Morphology (pp. 137--150).

\noindent Zhou, M., Roelvink, J. A., Verheij, H. J., \& Ligteringen, H. (2013). Study of Passing Ship Effects along a Bank by Delft3D-FLOW and XBeach. International Workshop on Nautical Traffic Models 2013, Delft, The Netherlands, July 5-7, 2013. Delft University of Technology.

\noindent Zhou, M., Roelvink, J. A., Zou, Z., \& van Wijhe, H. J. (2014). Effects of Passing Ship With a Drift Angle on a Moored Ship. ASME 2014 33rd International Conference on Ocean, Offshore and Arctic Engineering.

\noindent Zijlema, M., Stelling, G. S., \& Smit, P. B. (2011). SWASH: An operational public domain code for simulating wave fields and rapidly varied flows in coastal waters. Coastal Engineering, 58\eqref{GrindEQ__10_}, 992--1012. doi:10.1016/j.coastaleng.2011.05.015

\noindent \eject 


\section{ Appendices}


\subsection{ Hands on exercises (based on basic XBeach of the Delft Software Days 2014)}

\noindent The hands-on exercises can be downloaded via subversion.  Subversion is a well-known version management system that allows you to always have the most recent source code at hand. It also allows developers to commit changes to the source code, without interfering with other developers. In order to use Subversion, you will need a Subversion client. A well-known client for Windows is Tortoise. If you have registered, you can download the source code via the following URL:\textit{ https://svn.oss.deltares.nl/repos/xbeach/Courses/DSD\_2014/Examples -- Basic}. For the tools like Quickplot and Quickin the Delft3D environment is needed.


\subsubsection{ Dune erosion at Delfland, Netherlands (1D)}

\noindent The first case we will run is a relative simple 1D case. It concerns a profile along the Dutch coast and the hydraulic boundary conditions are based on the 1953 storm surge that caused substantial flooding in the Netherlands. 

\noindent 

\noindent You can work on the following assignments:

\noindent 

\begin{enumerate}
\item  Go to the folder ``Examples\textbackslash DelflandStorm'' and double click the file ``run\_model.bat''. The simulation will start. The model will run for a few minutes, but in the meantime you can already work on question 2 to 5.

\item  Open params.txt in which you specify the model input files and settings. Check the number of grid-points in x-direction (keyword: \textit{nx}) and y-direction (keyword: \textit{ny}). Check the filenames in which you specify the wave conditions (keyword: \textit{bcfile}) and the storm surge level (SSL) (keyword: \textit{zs0file}). 

\item  Do the wave conditions change during the simulation? What is/are the wave height(s) and wave period(s) applied in the simulation?

\item  Does the storm surge level change during the simulation? What is the maximum surge height in the simulation. Surge height is defined with respect to the mean sea level (MSL)?

\item  What is the simulation time (keyword: \textit{tstop})? Do we apply a morphological acceleration factor (keyword: \textit{morfac})? What variables are stored as output and with what time interval? How much hydrodynamic time is simulated?

\item  Probably the simulation has finished. When you start the model, it generates a file named XBlog.txt. Open this file and check what is stored in the file. What was the total simulation time?

\item  To check out the simulation results we make use of the Quickplot tool (A brief tutorial is attached to this document). You can start Quickplot via the Delft3D environment we installed (Start  Programs  Deltares  Delft3D  Delft3D). In the Delft 3D menu choose Utilities  Quickplot. Choose Files of type ``NetCDF files and GRIB files'' and open ``xboutput.nc'' in the simulation folder. 

\item  Use the Quickplot tutorial and try to make an animation in which you plot short wave height (H), water level (including long wave variations, zs) and bed level (zb) as function of time.

\item  Plot the offshore water level as function of time. Also open the file ``tide.tek'' (Tekal data files format), which contains the imposed surge level. Did the model correctly simulate the imposed surge level?

\item  Copy all model files to a new folder named ``superfast''. Edit params.txt and set ny=0 (instead of ny=2), and run the model. What is the simulation time compare to the original simulation?

\item  Compare simulation results for the ``superfast'' and ``default'' simulation. Are these the same? What option will you use in the future?  
\end{enumerate}


\subsubsection{ Nourishment scenarios near Kijkduin, Holland (1D) }

\noindent This case concerns the exploration of a nourishment strategy near Kijkduin along the Holland coast in the Netherlands. At this location a mega nourishment of 21 Mm${}^{3}$ named the Sand Engine was constructed. In this case we will explore to what extent nourishments can reduce the (dune and beach) erosion during a storm event. 

\noindent 

\noindent You can work on the following assignments:

\noindent 

\begin{enumerate}
\item  Go to the folder ``Examples\textbackslash Nourishment case'' and double click on the file ``runall.bat''. This batch file will run three simulations sequentially in which the profile configuration varies and corresponds with the undisturbed profile (folder reference), a shoreface nourishment (folder shoreface) and a beach nourishment (folder beach) respectively. Each model will run for a few minutes. While running you can already answer question 2 to 6.

\item  For the reference case open the params.txt in which you specify model input files and settings. Check the number of grid-points in x-direction (keyword: \textit{nx}) and y-direction (keyword: \textit{ny}). How many directional wave bins are defined and what is their width (keywords: \textit{thetamin}, \textit{thetamax}, \textit{dtheta}).

\item  Do the wave conditions change during the simulation? What is/are the wave height(s) and wave period(s) applied in the simulation?

\item  Does the storm surge level change during the simulation? What is the maximum surge height in the simulation. Surge height is defined with respect to the mean sea level (MSL)?

\item  What is the simulation time (keyword: \textit{tstop})? Do we apply a morphological acceleration factor (keyword: \textit{morfac})? What variables are stored as output and with what time interval? How much hydrodynamic time is simulated?

\item  Probably the simulation has finished. When you start the model, it generates a file named XBlog.txt. Open this file and check what is stored in the file. What was the total simulation time?

\item  Inspect the initial bathymetries of each simulation with QUICKPLOT. Choose Files of type ``NetCDF files and GRIB files'' and open ``xboutput.nc'' in the simulation folder). 

\begin{enumerate}
\item  At what cross-shore position were the shoreface nourishment and beach nourishment placed? 

\item  What is the (average) thickness of the nourishments? 

\item  Is the volume of the nourishments comparable? 

\item  Plot the reference profile with markers; does the grid resolution vary in cross-shore direction?    
\end{enumerate}

\item  Use the Quickplot tutorial and try to make an animation in which you plot short wave height (H), water level (including long wave variations, zs) and bed level (zb) as function of time.

\item  Plot the offshore water level as function of time. Also open the file ``tide.tek'' (Tekal data files format), which contains the imposed surge level. Did the model correctly simulate the imposed surge level?

\item  Inspect the final bathymetries of each simulation.

\begin{enumerate}
\item  What is the dune face retreat in the three simulations you have carried out?

\item  Where does the eroded sediment form the dunes deposit?

\item  What nourishment type is most effective in reducing the impact of a storm and do you have an explanation for this?
\end{enumerate}

\item  In the folder ``banquette'' you find a final simulation in which a special beach nourishment type is evaluated named a banquette. This beach nourishment has a highly elevated flat area that connects to the dune foot on which beach restaurants can be build.

\begin{enumerate}
\item  Run the model and compare in Quickplot the banquette design with the beach nourishment design we have evaluated before. Do you expect more or less erosion?

\item  Check your hypothesis by comparing the final profile of the banquette simulation to the other simulations.

\item  What would be your approach to further reduce beach and dune erosion? 
\end{enumerate}
\end{enumerate}


\subsubsection{ Overwash at Santa Rosa Island , USA (2DH) }

\noindent This case concerns overwash at Santa Rosa island in the Gulf of Mexico during hurricane Ivan in 2004. 

\noindent 

\noindent You can work on the following assignments.

\noindent \textbf{}

\begin{enumerate}
\item \textbf{ }For the reference case open the params.txt in which you specify model input files and settings. Check the number of grid-points in x-direction (keyword: \textit{nx}) and y-direction (keyword: \textit{ny}). How many directional wave bins are defined and what is their width (keywords: \textit{thetamin}, \textit{thetamax}, \textit{dtheta}).

\item  In this simulation the grid is specified in Delft3D format. Open Quickin in the Delft 3D menu (Grid  Quickin) and use the brief tutorial to read in the grid and bathymetry. Does the grid resolution vary in cross-shore direction? And in longshore direction? What are the minimum dx and dy? Why can the grid be coarse offshore?

\item  How many wave conditions do we apply in this simulation? What is the offshore mean wave direction? Does the surge level change in the simulation?

\item  What is the simulation time (hydrodynamic and morphologic)?

\item  Inspect the model results and make an animation of the short wave height (H) and the water levels (including long wave, zs). Describe what is happening.

\item  For the water levels set the color limits manual between -0.5 and 3.5.

\item  Make an animation of cumulative sedimentation/erosion. Describe what is happening.

\item  For the sedimentation/erosion set the color limits manual between -3 and 3

\item  Look at the mean flow field. Plot the flow field in colored vectors. Where are the flow velocities highest and what is the direction of the flow (cross-shore or longshore)? Is there (also) a longshore current present and what is its intensity?
\end{enumerate}

\noindent 

\noindent If you have time left feel free to: 

\begin{enumerate}
\item  Narrow or broaden the imposed spectrum by changing the parameter directional spreading (\textit{s}) in `jonswap.inp' (you could for example set s = 100 and s = 2 respectively). Make animations of the instantaneous short wave height to see what is happening to the size of the wave groups.  

\item  Design a nourishment in Quickin to reduce the impact of the storm on Santa Rosa Island. Change the depth file in params.txt to make a simulation with the updated bathymetry.
\end{enumerate}


\subsubsection{ Yanchep perched beach and natural breakwater (2DH) }

\noindent This case is an example of a beach 60km north of Perth most commonly known as Yanchep lagoon. Many beaches in WA like Yanchep are fronted by shallow reef and here we are investigating the effects of the reef on the morphodynamics. 

\noindent 

\noindent You can work on the following assignments:

\noindent \textbf{}

\begin{enumerate}
\item \textbf{ }Go to the folder ``Examples\textbackslash YanchepBeach'' and double click the file ``run\_model.bat''. The simulation will start (and will run about 15 minutes).

\item  Meanwhile, inspect the bathymetry file and the structure file (using Quickin). What is the depth in the lagoon? Is the reef enclosing the lagoon below or above the model initial water level? What is the wave height at the boundary condition?

\item  Use  Quickplot and try to make an animation in which you plot short wave height (H), water level (including long wave variations) (zs) and Eulerian velocities (ue and ve) as function of time. What happens in the lagoon?

\item  Use  Quickplot and try to make an animation of cumulative sedimentation/erosion. What happens in the lagoon?

\item  How is the lagoon affected by the mean water level? Increase or decrease the mean water level condition (`tide.tx'), run the model again (maybe for a shorter time by reducing keyword: \textit{tstop}). How are the circulation and sediment transport affected?

\item  What would happen if the lagoon was open at the southern end? Open the structure file (keyword: ne\_layer='reef.dep') with the Quickin tool and modify it to allow the southern end of the lagoon to be eroded. Modify the param.txt file to use this new structure file and run the model. Alternatively, remove the reef from the bathymetry and rerun the model without the structure file, by setting the keyword \textit{struct=0}.
\end{enumerate}

\noindent 

\noindent If you still have time;

\begin{enumerate}
\item  Reefs are very rough what happens in the model when the friction is increased? Reduce the Chezy roughness and increase the value of \textit{f${}_{w}$}. Rerun the model what do you observe?

\item  Is wave/current interaction (keyword: \textit{wci=1}) switched on? Rerun the model with the wave/current switch on/off. Compare the output with model you ran previously. How much effect do you see on the morphology?
\end{enumerate}

\noindent \eject \textbf{}


\subsection{ Advanced model coefficients}

\noindent In 4.1 the main input parameters and files required by XBeach to start a simulation are explained. It explained how the user can switch on and off specific processes and how the user can define the model initial and boundary conditions. XBeach offers, however, many more parameters to fine-tune the simulation of different processes. These parameters are listed in the following subsections grouped by process. Most parameters are not relevant for the average XBeach user. Parameters marked with a plus (+) are considered advanced options that are recommended to stay untouched unless you know what you are doing.


\subsubsection{ Wave numerics}

\noindent The parameters listed in the table below involve the numerical aspects of the wave action balance that solves the wave propagation in the model. The keyword \textit{scheme }can be used to set the numerical scheme. By default a higher-order upwind scheme is used to minimize numerical dissipation.

\noindent \textit{Table 6.1 Overview of available keyword related to wave numerics}

\begin{tabular}{|p{0.7in}|p{1.0in}|p{0.6in}|p{0.9in}|p{0.4in}|p{0.6in}|} \hline 
keyword & description & default & range & units & remark \\ \hline 
maxerror+ & Maximum wave height error in wave stationary iteration & 5e-05 & 1e-05 - 0.001 & m &  \\ \hline 
maxiter+ & Maximum number of iterations in wave stationary & 500 & 2 - 1000 & - &  \\ \hline 
scheme+ & Numerical scheme for wave propagation & upwind\_2 & upwind\_1, lax\_wendroff, upwind\_2 &  &  \\ \hline 
wavint & Interval between wave module calls (only in stationary wave mode) & 60.0 & 1.0 - 3600.0 & s &  \\ \hline 
\end{tabular}


\subsubsection{ Wave dissipation}

\noindent The parameters listed in the table below involve the wave dissipation process. For instationary model runs use either \textit{break=roelvink1},\textit{ roelvink2} or\textit{ roelvink\_daly.} Note that the standard value \textit{gamma=0.55} and \textit{n=10} was calibrated for option \textit{break=roelvink1}. For \textit{break=roelvink2} the wave dissipation is proportional to H${}^{3}$/h instead of H${}^{2}$; this affects the calibration. For stationary runs the \textit{break=baldock} option is suitable. The \textit{break=roelvink\_daly} option is a model in which waves start and stop breaking. Reducing \textit{gammax} will reduce wave heights in very shallow water, probably 2 is a reasonable value.

\noindent \textit{Table 6.2 Overview of available keyword related to the wave dissipation model}

\begin{tabular}{|p{0.8in}|p{0.8in}|p{0.6in}|p{0.9in}|p{0.4in}|p{0.5in}|} \hline 
keyword & description & default & range & units & remark \\ \hline 
alpha+ & Wave dissipation coefficient in Roelvink formulation & 1.0 & 0.5 - 2.0 & - &  \\ \hline 
break & Type of breaker formulation & roelvink2 & roelvink1, baldock, roelvink2, roelvink\_daly, janssen &  &  \\ \hline 
breakerdelay+ & Switch to enable breaker delay model & 1 & 0 - 1 & - &  \\ \hline 
delta+ & Fraction of wave height to add to water depth & 0.0 & 0.0 - 1.0 & - &  \\ \hline 
facrun+ & Calibration coefficient for short wave runup & 1.0 & 0.0 - 2.0 & - &  \\ \hline 
facsd+ & Fraction of the local wave length to use for shoaling delay depth & 1.0 & 0.0 - 2.0 & - &  \\ \hline 
fw+ & Bed friction factor & 0.0 & 0.0 - 1.0 & - &  \\ \hline 
fwcutoff & Depth greater than which the bed friction factor is not applied & 1000.0 & 0.0 - 1000.0 & - &  \\ \hline 
gamma & Breaker parameter in Baldock or Roelvink formulation & 0.55 & 0.4 - 0.9 & - &  \\ \hline 
gamma2 & End of breaking parameter in Roelvink Daly formulation & 0.3 & 0.0 - 0.5 & - &  \\ \hline 
gammax+ & Maximum ratio wave height to water depth & 2.0 & 0.4 - 5.0 & - &  \\ \hline 
n+ & Power in Roelvink dissipation model & 10.0 & 5.0 - 20.0 & - &  \\ \hline 
shoaldelay+ & Switch to enable shoaling delay & 0 & 0 - 1 & - &  \\ \hline 
\end{tabular}


\subsubsection{ Rollers}

\noindent The parameters listed in the table below involve the wave roller model. Using the roller model will give a shoreward shift in wave-induced setup, return flow and alongshore current. This shift becomes greater for lower \textit{beta} values.

\noindent \textit{Table 6.3 Overview of available keyword related to the roller model}

\begin{tabular}{|p{0.7in}|p{1.1in}|p{0.6in}|p{0.7in}|p{0.4in}|p{0.7in}|} \hline 
keyword & description & default & range & units & remark \\ \hline 
beta+ & Breaker slope coefficient in roller model & 0.1 & 0.05 - 0.3 & - &  \\ \hline 
rfb+ & Switch to feedback maximum wave surface slope in roller energy balance, otherwise rfb = Beta & 0 & 0 - 1 & - &  \\ \hline 
roller+ & Switch to enable roller model & 1 & 0 - 1 & - &  \\ \hline 
\end{tabular}


\subsubsection{ Wave-current interaction}

\noindent The parameters listed in the table below involve the process of wave-current interaction. With the switch \textit{wci} one can turn off or on the wave-current interaction, the wave current interaction will result in a feedback of currents on the wave propagation. On top of that, \textit{hwci} limits the computation of wave-current interaction in very shallow water where the procedure may not converge.

\noindent \textit{Table 6.4 Overview of available keyword related to the wave-current interaction (wci)}

\begin{tabular}{|p{0.7in}|p{1.1in}|p{0.6in}|p{0.7in}|p{0.4in}|p{0.7in}|} \hline 
keyword & description & default & range & units & remark \\ \hline 
cats+ & Current averaging time scale for wci, in terms of mean wave periods & 4.0 & 1.0 - 50.0 & Trep &  \\ \hline 
hwci+ & Minimum depth until which wave-current interaction is used & 0.1 & 0.001 - 1.0 & m &  \\ \hline 
wci & Turns on wave-current interaction & 0 & 0 - 1 & - &  \\ \hline 
\end{tabular}


\subsubsection{ Bed friction and viscosity}

\noindent The parameters listed in the table below involve the settings for bed friction and viscosity influencing the flow in XBeach. The bed friction is influenced by the dimensionless friction coefficient \textit{cf} or other formulation like the dimensional Ch�zy or Manning. The bed friction formulation applied needs to be determined with the keyword \textit{bedfriction}. It is possible both to define one value (keyword: \textit{bedfriccoef}) or to apply, spatially varying values for the bed friction. A spatial varying friction can be provided through an external file referenced via the keyword \textit{bedfricfile}. The file has the same format as the bathymetry file explained in 4.3 (Grid and bathymetry).

\noindent The horizontal viscosity is composed of an overall background viscosity \textit{nuh} and a viscosity depending on the roller dissipation tuned by \textit{nuhfac}. In the alongshore direction the viscosity may be multiplied by a factor \textit{nuhv} to account for additional advective mixing. It is also possible to use a user-defined value for the horizontal viscosity (keyword \textit{smag = 0})

\noindent \textit{Table 6.5 Overview of available keyword related to the bed friction and viscosity }

\begin{tabular}{|p{0.8in}|p{0.9in}|p{0.6in}|p{0.8in}|p{0.6in}|p{0.6in}|} \hline 
keyword & description & default & range & units & remark \\ \hline 
bedfriccoef & Bed friction coefficient & 0.01 & 3.5e-05 - 0.9 & - &  \\ \hline 
bedfricfile & Bed friction file  &  &  & $<$file$>$ &  \\ \hline 
bedfriction & Bed friction formulation & chezy & chezy, cf, white-colebrook, manning, white-colebrook-grainsize &  &  \\ \hline 
nuh & Horizontal background viscosity & 0.1 & 0.0 - 1.0 & m\^{}2s\^{}-1 &  \\ \hline 
nuhfac+ & Viscosity switch for roller induced turbulent horizontal viscosity & 1.0 & 0.0 - 1.0 & - &  \\ \hline 
nuhv+ & Longshore viscosity enhancement factor, following Svendsen  & 1.0 & 1.0 - 20.0 & - &  \\ \hline 
smag+ & Switch for smagorinsky sub grid model for viscosity & 1 & 0 - 1 & - &  \\ \hline 
\end{tabular}


\subsubsection{ Flow numerics}

\noindent The parameters listed in the table below involve the numerical aspects of the shallow water equations that solve the water motions in the model. Especially in very shallow water some processes need to be limited to avoid unrealistic behavior. For example \textit{hmin} prevents very strong return flows or high concentrations and the \textit{eps} determines whether points are dry or wet and can be taken quite small. 

\noindent \eject 

\noindent \textit{Table 6.6 Overview of available keyword related to flow numerics}

\begin{tabular}{|p{0.7in}|p{1.1in}|p{0.6in}|p{0.7in}|p{0.4in}|p{0.6in}|} \hline 
keyword & description & default & range & units & remark \\ \hline 
eps & Threshold water depth above which cells are considered wet & 0.005 & 0.001 - 0.1 & m &  \\ \hline 
eps\_sd & Threshold velocity difference to determine conservation of energy head versus momentum & 0.5 & 0.0 - 1.0 & m/s &  \\ \hline 
hmin & Threshold water depth above which Stokes drift is included & 0.2 & 0.001 - 1.0 & m &  \\ \hline 
oldhu+ & Switch to enable old hu calculation & 0 & 0 - 1 & - &  \\ \hline 
secorder+ & Use second order corrections to advection/non-linear terms based on MacCormack scheme & 0 & 0 - 1 & - &  \\ \hline 
umin & Threshold velocity for upwind velocity detection and for vmag2 in equilibrium sediment concentration & 0.0 & 0.0 - 0.2 & m/s &  \\ \hline 
\end{tabular}


\subsubsection{ Sediment transport}

\noindent The parameters listed in the table below involve the process of sediment transport. The keywords \textit{facAs }and \textit{facSk} determine the effect of the wave form on the sediment transport, this is especially important in the nearshore. The \textit{facua }is an alias setting in which both parameters can be varied at once. The wave form model itself is selected using the keyword \textit{waveform}. Processes like short- and long-wave stirring and turbulence can be switched on or off using the keywords \textit{sws, lws }and \textit{lwt}. Several options for calibrating the sediment transport formulations are available as well as keywords to incorporate the bed slope effect.

\noindent \textit{Table 6.7 Overview of available keyword related to the sediment transport model}

\begin{tabular}{|p{0.7in}|p{0.8in}|p{0.8in}|p{0.9in}|p{0.5in}|p{0.5in}|} \hline 
keyword & description & default & range & units & remark \\ \hline 
BRfac+ & Calibration factor surface slope & 1.0 & 0.0 - 1.0 & - &  \\ \hline 
Tbfac+ & Calibration factor for bore interval Tbore: Tbore = Tbfac+Tbore & 1.0 & 0.0 - 1.0 & - &  \\ \hline 
Tsmin+ & Minimum adaptation time scale in advection diffusion equation sediment & 0.5 & 0.01 - 10.0 & s &  \\ \hline 
bdslpeffdir & Modify the direction of the sediment transport based on the bed slope & none & none, talmon &  &  \\ \hline 
bdslpeffdirfac & Calibration factor in the modification of the direction & 1.0 & 0.0 - 2.0 & - &  \\ \hline 
bdslpeffini & Modify the critical shields parameter based on the bed slope & none & none, total, bed &  &  \\ \hline 
bdslpeffmag & Modify the magnitude of the sediment transport based on the bed slope, uses facsl & roelvink\_total & none, roelvink\_total, roelvink\_bed, soulsby\_total, soulsby\_bed &  &  \\ \hline 
bed+ & Calibration factor for bed transports & 1 & 0 - 1 & - &  \\ \hline 
betad+ & Dissipation parameter long wave breaking turbulence & 1.0 & 0.0 - 10.0 & - &  \\ \hline 
bulk+ & Switch to compute bulk transport rather than bed and suspended load separately & 0 & 0 - 1 & - &  \\ \hline 
dilatancy & Switch to reduce critical shields number due dilatancy & 0 & 0 - 1 & - &  \\ \hline 
facAs+ & Calibration factor time averaged flows due to wave asymmetry & 0.1 & 0.0 - 1.0 & - &  \\ \hline 
facDc+ & Option to control sediment diffusion coefficient & 1.0 & 0.0 - 1.0 & - &  \\ \hline 
facSk+ & Calibration factor time averaged flows due to wave skewness & 0.1 & 0.0 - 1.0 & - &  \\ \hline 
facsl+ & Factor bedslope effect & 1.6 & 0.0 - 1.6 & - &  \\ \hline 
facua+ & Calibration factor time averaged flows due to wave skewness and asymmetry & 0.1 & 0.0 - 1.0 & - &  \\ \hline 
fallvelred & Switch to reduce fall velocity for high concentrations & 0 & 0 - 1 & - &  \\ \hline 
form & Equilibrium sediment concentration formulation & vanthiel\_vanrijn & soulsby\_vanrijn, vanthiel\_vanrijn &  &  \\ \hline 
jetfac+ & Option to mimic turbulence production near revetments & 0.0 & 0.0 - 1.0 & - &  \\ \hline 
lws+ & Switch to enable long wave stirring & 1 & 0 - 1 & - &  \\ \hline 
lwt+ & Switch to enable long wave turbulence & 0 & 0 - 1 & - &  \\ \hline 
pormax & Max porosity used in the expression of Van Rhee & 0.5 & 0.3 - 0.6 & - &  \\ \hline 
reposeangle & Angle of internal friction & 30.0 & 0.0 - 45.0 & deg &  \\ \hline 
rheeA & A parameter in the Van Rhee expression  & 0.75 & 0.75 - 2.0 & - &  \\ \hline 
smax+ & Maximum Shields parameter for equilibrium sediment concentration acc. Diane Foster & -1.0 & -1.0 - 3.0 & - &  \\ \hline 
sus+ & Calibration factor for suspensions transports & 1 & 0 - 1 & - &  \\ \hline 
sws+ & Switch to enable short wave and roller stirring and undertow & 1 & 0 - 1 & - &  \\ \hline 
tsfac+ & Coefficient determining Ts = tsfac + h/ws in sediment source term & 0.1 & 0.01 - 1.0 & - &  \\ \hline 
turb+ & Switch to include short wave turbulence & bore\_averaged & none, wave\_averaged, bore\_averaged &  &  \\ \hline 
turbadv+ & Switch to activate turbulence advection model for short and or long wave turbulence & none & none, lagrangian, eulerian &  &  \\ \hline 
waveform & Wave shape model & vanthiel & ruessink\_vanrijn, vanthiel &  &  \\ \hline 
z0+ & Zero flow velocity level in Soulsby and van Rijn (1997) sediment concentration & 0.006 & 0.0001 - 0.05 & m &  \\ \hline 
\end{tabular}


\subsubsection{ Sediment transport numerics}

\noindent The parameters listed in the table below involve the numerical aspects of sediment transport that are all considered advanced options. For example the maximum allowed sediment concentration can be varied with the keyword \textit{cmax}. It is however not recommended varying these settings.

\noindent \textit{Table 6.8 Overview of available keyword related to sediment transport numerics}

\begin{tabular}{|p{0.8in}|p{1.1in}|p{0.6in}|p{0.7in}|p{0.4in}|p{0.6in}|} \hline 
keyword & description & default & range & units & remark \\ \hline 
cmax+ & Maximum allowed sediment concentration & 0.1 & 0.0 - 1.0 & - &  \\ \hline 
sourcesink+ & Switch to enable source-sink terms to calculate bed level change rather than suspended transport gradients & 0 & 0 - 1 & - &  \\ \hline 
thetanum+ & Coefficient determining whether upwind \eqref{GrindEQ__1_} or central scheme \eqref{GrindEQ__0_5_} is used. & 1.0 & 0.5 - 1.0 & - &  \\ \hline 
\end{tabular}


\subsubsection{ Quasi-3D sediment transport}

\noindent The parameters listed in the table below involve the tuning of quasi-3D sediment transport, if enabled. The most important setting is the \textit{kmax} in which the user specifies the number of layers used in the quasi 3D sediment model.

\noindent \textit{Table 6.9 Overview of available keyword related to the quasi 3D sediment transport}

\begin{tabular}{|p{0.7in}|p{1.1in}|p{0.6in}|p{0.7in}|p{0.4in}|p{0.7in}|} \hline 
keyword & description & default & range & units & remark \\ \hline 
kmax+ & Number of sigma layers in Quasi-3D model; kmax = 1 is without vertical structure of flow and suspensions & 1 & 1 - 1000 & - &  \\ \hline 
sigfac+ & Dsig scales with log(sigfac) & 1.3 & 0.0 - 10.0 & - &  \\ \hline 
vicmol+ & Molecular viscosity & 1e-06 & 0.0 - 0.001 & - &  \\ \hline 
vonkar+ & Von Karman constant & 0.4 & 0.01 - 1.0 & - &  \\ \hline 
\end{tabular}


\subsubsection{ Morphology}

\noindent The parameters listed in the table below involve the morphological processes. The \textit{dryslp} and \textit{wetslp} keyword define the critical avalanching slope above and below water respectively. If the bed exceeds the relevant critical slope it collapses and slides downward (avalanching). To reduce the impact of these landslides the maximum bed level change due to avalanching is limited by the \textit{dzmax} value. Which of the two slopes is applied to a grid cell is determined by the \textit{hswitch} keyword.

\noindent The keyword \textit{morfac} enables the user to decouple the hydrodynamical and the morphological time. This is suitable for situations where the morphological process is much slower than the hydrodynamic process. The factor defined by the \textit{morfac }keyword is applied to all morphological change. A \textit{morfac=10} therefore results in 10 times more erosion and deposition in a given time step than usual. The simulation time is however then shortened with the same factor to obtain an approximate result more quickly. The user can prevent the simulation time to be adapted to the \textit{morfac} value by setting \textit{morfacopt} to zero. The keywords \textit{morstart} and \textit{morstop} let the user enable the morphological processes in XBeach only for a particular period during the (hydrodynamic) simulation. These options can be useful if a spin-up time is needed for the hydrodynamics.

\noindent The \textit{struct }and \textit{ne\_layer} keywords enable the user to specify non-erodible structures in the model. To switch on non-erodible structures use \textit{struct=1}. The location of the structures is specified in an external file referenced by the \textit{ne\_layer }keyword. The file has the same format as the bathymetry file explained in 4.3 (Grid and bathymetry). The values of the file define the thickness of the erodible layer on top of the non-erodible layer. A \textit{ne\_layer} file with only zeros therefore defines a fully non-erodible bathymetry and a file with only tens means a erodible layer of 10 meters. Only at the grid cells where the value in the \textit{ne\_layer }file is larger than zero erosion can occur. Non-erodible layers are infinitely deep and thus no erosion underneath these layers can occur.

\noindent \textit{Table 6.10 Overview of available keyword related to morphology}

\begin{tabular}{|p{0.7in}|p{1.0in}|p{0.6in}|p{0.8in}|p{0.5in}|p{0.6in}|} \hline 
keyword & description & default & range & units & remark \\ \hline 
dryslp & Critical avalanching slope above water (dz/dx and dz/dy) & 1.0 & 0.1 - 2.0 & - &  \\ \hline 
dzmax+ & Maximum bed level change due to avalanching & 0.05 & 0.0 - 1.0 & m/s/m &  \\ \hline 
hswitch+ & Water depth at which is switched from wetslp to dryslp & 0.1 & 0.01 - 1.0 & m &  \\ \hline 
morfac & Morphological acceleration factor & 1.0 & 0.0 - 1000.0 & - &  \\ \hline 
morfacopt+ & Switch to adjusting output times for morfac & 1 & 0 - 1 & - &  \\ \hline 
morstart & Start time morphology, in morphological time & 120.0 & 0.0 - 10000000.0 & s &  \\ \hline 
morstop & Stop time morphology, in morphological time & 2000.0 & 0.0 - 10000000.0 & s &  \\ \hline 
ne\_layer & Name of file containing depth of hard structure &  &  & $<$file$>$ &  \\ \hline 
struct & Switch for enabling hard structures & 0 & 0 - 1 & - &  \\ \hline 
wetslp & Critical avalanching slope under water (dz/dx and dz/dy) & 0.3 & 0.1 - 1.0 & - &  \\ \hline 
\end{tabular}


\subsubsection{ Bed update}

\noindent The parameters listed in the table below involve the settings for the bed update process especially in the case multiple sediment fractions and bed layers are involved. The \textit{frac\_dz, split} and \textit{merge} keywords determine the fraction of the variable bed layer thickness at which the layer is split or merged respectively with the surrounding bottom layers. The variable layer is chosen using the \textit{nd\_var} keyword.

\noindent Pre-defined bed evolution can be used with the keywords \textit{setbathy} that is used to turn user-imposed bed level change on, \textit{nsetbathy} that determines the number of imposed bed level conditions and \textit{setbathyfile} that references a file that specifies the time series of imposed bed levels. If the \textit{setbathy} option is used, XBeach will automatically interpolate the bed level at each computational time step from the imposed bed level time series. 

\noindent The \textit{setbathyfile }file must contain a time series (of length \textit{nsetbathy}) of the bed level at every grid point in the XBeach model. The format of the \textit{setbathyfile} file is such that each imposed bed level starts with the time at which the bed level should be applied (in seconds relative to the start of the simulation), followed on a new line by the bed level in each grid cell in an identical manner to the normal initial bathymetry file (e.g., one line per transect in the x-direction). The following imposed bed level condition starts on a new line with the time at which the condition is imposed and the bed levels at that time, etc. An example of a \textit{setbathyfile }is given below.

\noindent \textbf{setbathyfile.dep (generalized)}

\noindent $<$t 0$>$

\noindent $<$z 1,1$>$ $<$z 2,1$>$ $<$z 3,1$>$ ... $<$z nx,1$>$ $<$z nx+1,1$>$

\noindent $<$z 1,2$>$ $<$z 2,2$>$ $<$z 3,2$>$ ... $<$z nx,2$>$ $<$z nx+1,2$>$

\noindent $<$z 1,3$>$ $<$z 2,3$>$ $<$z 3,3$>$ ... $<$z nx,3$>$ $<$z nx+1,3$>$

\noindent ...

\noindent $<$z 1,ny$>$ $<$z 2,ny$>$ $<$z 3,ny$>$ ... $<$z nx,ny$>$ $<$z nx+1,ny$>$

\noindent $<$z 1,ny+1$>$ $<$z 2,ny+1$>$ $<$z 3,ny+1$>$ ... $<$z nx,ny+1$>$ $<$z nx+1,ny+1$>$

\noindent $<$t 1$>$

\noindent $<$z 1,1$>$ $<$z 2,1$>$ $<$z 3,1$>$ ... $<$z nx,1$>$ $<$z nx+1,1$>$

\noindent $<$z 1,2$>$ $<$z 2,2$>$ $<$z 3,2$>$ ... $<$z nx,2$>$ $<$z nx+1,2$>$

\noindent $<$z 1,3$>$ $<$z 2,3$>$ $<$z 3,3$>$ ... $<$z nx,3$>$ $<$z nx+1,3$>$

\noindent ...

\noindent $<$z 1,ny$>$ $<$z 2,ny$>$ $<$z 3,ny$>$ ... $<$z nx,ny$>$ $<$z nx+1,ny$>$

\noindent $<$z 1,ny+1$>$ $<$z 2,ny+1$>$ $<$z 3,ny+1$>$ ... $<$z nx,ny+1$>$ $<$z nx+1,ny+1$>$

\noindent ...

\noindent ...

\noindent $<$t nsetbathy$>$

\noindent $<$z 1,1$>$ $<$z 2,1$>$ $<$z 3,1$>$ ... $<$z nx,1$>$ $<$z nx+1,1$>$

\noindent $<$z 1,2$>$ $<$z 2,2$>$ $<$z 3,2$>$ ... $<$z nx,2$>$ $<$z nx+1,2$>$

\noindent $<$z 1,3$>$ $<$z 2,3$>$ $<$z 3,3$>$ ... $<$z nx,3$>$ $<$z nx+1,3$>$

\noindent ...

\noindent $<$z 1,ny$>$ $<$z 2,ny$>$ $<$z 3,ny$>$ ... $<$z nx,ny$>$ $<$z nx+1,ny$>$

\noindent $<$z 1,ny+1$>$ $<$z 2,ny+1$>$ $<$z 3,ny+1$>$ ... $<$z nx,ny+1$>$ $<$z nx+1,ny+1$>$

\noindent \textbf{setbathyfile.dep (short 1D example, nx = 20, ny = 0, nsetbathy = 4)}
\[0\] 
-20 -18 -16 -14 -12 -10 -8 -6 -5 -4 -3 -2 -1 -0.5 0 0.5 1 2 3 4 5
\[1200\] 
-20 -18 -16 -14 -12 -10 -8 -6 -5 -4 -3 -1.8 -0.75 -0.25 0 0.2 0.7 2 3 4 5
\[2400\] 
-20 -18 -16 -14 -12 -10 -8 -6 -5 -4 -2.5 -1.6 -0.5 -0.3 0 0.1 0.6 1.5 3 4 5
\[3600\] 
-20 -18 -16 -14 -12 -10 -8 -6 -5 -4 -2 -1.4 -0.3 0 0.05 0.1 0.6 1 2.2 3.5 5

\noindent Note that if the \textit{setbathy} option is used, the initial bed level is derived from an interpolation of the \textit{setbathyfile }file time series, not from the \textit{depfile }file. It is strongly advised to turn of the computation of morphological updating (keyword: \textit{morphology} = \textit{0}) if the \textit{setbathy} option is used, as the computed morphological change will be overridden by the imposed morphological change.

\noindent \textit{Table 6.11 Overview of available keyword related to the bed update module}

\begin{tabular}{|p{0.8in}|p{1.0in}|p{0.6in}|p{0.7in}|p{0.5in}|p{0.6in}|} \hline 
keyword & description & default & range & units & remark \\ \hline 
frac\_dz* & Relative thickness to split time step for bed updating & 0.7 & 0.5 - 0.98 & - &  \\ \hline 
merge* & Merge threshold for variable sediment layer (ratio to nominal thickness) & 0.01 & 0.005 - 0.1 & - &  \\ \hline 
nd\_var* & Index of layer with variable thickness & 2 & 2 - nd & - &  \\ \hline 
nsetbathy* & Number of prescribed bed updates & 1 & 1 - 1000 & - &  \\ \hline 
setbathyfile* & Name of prescribed bed update file &  &  & $<$file$>$ &  \\ \hline 
split* & Split threshold for variable sediment layer (ratio to nominal thickness) & 1.01 & 1.005 - 1.1 & - &  \\ \hline 
\end{tabular}


\subsubsection{ Groundwater flow}

\noindent The parameters listed in the table below involve the process of groundwater flow. The vertical permeability coefficient in the vertical can be set differently than that of the horizontal using the keywords \textit{kz }and \textit{kz} respectively. The initial bed level of the aquifer is read from an external file referenced by the \textit{aquiferbotfile} keyword and the initial groundwater head can be set to either a uniform value using the \textit{gw0} keyword or to spatially varying values using an external file referenced by the \textit{gw0file }keyword. Both files have the same format as the bathymetry file explained in 4.3 (Grid and bathymetry).

\noindent \textit{Table 6.12 Overview of available keyword related to the groundwater module}

\begin{tabular}{|p{0.9in}|p{0.9in}|p{0.6in}|p{0.8in}|p{0.5in}|p{0.5in}|} \hline 
keyword & description & default & range & units & remark \\ \hline 
aquiferbot+ & Level of uniform aquifer bottom & -10.0 & -100.0 - 100.0 & m &  \\ \hline 
aquiferbotfile+ & Name of the aquifer bottom file &  &  & $<$file$>$ &  \\ \hline 
dwetlayer+ & Thickness of the top soil layer interacting more freely with the surface water & 0.1 & 0.01 - 1.0 & m &  \\ \hline 
gw0+ & Level initial groundwater level & 0.0 & -5.0 - 5.0 & m &  \\ \hline 
gw0file+ & Name of initial groundwater level file &  &  & $<$file$>$ &  \\ \hline 
gwReturb+ & Reynolds number for start of turbulent flow in case of gwscheme = turbulent & 100.0 & 1.0 - 600.0 & - &  \\ \hline 
gwfastsolve+ & Reduce full 2D non-hydrostatic solution to quasi-explicit in longshore direction & 0 & 0 - 1 & - &  \\ \hline 
gwheadmodel+ & Model to use for vertical groundwater head & parabolic & parabolic, exponential &  &  \\ \hline 
gwhorinfil+ & Switch to include horizontal infiltration from surface water to groundwater & 0 & 0 - 1 & - &  \\ \hline 
gwnonh+ & Switch to turn on or off non-hydrostatic pressure for groundwater & 0 & 0 - 1 & - &  \\ \hline 
gwscheme+ & Scheme for momentum equation & laminar & laminar, turbulent &  &  \\ \hline 
kx+ & Darcy-flow permeability coefficient in x-direction & 0.0001 & 1e-05 - 0.1 & ms\^{}-1 &  \\ \hline 
ky+ & Darcy-flow permeability coefficient in y-direction & 0.0001 & 1e-05 - 0.1 & ms\^{}-1 &  \\ \hline 
kz+ & Darcy-flow permeability coefficient in z-direction & 0.0001 & 1e-05 - 0.1 & ms\^{}-1 &  \\ \hline 
\end{tabular}


\subsubsection{ Non-hydrostatic correction}

\noindent The parameters listed in the table below involve the settings for the non-hydrostatic option (keyword: \textit{wavemodel = nonh}). These are all considered advanced options and it is thus recommended not to change these

\noindent \textit{Table 6.13 Overview of available keyword related to the non-hydrostatic module}

\begin{tabular}{|p{0.8in}|p{0.9in}|p{0.8in}|p{0.9in}|p{0.4in}|p{0.4in}|} \hline 
keyword & description & default & range & units & remark \\ \hline 
Topt+ & Absolute period to optimize coefficient & 10.0 & 1.0 - 20.0 & s &  \\ \hline 
breakviscfac+ & Factor to increase viscosity during breaking & 1.5 & 1.0 - 3.0 & - &  \\ \hline 
breakvisclen+ & Ratio between local depth and length scale in extra breaking viscosity & 1.0 & 0.75 - 3.0 & - &  \\ \hline 
dispc+ & Coefficient in front of the vertical pressure gradient & 1.0 & 0.1 - 2.0 & ? &  \\ \hline 
kdmin+ & Minimum value of kd (pi/dx $>$ min(kd)) & 0.0 & 0.0 - 0.05 & - &  \\ \hline 
maxbrsteep+ & Maximum wave steepness criterium & 0.6 & 0.3 - 0.8 & - &  \\ \hline 
nhbreaker+ & Non-hydrostatic breaker model & 2 & 0 - 3 & - &  \\ \hline 
reformsteep+ & Wave steepness criterium to reform after breaking & 0.25 + maxbrsteep & 0.0 - 0.95+ maxbrsteep & - &  \\ \hline 
secbrsteep+ & Secondary maximum wave steepness criterium & 0.5 + maxbrsteep & 0.0 - 0.95+ maxbrsteep & - &  \\ \hline 
solver+ & Solver used to solve the linear system & tridiag & sip, tridiag &  &  \\ \hline 
solver\_acc+ & Accuracy with respect to the right-hand side used in the following termination criterion: \textbar \textbar b-Ax \textbar \textbar  $<$ acc+\textbar \textbar b\textbar \textbar   & 0.005 & 1e-05 - 0.1 & - &  \\ \hline 
solver\_maxit+ & Maximum number of iterations in the linear sip solver & 30 & 1 - 1000 & - &  \\ \hline 
solver\_urelax+ & Under relaxation parameter & 0.92 & 0.5 - 0.99 & - &  \\ \hline 
\end{tabular}


\subsubsection{ Physical constants}

\noindent The parameters listed in the table below involve physical constants used by XBeach. The gravitational acceleration and density of water are universally used coefficient. The \textit{depthscale} is a factor in order to set different cut-off values like \textit{eps} and \textit{hswitch}. A value of the \textit{depthscale} lower than one means the cut-off values will increase. 

\noindent \textit{Table 6.14 Overview of available keyword related to physics}

\begin{tabular}{|p{0.8in}|p{1.1in}|p{0.6in}|p{0.7in}|p{0.4in}|p{0.6in}|} \hline 
keyword & description & default & range & units & remark \\ \hline 
depthscale+ & Depthscale of (lab)test simulated, affects eps, hmin, hswitch and dzmax & 1.0 & 0.1 - 200.0 & - &  \\ \hline 
g & Gravitational acceleration & 9.81 & 9.7 - 9.9 & ms\^{}-2 &  \\ \hline 
rho & Density of water & 1025.0 & 1000.0 - 1040.0 & kgm\^{}-3 &  \\ \hline 
\end{tabular}


\subsubsection{ Coriolis force}

\noindent The parameters listed in the table below involve the settings for incorporating the effect of Coriolis on the shallow water equations. The keywords are universally used coefficients.

\noindent \textit{Table 6.15 Overview of available keyword related to the Coriolis force}

\begin{tabular}{|p{0.6in}|p{1.1in}|p{0.7in}|p{0.6in}|p{0.5in}|p{0.6in}|} \hline 
keyword & description & default & range & units & remark \\ \hline 
lat+ & Latitude at model location  for computing Coriolis & 0.0 & -90.0 - 90.0 & deg &  \\ \hline 
wearth+ & Angular velocity of earth calculated as: 1/rotation\_time (in hours) & 0.0417 & 0.0 - 1.0 & hour\^{}-1 &  \\ \hline 
\end{tabular}


\subsubsection{ MPI}

\noindent The parameters listed in the table below involve the settings for parallelization of XBeach. When running XBeach in parallel mode, the model domain is subdivided in sub models and each sub model is then computed on a separate core. This will increase the computational speed of the model. The sub models only exchange information over their boundaries when necessary. The MPI parameters determine how the model domain is subdivided. The keyword \textit{mpiboundary} can be set to \textit{auto, x, y }or \textit{man}. In \textit{auto} mode the model domain is subdivided such that the internal boundary is smallest. In \textit{x} or \textit{y} mode the model domain is subdivided in sub models extending to either the full alongshore or the full cross-shore extent of the model domain. In \textit{man} mode the model domain is manually subdivided using the values specified with the \textit{mmpi} and \textit{nmpi} keywords. The number of sub models is not determined by XBeach itself, but by the MPI wrapper (e.g. MPICH2 or OpenMPI). It is important to note that information about slopes isn't exchanged over the boundaries. Therefore the avalanching algorithm will not function over MPI boundaries.

\noindent \textit{Table 6.16 Overview of available keyword related to MPI}

\begin{tabular}{|p{0.8in}|p{1.1in}|p{0.6in}|p{0.6in}|p{0.4in}|p{0.6in}|} \hline 
keyword & description & default & range & units & remark \\ \hline 
mmpi+ & Number of domains in cross-shore direction when manually specifying mpi domains & 2 & 1 - 100 & - &  \\ \hline 
mpiboundary+ & Fix mpi boundaries along y-lines, x-lines, use manual defined domains or find shortest boundary automatically & auto & auto, x, y, man &  &  \\ \hline 
nmpi+ & Number of domains in alongshore direction when manually specifying mpi domains & 4 & 1 - 100 & - &  \\ \hline 
\end{tabular}


\subsubsection{ Output projection}

\noindent The parameters listed in the table below involve the projection of the model output. These settings do not influence the model results in anyway. The \textit{rotate} keyword can be used to rotate the model output with an angle specified by the keyword \textit{alfa}. The \textit{projection} string can hold a string specifying the coordinate reference system used and is stored in the netCDF output file as metadata.

\noindent 

\noindent \textit{Table 6.17 Overview of available keyword related to the output projection}

\begin{tabular}{|p{0.8in}|p{1.0in}|p{0.6in}|p{0.7in}|p{0.4in}|p{0.6in}|} \hline 
keyword & description & default & range & units & remark \\ \hline 
projection+ & Projection string & ' ' &  & - &  \\ \hline 
rotate & Rotate output as post processing with given angle & 1 & 0 - 1 & - &  \\ \hline 
 &  &  &  &  &  \\ \hline 
\end{tabular}



\noindent \eject \textbf{}


\subsection{ Numerical implementation }


\subsubsection{ Grid set-up}

\noindent The new implementation utilizes a curvilinear, staggered grid where depths, water levels, wave action and sediment concentrations are given in the cell centers (denoted by subscript z) and velocities and sediment fluxes at the cell interfaces (denoted by subscript u or v). In Figure 6.1 the z, u, v and c (corner) points with the same numbering are shown. The grid directions are named s and n; grid distances are denoted by $\Delta s$and $\Delta n$, with subscripts referring to the point where they are defined. A finite-volume approach is utilized where mass, momentum and wave action are strictly conserved.  In the  middle panel of Figure 6.1, the control volume for the mass balance is shown with the corresponding grid distances around the \textit{u-} and \textit{v-}points. The right panel explains the numbering of the fluxes \textit{Q }and the volume \textit{V}.

\noindent \includegraphics*[width=6.65in, height=1.51in, keepaspectratio=false]{image28}

\noindent \textit{Figure 6.1 Location of staggered grid points (left panel); definition of grid distances (middle) and terms in volume balance (right)}

\noindent \textbf{\textit{}}


\subsubsection{ Wave action balance}


\paragraph{ Surfbeat solver}

\noindent The time-varying wave action balance solved in XBeach is as follows:

\noindent 
\begin{equation} \label{6.1)} 
\frac{\partial E}{\partial t} +\frac{\partial EC_{g,u} }{\partial s} +\frac{\partial EC_{g,v} }{\partial n} +\frac{\partial EC_{\vartheta } }{\partial \vartheta } =-Sink 
\end{equation} 


\noindent Where \textit{E} is the wave energy or wave action, Cg is the group velocity, $C_{\vartheta } $the refraction speed in theta-space and \textit{Sink} refers to effects of wave breaking and bottom friction.

\noindent Again, the advection terms are the only ones affected by the curvilinear scheme so we will discuss their treatment in detail. The control volume is the same as for the mass balance. In equation\eqref{ZEqnNum633932} the procedure to compute the wave energy fluxes across the cell boundaries is outlined. All variables should also have an index \textit{itheta} referring to the directional grid, but for brevity these are omitted here. 

\noindent 

\noindent The component of the group velocity normal to the cell boundary, at the cell boundary, is interpolated from the two adjacent cell center points. Depending on the direction of this component, the wave energy at the cell boundary is computed using linear extrapolation based on the two upwind points, taking into account their grid distances. This second order upwind discretization preserves the propagation of wave groups with little numerical diffusion.

\noindent 
\begin{equation} \label{ZEqnNum633932} 
\begin{array}{l} {Cg_{u,u}^{i,j} =\frac{1}{2} (Cg_{u}^{i,j} +Cg_{u}^{i+1,j} )} \\ {E_{u}^{i,j} =E^{i,j} +\frac{1}{2} \Delta s_{u}^{i,j} \frac{E^{i,j} -E^{i-1,j} }{\Delta s_{u}^{i-1,j} } =} \\ {=\left(\left(\Delta s_{u}^{i-1,j} +\frac{1}{2} \Delta s_{u}^{i,j} \right)E^{i,j} -\frac{1}{2} \Delta s_{u}^{i,j} E^{i-1,j} \right)/\Delta s_{u}^{i-1,j} \, \, \, \, \, ,Cg_{u,u}^{i,j} >0} \\ {E_{u}^{i,j} =E^{i+1,j} -\frac{1}{2} \Delta s_{u}^{i,j} \frac{E^{i+2,j} -E^{i+1,j} }{\Delta s_{u}^{i+1,j} } =} \\ {=\left(\left(\Delta s_{u}^{i+1,j} +\frac{1}{2} \Delta s_{u}^{i,j} \right)E^{i+1,j} -\frac{1}{2} \Delta s_{u}^{i,j} E^{i+2,j} \right)/\Delta s_{u}^{i+1,j} \, \, \, \, \, ,Cg_{u,u}^{i,j} <0} \\ {} \\ {Flux_{u}^{i,j} =Cg_{u,u}^{i,j} E_{u}^{i,j} \Delta n_{u}^{i,j} } \end{array} 
\end{equation} 


\noindent The other three fluxes are computed in a similar way; for brevity we will not present all formulations.

\noindent 

\noindent The time integration is explicit and the same as in the original implementation. The advection in u- and v-direction is computed simply by adding the four fluxes and dividing by the cell area. This procedure guarantees conservation of wave energy.

\noindent 
\[\frac{E^{i,j,n+1} -E^{i,j,n} }{\Delta t} +\frac{Flux_{u}^{i,j,n} -Flux_{u}^{i-1,j,n} +Flux_{v}^{i,j,n} -Flux_{v}^{i,j-1,n} }{A_{z}^{i,j} } +\left(\frac{\partial EC_{\vartheta } }{\partial \vartheta } \right)^{i,j,n} =-Sink^{i,j,n} \] 

    \label{ZEqnNum723168}

\noindent The procedure for the roller energy balance is identical to that for the wave energy balance and will not be repeated here.

\noindent 


\paragraph{ Stationary solver}

\noindent In the stationary solver the wave energy and roller energy balances are solved line by line, from the seaward boundary landward. For each line the automatic timestep is computed and the quasi-time-dependent balance according to equation \eqref{ZEqnNum723168} is solved until convergence or the maximum number of iterations is reached, after which the solver moves to the next line. 

\noindent The iteration is controlled by the parameters \textit{maxiter } and \textit{maxerror.}

\noindent 


\subsubsection{ Shallow water equations}


\paragraph{ Mass balance equation}

\noindent The mass balance reads as follows: 

\noindent 
\begin{equation} \label{6.4)} 
\frac{\partial V}{\partial t} =Q_{u}^{i,j} -Q_{u}^{i-1,j} +Q_{v}^{i,j} -Q_{v}^{i,j-1}  
\end{equation} 


\noindent This is discretized according to:

   
\begin{equation} \label{6.5)} 
\begin{array}{l} {A_{z}^{i,j} \frac{z_{s}^{i,j,n+1} -z_{s}^{i,j,n} }{\Delta t} =u_{u}^{i,j,n+1/2} h_{u}^{i,j,n} \Delta n_{u}^{i,j} -u_{u}^{i-1,j,n+1/2} h_{u}^{i-1,j,n} \Delta n_{u}^{i-1,j} } \\ {\, \, \, \, \, \, \, \, \, \, \, \, \, \, \, \, \, \, \, \, \, \, \, \, \, \, \, \, \, \, \, \, \, \, \, \, \, \, +v_{v}^{i,j,n+1/2} h_{v}^{i,j,n} \Delta s_{v}^{i,j} -v_{v}^{i,j-1,n+1/2} h_{v}^{i,j-1,n} \Delta s_{v}^{i,j-1} } \end{array} 
\end{equation} 


\noindent Here, \textit{A${}_{z}$ }is the area of the cell around  the cell centre, \textit{z${}_{s}$ }is the surface elevation, \textit{u${}_{u}$} is the u-velocity in the u-point, \textit{h${}_{u}$} the water depth in the u-point and \textit{v${}_{v}$} the v-velocity in the v-point. The indices \textit{i,j} refer to the grid number in u resp. v direction; the index \textit{n} refers to the time step.


\paragraph{ Momentum balance equation}

\noindent Second, we will outline the derivation of the u-momentum balance. The control volume is given in Figure 6.2.  It is centeredaround the u-point. We now consider the rate of change of the momentum in the local u-direction as follows:

\noindent 
\begin{equation} \label{6.6)} 
\frac{d\left(Vu\right)}{dt} -\sum Q_{in} u_{in}  +\sum Q_{out} u +Vg\frac{\partial z_{s} }{\partial s} +A\frac{\tau _{b,u} }{\rho } =A\frac{\tau _{s,u} }{\rho } +A\frac{F_{u} }{\rho }  
\end{equation} 


where V is the cell volume, u the velocity in local grid direction, Q the fluxes,$\rho $ the density, g acceleration of gravity, $\tau _{b,u} ,\tau _{b,u} ,F_{u} $ the bed shear stress, wind shear stress and wave force in u-direction. We consider that the outgoing fluxes carry the velocity inside the cell, \textit{u }and that \textit{u${}_{in}$ }is determined at each inflow boundary by interpolation, reconstructing the component in the same direction as \textit{u}.

The volume  balance for the same volume reads:

\noindent 
\begin{equation} \label{6.7)} 
\frac{dV}{dt} -\sum Q_{in} + \sum Q_{out} =0  
\end{equation} 


\noindent By multiplying the volume balance by \textit{u}, subtracting it from the momentum balance and dividing the result by \textit{V } we arrive at the following equation:

\noindent 
\begin{equation} \label{6.8)} 
\frac{du}{dt} +\frac{\sum Q_{in} \left(u-u_{in} \right) }{Ah_{um} } +g\frac{\partial z_{s} }{\partial s} +\frac{\tau _{b,s} }{\rho h_{um} } =\frac{\tau _{s,u} }{\rho h_{um} } +\frac{F_{u} }{\rho h_{um} }  
\end{equation} 


\noindent where \textit{A } is the cell area and \textit{h${}_{um}$} is the average depth of the cell around the \textit{u}-point. The procedure for the second term (the others are straightforward) now boils down to integrating (only) the incoming fluxes over the interfaces and multiplying them with the difference between \textit{u }in the cell and the component of velocity in the same direction at the upwind cell. 

\noindent \includegraphics*[width=2.81in, height=2.27in, keepaspectratio=false, trim=1.10in 0.93in 0.00in 0.31in]{image29}

\noindent \textbf{\textit{Figure }}\textit{6}\textbf{\textit{.}}\textit{2}\textbf{\textit{ Control volume u-momentum balance and definition of fluxes}}

\noindent 

\noindent In equations \eqref{ZEqnNum256620} and \eqref{ZEqnNum823991}  and \eqref{ZEqnNum473828} the procedure for computing the u-momentum balance is outlined. The discharges in the u-points are computed by multiplying the velocity in the u- or v-point  by the water depth at that point. These discharges are then interpolated to the borders of the control volume around the u-point. The difference $\Delta \alpha $in grid orientation between the incoming cell and the u-point is computed and used to compute the component of the incoming velocity in the local u-direction, from the left and right side of the control volume.
\begin{equation} \label{ZEqnNum256620} 
\begin{array}{l} {{\rm q}_{u}^{i,j} =u_{u}^{i,j,n-1/2} h_{u}^{i,j,n} } \\ {{\rm q}_{in}^{left} {\rm \; =\; }\frac{1}{2} \left({\rm q}_{u}^{i,j} _{{\rm \; }} {\rm +q}_{u}^{i-1,j} \right)\, \, \, \, \, \, \, \, \, \, \, \, \, \, \, \, \, \, \, \, \, \, \, \, \, \, \, \, \, \, \, \, \, \, \, \, \, \, \, \, \, \, \, \, \, \, ,\, {\rm q}_{in}^{right} {\rm \; =\; -}\frac{1}{2} \left({\rm q}_{u}^{i+1,j} _{{\rm \; }} {\rm +q}_{u}^{i,j} \right)\, \, } \\ {\Delta \alpha ^{left} =\alpha _{u}^{i,j} _{{\rm \; }} -\alpha _{u}^{i-1,j} \, \, \, \, \, \, \, \, \, \, \, \, \, \, \, \, \, \, \, \, \, \, \, \, \, \, \, \, \, \, \, \, \, \, \, \, \, \, \, \, \, \, \, \, \, \, \, \, ,\, \Delta \alpha ^{right} =\alpha _{u}^{i+1,j} _{{\rm \; }} -\alpha _{u}^{i,j} \, } \\ {u_{in}^{left} =u_{u}^{i-1,j} _{{\rm \; }} \cos \left(\Delta \alpha \right)-v_{u}^{i-1,j} \sin \left(\Delta \alpha \right)\, \, \, \, \, \, \, \, \, \, \, \, \, \, ,\, u_{in}^{right} =u_{u}^{i+1,j} _{{\rm \; }} \cos \left(\Delta \alpha \right)-v_{u}^{i+1,j} \sin \left(\Delta \alpha \right)\, } \end{array} 
\end{equation} 


\noindent The same is done for the top and bottom of the control volume, based on the discharges in v-direction:
\begin{equation} \label{ZEqnNum823991} 
\begin{array}{l} {{\rm q}_{v}^{i,j} =v_{v}^{i,j,n-1/2} h_{v}^{i,j,n} } \\ {{\rm q}_{in}^{bottom} {\rm \; =\; }\frac{1}{2} \left({\rm q}_{v}^{i,j-1} _{{\rm \; }} {\rm +q}_{v}^{i+1,j-1} \right)\, \, \, \, \, \, \, \, \, \, \, \, \, \, \, \, \, \, \, \, \, \, \, \, \, \, \, \, \, \, \, \, \, \, ,\, {\rm q}_{in}^{top} {\rm \; =\; -}\frac{1}{2} \left({\rm q}_{v}^{i,j} _{{\rm \; }} {\rm +q}_{v}^{i+1,j} \right)\, \, } \\ {\Delta \alpha ^{bottom} =\alpha _{u}^{i,j} _{{\rm \; }} -\alpha _{u}^{i,j-1} \, \, \, \, \, \, \, \, \, \, \, \, \, \, \, \, \, \, \, \, \, \, \, \, \, \, \, \, \, \, \, \, \, \, \, \, \, \, \, \, \, \, \, ,\, \Delta \alpha ^{top} =\alpha _{u}^{i,j+1} _{{\rm \; }} -\alpha _{u}^{i,j} \, } \\ {u_{in}^{bottom} =u_{u}^{i,j-1} _{{\rm \; }} \cos \left(\Delta \alpha \right)-v_{u}^{i,j-1} \sin \left(\Delta \alpha \right)\, \, \, \, \, \, \, \, \, ,\, u_{in}^{top} =u_{u}^{i,j+1} _{{\rm \; }} \cos \left(\Delta \alpha \right)-v_{u}^{i,j+1} \sin \left(\Delta \alpha \right)\, } \\ {} \end{array} 
\end{equation} 
Finally, the advective term in the momentum balance is given in equation \eqref{ZEqnNum473828}. 
\begin{equation} \label{ZEqnNum473828} 
\begin{array}{l} {\left(\frac{\sum Q_{in} \left(u-u_{in} \right) }{Ah_{um} } \right)^{i,j} =\, {\rm max(q}_{in}^{left} ,0){\rm (}u_{u}^{i,j} {\rm -}u_{in}^{left} {\rm )}\frac{\Delta n_{z}^{i,j} }{h_{um}^{i,j} A_{u}^{i,j} } \, \, \, } \\ {\, \, \, \, \, \, \, \, \, \, \, \, \, \, \, \, \, \, \, \, \, \, \, \, \, \, \, \, \, \, \, \, \, \, \, \, \, \, \, \, +\, \, {\rm max(q}_{in}^{right} ,0){\rm (}u_{u}^{i,j} {\rm -}u_{in}^{right} {\rm )}\frac{\Delta n_{z}^{i+1,j} }{h_{um}^{i,j} A_{u}^{i,j} } \, } \\ {\, \, \, \, \, \, \, \, \, \, \, \, \, \, \, \, \, \, \, \, \, \, \, \, \, \, \, \, \, \, \, \, \, \, \, \, \, \, \, \, \, +\, {\rm max(q}_{in}^{bottom} ,0){\rm (}u_{u}^{i,j} {\rm -}u_{in}^{bottom} {\rm )}\frac{\Delta s_{c}^{i,j-1} }{h_{um}^{i,j} A_{u}^{i,j} } } \\ {\, \, \, \, \, \, \, \, \, \, \, \, \, \, \, \, \, \, \, \, \, \, \, \, \, \, \, \, \, \, \, \, \, \, \, \, \, \, \, \, +\, \, {\rm max(q}_{in}^{top} ,0){\rm (}u_{u}^{i,j} {\rm -}u_{in}^{top} {\rm )}\frac{\Delta s_{c}^{i,j} }{h_{um}^{i,j} A_{u}^{i,j} } \, } \end{array} 
\end{equation} 



\paragraph{ Time integration scheme}

\noindent The time integration of the mass and momentum balance equations is combined in an explicit leap-frog scheme, as depicted in Figure 6.3. The velocities (in the '-' points) are updated using the momentum balance, the water levels are updated using the mass balance. The water level gradients influence the momentum balance and the velocities and derived discharges affect the mass balance. Because of the leap-frog scheme these influences are always computed at the half time step level, which  makes the scheme second order accurate.

\noindent 

\noindent \includegraphics*[width=2.39in, height=1.42in, keepaspectratio=false, trim=0.00in 1.22in 0.00in 0.00in]{image30}

\noindent \textbf{\textit{Figure }}\textit{6}\textbf{\textit{.}}\textit{3}\textbf{\textit{ Leap-frog time integration scheme}}

\noindent Using this straightforward finite volume approach, complicated transformations of the equations are avoided and the solution scheme remains transparent. It is also completely compatible with the original rectilinear implementation and is even slightly more efficient.

\noindent 


\subsubsection{ Groundwater flow}

\noindent In order to solve the equations in xx, the spatial and temporal domain of the groundwater system is split into the same spatial grid and time steps as the XBeach surface water model it is coupled to. At each time step in the numerical model, the depth average groundwater head is calculated in the center of the groundwater cells, and the fluxes (specific discharge, submarine exchange, infiltration and exfiltration) are calculated on the cell interfaces

\noindent 

\noindent At the start of the time step, every cell is evaluated whether the groundwater and surface water are connected:

\noindent 
\begin{equation} \label{ZEqnNum222755} 
\kappa _{i,j} =\eta _{gw,i,j} \ge z_{b,i,j} -\varepsilon \wedge \eta _{i,j} \ge z_{b,i,j} +\varepsilon  
\end{equation} 


\noindent In \eqref{ZEqnNum222755} \textit{$\varepsilon $} is a numerical smoothing constant used to deal with numerical round off errors near the bed (equal to the parameter \textit{dwetlayer} in the case of hydrostatic groundwater flow, and to the parameter \textit{eps} in the case of non-hydrostatic groundwater flow), and \textit{i} and \textit{j} represent cross-shore and longshore coordinates in the numerical solution grid, respectively. Infiltration is calculated in cells where the groundwater and surface water are not connected and there exists surface water. As shown in \eqref{ZEqnNum418429} the infiltration rate is a function of the thickness of the wetting front, which is zero at the start of infiltration, and increases as a function of the infiltration rate. The equations for the infiltration rate and the thickness of the wetting front are approximated by first-order schemes, in which the wetting front is updated using a backward-Euler scheme, which ensures numerical stability:

\noindent 
\begin{equation} \label{ZEqnNum753693} 
\begin{array}{c} {S_{inf,i,j}^{n} =K_{i,j} \left(\frac{1}{\rho g} \frac{\left. p_{lpf,i,j} \right|^{z=z_{b} } }{\delta _{infill,i,j}^{n} } 1\right)} \\ {{\rm \; }\delta _{infill,i,j}^{n} =\delta _{infill,i,j}^{n-1} +S_{inf,i,j}^{n} \frac{\Delta t}{n_{p} } } \end{array} 
\end{equation} 


\noindent In \eqref{ZEqnNum753693} the pressure term \textit{p${}_{lpf}$}  is the surface water pressure at the bed, which in the case of non-hydrostatic surface water flow is high-pass filtered at 4/\textit{T${}_{rep}$}, the superscript \textit{n} corresponds to the time step number and\textit{ $\Delta $t }is the size of the time step. The infiltration rate in the coupled relationship can be solved through substitution:

\noindent 
\begin{equation} \label{ZEqnNum918911} 
S_{inf,i,j}^{n} =\frac{-\delta _{infill,i,j}^{n-1} +\frac{\Delta t}{n_{p} K_{i,j} } +\sqrt{\left(\delta _{infill,i,j}^{n-1} \right)^{2} +\frac{2}{n_{p} } \frac{\Delta t}{\delta _{infill,i,j}^{n-1} K_{i,j} } +\frac{4\rho g}{n_{p} } \frac{\Delta t}{p_{i,j} \left|^{z=\xi } \right. K_{i,j} } +\frac{\Delta t^{2} }{K_{_{i,j} }^{2} } } }{\frac{2\Delta t}{n_{p} } }  
\end{equation} 
At the end of infiltration, i.e. when the groundwater and surface water become connected or there is no surface water left, the wetting front thickness is reset to zero. If the infiltration rate exceeds the Reynolds number for the start of turbulence, the local hydraulic conductivity is updated using the local Reynolds number:

\noindent 
\begin{equation} \label{ZEqnNum940656} 
K_{i,j} =K_{lam} \sqrt{\frac{Re_{crit} }{\max (Re_{i,j} ,Re_{crit} )} }  
\end{equation} 


\noindent XBeach iterates until a minimum threshold difference between iterations is found for \eqref{ZEqnNum918911} and \eqref{ZEqnNum940656}. Infiltration in one time step is limited to the amount of surface water available in the cell and to the amount of water required to raise the groundwater level to the level of the bed:

\noindent 
\begin{equation} \label{6.11)} 
S_{inf,i,j}^{n} =\min \left(S_{inf,i,j}^{n} ,\frac{\eta _{i,j-} z_{b,i,j} }{\Delta t} ,\frac{z_{b,i,j} -\eta _{gw,i,j-} }{n_{p} \Delta t} \right) 
\end{equation} 


\noindent If during infiltration the groundwater level reaches the bed level, the fraction of the time step required to do so is estimated (x) and the remaining fraction is used in the submarine exchange.

\noindent 
\begin{equation} \label{6.11)} 
\begin{array}{c} {\Lambda _{t,i,j} =\frac{n_{p} \left(z_{b,i,j} -\eta _{gw,i,j} \right)}{K_{i,j} \left(\frac{1}{\rho g} \frac{\left. p_{i,j} \right|^{z=\xi } }{z_{b,i,j} -\eta _{gw,i,j} } +1\right)} \frac{1}{\Delta t} } \\ {} \\ {0\le \Lambda _{t,i,j} \le 1} \end{array} 
\end{equation} 


\noindent Exfiltration is calculated in cells where the groundwater and surface water are not connected and the groundwater level exceeds the bed level:

\noindent 
\begin{equation} \label{6.11)} 
S_{exf,i,j} =n_{p} \frac{z_{b,i,j} -\eta _{gw,i,j} }{\Delta t}  
\end{equation} 


\noindent Horizontal infiltration and exfiltration (keyword: \textit{gwhorinfil = 1}) is computed across the numerical vertical interface between the groundwater domain and the surface water domain in adjoining cells. The direction of exchange is determined by the head gradient, and the bed level slope direction:

\noindent 
\begin{equation} \label{6.11)} 
\begin{array}{l} {S_{i,j,hor}^{u} =\left\{\begin{array}{cc} {K_{_{i,j} }^{u} \left(\frac{\eta _{i,j} +\frac{q_{i,j} }{g} -H_{gw,i+1,j} }{\Delta x} \right)\left(z_{b,i+1,j} -z_{b,i,j} \right)} & {z_{b,i+1,j} >z_{b,i,j} \wedge \eta _{i,j} >eps} \\ {K_{_{i,j} }^{u} \left(\frac{\eta _{i+1,j} +\frac{q_{i+1,j} }{g} -H_{gw,i,j} }{\Delta x} \right)\left(z_{b,i,j} -z_{b,i+1,j} \right)} & {z_{b,i+1,j} <z_{b,i,j} \wedge \eta _{i+1,j} >eps} \end{array}\right. } \\ {S_{i,j,hor}^{v} =\left\{\begin{array}{cc} {K_{_{i,j} }^{v} \left(\frac{\eta _{i,j} +\frac{q_{i,j} }{g} -H_{gw,i,j+1} }{\Delta y} \right)\left(z_{b,i,j+1} -z_{b,i,j} \right)} & {z_{b,i,j+1} >z_{b,i,j} \wedge \eta _{i,j} >eps} \\ {K_{_{i,j} }^{v} \left(\frac{\eta _{i,j+1} +\frac{q_{i,j+1} }{g} -H_{gw,i,j} }{\Delta y} \right)\left(z_{b,i,j} -z_{b,i,j+1} \right)} & {z_{b,i,j+1} <z_{b,i,j} \wedge \eta _{i,j+1} >eps} \end{array}\right. } \end{array} 
\end{equation} 
 

\noindent 

\noindent After infiltration and exfiltration have been calculated, the groundwater level and surface water level are updated:

\noindent 
\begin{equation} \label{6.11)} 
\begin{array}{l} {\eta _{gw,i,j}^{n+\frac{1}{2} } =\eta _{gw,i,j}^{n} +\frac{\Delta t}{n_{p} } \left(S_{\inf } +S_{exf} +S_{hor} \right)} \\ {\eta _{i,j}^{n+\frac{1}{2} } =\eta _{i,j}^{n} +\Delta t\left(-S_{\inf } -S_{exf} -S_{hor} \right)} \end{array} 
\end{equation} 


\noindent All updated cells are subsequently re-evaluated on whether the surface water and groundwater are connected or unconnected

\noindent \eject \textit{}

\noindent The cell height at the center of the groundwater cells (\textit{$\Delta $z${}_{H,i,j}$}) is calculated from the groundwater level and the bottom of the aquifer in the center of the cell, whereas the cell heights at the horizontal cell interfaces are calculated using an upwind procedure:

\noindent 
\begin{equation} \label{ZEqnNum516818} 
\begin{array}{c} {\Delta z_{H,i,j} =\eta _{gw,i,j} -z_{aquifier,i,j} =h_{gw,i,j} } \\ {} \\ {\Delta z_{u,i,j} =\left\{\begin{array}{l} {\Delta z_{H,i+1,j} {\rm \; \; if\; }q_{qw,i,j}^{x} <0} \\ {\Delta z_{H,i,j} {\rm \; \; \; \; if\; }q_{qw,i,j}^{x} \ge 0} \end{array}\right. } \\ {\Delta z_{v,i,j} =\left\{\begin{array}{l} {\Delta z_{H,i+1,j} {\rm \; \; if\; }q_{qw,i,j}^{y} <0} \\ {\Delta z_{H,i,j} {\rm \; \; \; \; if\; }q_{qw,i,j}^{y} \ge 0} \end{array}\right. } \end{array} 
\end{equation} 


\noindent In \eqref{ZEqnNum516818} \textit{z${}_{aquifer}$} is the level of the bottom of the aquifer. As described in Section 3.3.6, the head applied on the top boundary of the groundwater domain (\textit{H${}_{bc}$}) depends on whether the groundwater and surface water are connected or unconnected:

\noindent 
\begin{equation} \label{ZEqnNum702689} 
\begin{array}{l} {H_{bc,i,j} =\left(1-\kappa _{r,i,j} \right)\eta _{gw,i,j} +\kappa _{r,i,j} \left(z_{b,i,j} {\rm +}\frac{\left. p_{i,j} \right|^{z=z_{b} } }{\rho g} \right)} \\ {\begin{array}{ccc} {\kappa _{r,i,j} =1-\frac{z_{b,i,j} -\eta _{gw,i,j} }{\varepsilon } } & {} & {0\le \kappa _{r,i,j} \le 1} \end{array}} \end{array} 
\end{equation} 


\noindent 

\noindent In \eqref{ZEqnNum702689} the parameter \textit{$\kappa $${}_{r}$} is the relative numerical `connectedness' of the groundwater and surface water head, determined by linear interpolation across the numerical smoothing constant \textit{$\varepsilon $} \eqref{ZEqnNum222755}.

\noindent 

\noindent In the case of hydrostatic groundwater flow, the groundwater head in each cell is set equal to the head applied on the top boundary of the groundwater domain (\textit{H${}_{bc}$}) and the horizontal groundwater flux is computed from the groundwater head gradient:

\noindent 
\begin{equation} \label{ZEqnNum473196} 
\begin{array}{l} {q_{gw,i,j}^{x} =-K_{u,i,j} \Delta z_{u,i} \frac{H_{i+1,j} -H_{i,j} }{\Delta x_{u,i,j} } =-K_{u,i,j} \Delta z_{u,i} \frac{H_{bc,i+1,j} -H_{bc,i,j} }{\Delta x_{u,i,j} } } \\ {q_{gw,i,j}^{y} =-K_{v,i,j} \Delta z_{v,i} \frac{H_{i+1,j} -H_{i,j} }{\Delta y_{v,i,j} } =-K_{v,i,j} \Delta z_{v,i} \frac{H_{bc,i+1,j} -H_{bc,i,j} }{\Delta y_{v,i,j} } } \end{array} 
\end{equation} 


\noindent In \eqref{ZEqnNum473196} the superscripts \textit{x} and \textit{y} refer to the components of the variable in the cross-shore and longshore direction, respectively, and the subscripts \textit{u} and \textit{v} refer to variables approximated at the horizontal cell interfaces in the cross-shore and longshore direction, respectively.

\noindent 

\noindent In the case of non-hydrostatic groundwater flow, the horizontal specific discharge on each cell interface can be found through an approximation of the non-hydrostatic groundwater head gradient:

\noindent 
\begin{equation} \label{ZEqnNum628356} 
\begin{array}{l} {q_{gw,i,j}^{x} =-K_{u,i,j} \Delta z_{u,i} \frac{H_{i+1,j} -H_{i,j} }{\Delta x_{u,i,j} } =-K_{u,i,j} \Delta z_{u,i} \frac{H_{bc,i+1,j} -\frac{2}{3} \beta _{i+1,j} \Delta z_{H,i+1,j}^{2} -H_{bc,i,j} +\frac{2}{3} \beta _{i,j} \Delta z_{H,i,j}^{2} }{\Delta x_{u,i,j} } } \\ {q_{gw,i,j}^{y} =-K_{v,i,j} \Delta z_{v,i} \frac{H_{i+1,j} -H_{i,j} }{\Delta y_{v,i,j} } =-K_{v,i,j} \Delta z_{v,i} \frac{H_{bc,i+1,j} -\frac{2}{3} \beta _{i+1,j} \Delta z_{H,i+1,j}^{2} -H_{bc,i,j} +\frac{2}{3} \beta _{i,j} \Delta z_{H,i,j}^{2} }{\Delta y_{v,i,j} } } \end{array} 
\end{equation} 


\noindent In \eqref{ZEqnNum628356} the subscript \textit{H} refers to variables approximated at the cell centers. The hydraulic conductivity may be different at each cell interface and is therefore computed at every interface where every K is calculated separately.

\noindent 

\noindent Continuity in the groundwater cell is found following:

\noindent 
\begin{equation} \label{ZEqnNum299822} 
q_{gw,i-1,j}^{x} -q_{gw,i,j}^{x} +q_{gw,i,j-1}^{y} -q_{gw,i,j}^{y} -q_{gw,i,j}^{z} =0 
\end{equation} 


\noindent In \eqref{ZEqnNum299822} the variable \textit{q${}^{z}$} refers to the vertical groundwater discharge (e.g., submarine exchange if connected to the surface water, or groundwater level fluctuations if the groundwater is not connected to the surface water).

\noindent 

\noindent In the case of hydrostatic groundwater flow, the variable \textit{q${}^{z}$} can be solved through the known variables \textit{q${}^{x}$} and \textit{q${}^{y}$}. However, in the case of non-hydrostatic groundwater flow, all variables in \eqref{ZEqnNum299822} contain an unknown value for the groundwater pressure head, described in terms of a known head at the surface of the groundwater (\textit{H${}_{bc}$}) and the unknown curvature of the vertical groundwater head function (\textit{$\beta $}). Since water is incompressible, the groundwater pressure must be solved for all cells simultaneously using matrix algebra:

\noindent 
\begin{equation} \label{ZEqnNum734155} 
Ax+b=0 
\end{equation} 


\noindent In \eqref{ZEqnNum734155} A is a matrix containing coefficients for the horizontal and vertical specific discharge, x is a vector containing the unknown groundwater head curvature, and b contains the known forcing terms. For a one dimensional cross-shore case, A is reduced to a tridiagonal matrix. The vector of known forcing consists of the numerical gradients in the contribution of the head applied on the top boundary of the groundwater domain to the horizontal specific discharge.

\noindent 

\noindent In the one dimensional case, the solution to the tridiagonal matrix A can be computed using the efficient Thomas algorithm (Thomas 1949). In the two dimensional case, matrix A contains two additional diagonals that are not placed along the main diagonal, and vector b contains additional forcing terms from the alongshore contribution. The solution to the two dimensional case requires a more complex and less computationally efficient matrix solver. In this case the Strongly Implicit Procedure (Stone 1968) is used in a manner similar to Zijlema et al. (2011). The horizontal and vertical groundwater fluxes are calculated using the solution of\textit{ x} plus \eqref{ZEqnNum464450}\eqref{ZEqnNum628356} and \eqref{ZEqnNum299822}\eqref{ZEqnNum294806}. 

\noindent 

\noindent Since in both hydrostatic and non-hydrostatic groundwater flow some local velocities may exceed the critical Reynolds number for the start of turbulence (\textit{Re${}_{crit}$}), the turbulent hydraulic conductivity (\textit{K}) is updated using the local Reynolds number. The solution to \eqref{ZEqnNum734155} and the update of the turbulent hydraulic conductivity are iterated until a minimum threshold difference between iterations is found. Note that this approach is only used is the turbulent groundwater model is selected (keyword: \textit{gwscheme} = \textit{turbulent}).

\noindent 

\noindent The (iterated) solution for the specific vertical discharge is used to update the groundwater level and surface water level:

\noindent 
\begin{equation} \label{6.11)} 
\begin{array}{rcl} {\eta _{_{gw,i,j} }^{n+1} } & {=} & {\left\{\begin{array}{l} {\eta _{_{gw,i,j} }^{n+\frac{1}{2} } {\rm \; \; \; \; \; \; \; \; \; \; \; \; \; \; \; \; \; \; \; \; \; \; \; \; \; }C_{i,j} } \\ {\eta _{_{gw,i,j} }^{n+\frac{1}{2} } +\frac{\Delta t}{n_{p} } \frac{q_{i,j}^{z} }{\Delta x_{H,i,j} } {\rm \; \; \; }\neg C_{i,j} {\rm \; \; \; }} \end{array}\right. } \\ {\eta _{_{i,j} }^{n+1} } & {=} & {\left\{\begin{array}{l} {\eta _{_{i,j} }^{n+\frac{1}{2} } +\Delta t\frac{q_{i,j}^{z} }{\Delta x_{H,i,j} } {\rm \; \; \; \; \; }C_{i,j} } \\ {\eta _{_{i,j} }^{n+\frac{1}{2} } {\rm \; \; \; \; \; \; \; \; \; \; \; \; \; \; \; \; \; \; \; \; \; \; \; }\neg C_{i,j} } \end{array}\right. } \end{array} 
\end{equation} 


\noindent If the groundwater and surface water are connected, and the submarine exchange from the surface water to the groundwater estimated in \eqref{ZEqnNum734155} is greater than the amount of surface water available in the cell, continuity is enforced by lowering the groundwater level to compensate for the lack of permeating water:

\noindent 
\begin{equation} \label{6.11)} 
\eta _{_{gw,i,j} }^{n+1} =\eta _{_{gw,i,j} }^{n+1} +\frac{1}{n_{p} } \left(\eta _{_{i,j} }^{n+\frac{1}{2} } -z_{b,i,j} -\Delta t\frac{q_{i,j}^{z} }{\Delta x_{H,i,j} } \right){\rm \; if\; \; }\kappa _{i,j} \wedge \eta _{i,j}^{n+\frac{1}{2} } -z_{b,i,j} <\Delta t\frac{q_{i,j}^{z} }{\Delta x_{H,i,j} }  
\end{equation} 
\textbf{}


\subsubsection{ Sediment transport}

\noindent The advection-diffusion equation for suspended sediment is the basis for the sediment transport computations in XBeach. The partial differential equation to solve is:

\noindent 
\begin{equation} \label{6.12)} 
\frac{\partial hc}{\partial t} +\nabla \vec{S}_{s} =\frac{hc_{eq} -hc}{T_{s} } =-\frac{hc}{T_{s} } +Ero 
\end{equation} 
Here \textit{c }is the depth-averaged concentration, \textit{c${}_{eq}$} the equilibrium concentration, \textit{Ts} a typical timescale proportional to water depth divided by fall velocity. As is often done to increase robustness, we treat the erosion term explicitly but take an implicit scheme for the sedimentation term:

\noindent 
\begin{equation} \label{6.13)} 
\frac{{\rm c}_{{\rm z}}^{{\rm i,j,n+1}} h_{h}^{i,j,n+1} -h_{h}^{i,j,n} {\rm c}_{z}^{{\rm i,j,n}} }{\Delta {\rm t}} {\rm \; =\; }-\frac{h_{h}^{i,j,n+1} {\rm c}_{z}^{{\rm i,j,n+1}} }{{\rm T}_{{\rm s}}^{{\rm i,j}} } -\left(\nabla \vec{S}_{s} \right)^{i,j,n} {\rm +\; \; }Ero^{i,j,n}  
\end{equation} 
This can be rewritten as:

 
\begin{equation} \label{6.14)} 
{\rm c}_{{\rm c}}^{{\rm i,j,n+1}} {\rm \; =\; }\frac{{\rm T}_{{\rm s}}^{{\rm i,j}} \Delta {\rm t}}{h_{h}^{i,j,n+1} \left({\rm T}_{{\rm s}}^{{\rm i,j}} +\Delta {\rm t}\right)} \left(\frac{h_{h}^{i,j,n} {\rm c}_{{\rm c}}^{{\rm i,j,n}} }{\Delta t} -\left(\nabla \vec{S}_{s} \right)^{i,j,n} {\rm +}Ero^{i,j,n} \right) 
\end{equation} 


\noindent The sediment transport gradient is discretized in a similar way as the mass balance:
\begin{equation} \label{6.15)} 
\left(\nabla \vec{S}_{s} \right)^{i,j} =\frac{\left(S_{u,s}^{i,j} \Delta n_{u}^{i,j} -S_{u,s}^{i-1,j} \Delta n_{u}^{i-1,j} +S_{v,s}^{i,j} \Delta s_{v}^{i,j} -S_{v,s}^{i,j-1} \Delta s_{v}^{i,j-1} \right)}{A_{z}^{i,j} }  
\end{equation} 
The sediment transports in the u- points contain an advective term, a diffusive term and a bed slope term:

\noindent 
\begin{equation} \label{6.16)} 
{\rm S}_{{\rm u,s}} {\rm =c}_{u} {\rm u}_{{\rm rep,s}} {\rm h}_{{\rm u}} {\rm -D}_{{\rm c}} {\rm h}_{{\rm u}} \frac{\partial c}{\partial s} {\rm -f}_{slope} {\rm c}_{{\rm u}} \left|v\right|_{u} {\rm h}_{{\rm u}} \frac{\partial z_{b} }{\partial s} {\rm )} 
\end{equation} 


\noindent Here \textit{u${}_{rep,s}$} is a representative velocity for suspended transport, which contains contributions due to return flow, wave skewness and wave asymmetry; \textit{D${}_{c}$} is a horizontal diffusion coefficient and \textit{f${}_{slope}$} a coefficient. In  discretized form the expression for the suspended transport in the u-point is:

\noindent 
\begin{equation} \label{6.17)} 
{\rm S}_{u,s}^{i,j} {\rm =c}_{u}^{i,j} {\rm u}_{rep,s}^{i,j} _{{\rm rep,s}} {\rm h}_{u}^{i,j} {\rm -D}_{c}^{i,j} {\rm h}_{u}^{i,j} \frac{c_{c}^{i+1,j} -c_{c}^{i,j} }{\Delta s_{u}^{i,j} } {\rm -f}_{slope} {\rm c}_{u}^{i,j} \left|v\right|_{u}^{i,j} {\rm h}_{u}^{i,j} \frac{z_{b}^{i+1,j} -z_{b}^{i,j} }{\Delta s_{u}^{i,j} } {\rm )} 
\end{equation} 


\noindent The concentrations in the u-points are computed with a $\theta $-method, where $\theta =1$ means a fully upwind approximation, and $\theta =0.5$ a central scheme. In practice, we mostly use the upwind approximation for its robustness.

\noindent 
\begin{equation} \label{6.18)} 
\begin{array}{l} {c_{u}^{i,j} =\theta c_{z}^{i,j} +\left(1-\theta \right)c_{z}^{i+1,j} \, \, \, \, \, \, \, ,u_{rep,s}^{i,j} >0} \\ {c_{u}^{i,j} =\left(1-\theta \right)c_{z}^{i,j} +\theta c_{z}^{i+1,j} \, \, \, \, \, \, \, ,u_{rep,s}^{i,j} <0} \end{array} 
\end{equation} 


\noindent The erosion and deposition terms, which may also be used in the bed updating, are finally computed from:

\noindent 
\begin{equation} \label{6.19)} 
\begin{array}{l} {Ero^{i,j,n} {\rm =\; h}_{{\rm h}}^{i,j,n} {\rm c}_{{\rm eq}}^{{\rm i,j,n}} {\rm /T}_{{\rm s}}^{{\rm i,j}} } \\ {Depo^{i,j} ={\rm h}_{{\rm h}}^{i,j,n+1} {\rm c}_{z}^{{\rm i,j,n+1}} {\rm /T}_{{\rm s}}^{{\rm i,j}} } \end{array} 
\end{equation} 


\noindent The evaluation of the bedload transport takes place in the same way as in the previous versions of XBeach, except for the fact that the directions are taken in local grid direction, and will not be repeated here.

\noindent 


\subsubsection{ Bottom updating schemes}

\noindent Two alternative formulations are available for the bed updating: one where the bottom changes are computed based on the gradients of suspended and bed load transport, equation \eqref{ZEqnNum990933}, and one where the changes due to suspended transport are accounted for through the erosion and deposition terms, equation \eqref{ZEqnNum346072}.  

\noindent 
\begin{equation} \label{ZEqnNum990933} 
\left(1-n_{p} \right)\frac{\partial z_{b} }{\partial t} +MF\left(\nabla \overrightarrow{S}_{s} +\nabla \overrightarrow{S}_{b} \right)=0 
\end{equation} 
\begin{equation} \label{ZEqnNum346072} 
\left(1-n_{p} \right)\frac{\partial z_{b} }{\partial t} +MF\left(Ero-Depo+\nabla \overrightarrow{S}_{b} \right)=0 
\end{equation} 


\noindent In both cases \textit{MF }is the morphological factor used to accelerate morphological changes. In the first case, the sediment in the bottom is conserved in all cases, but changes in the amount of sediment in the water are not considered; one can also say that the sediment in suspension is added to the bottom sediment. In the second case, the storage of sediment in the water is accounted for, but will be distorted in cases of high \textit{MF}. Since under most circumstances the real effect of the storage in the water phase is small we prefer the first formulation which guarantees mass conservation in the bottom. Both formulations are calculated using an explicit scheme, see also equations \eqref{ZEqnNum578684} and \eqref{ZEqnNum428827}.

\noindent 
\begin{equation} \label{ZEqnNum578684} 
\Delta z_{b} ^{i,j,n+1} =\Delta t\frac{MF}{\left(1-n_{p} \right)} \left(\left(\nabla \overrightarrow{S}_{s} \right)^{i,j,n} +\left(\nabla \overrightarrow{S}_{b} \right)^{i,j,n} \right) 
\end{equation} 
\begin{equation} \label{ZEqnNum428827} 
\Delta z_{b} ^{i,j,n+1} =\Delta t\frac{MF}{(1-n_{p} )} \left(Ero^{i,j,n} -Depo^{i,j,n} +\left(\nabla \overrightarrow{S}_{b} \right)^{i,j,n} \right) 
\end{equation} 



\paragraph{ Avalanching}

\noindent XBeach implements avalanching as described in section 2.8.2. It first calculates bed level change due to avalanching in the cross-shore dimension. It then calculates the slopes in alongshore direction and bottom change due to avalanching in this direction. To avoid disrupted sediment balance XBeach does not calculate bottom change due to avalanching at the boundary grid cells. Consequently XBeach cannot calculate avalanching at the boundary between two MPI domains.


\paragraph{ Bed composition}

\noindent The bed is discretized into layers with mass \textit{M(i,j)} in which \textit{i }refers to the layer number and\textit{ j} to the sediment class. The mass fraction per sediment class \textit{p}, layer thickness \textit{$\Delta $} and bed level \textit{z${}_{b}$} are defined by:

\noindent 
\begin{equation} \label{6.20)} 
\begin{array}{l} {p(i,j)=\frac{M(i,j)}{\sum _{j=1}^{J}M(i,j) } } \\ {\Delta (i)=\frac{1}{\left(1-n_{p} \right)\rho _{s} } \sum _{j=1}^{J}M(i,j) } \\ {z_{b} =z_{0} +\sum _{i=1}^{I}\Delta (i) } \end{array} 
\end{equation} 
 

\noindent with porosity \textit{n${}_{p}$ }and sediment density \textit{$\rho $. }The level \textit{z${}_{0}$} is the lowest point of the array of bed layers.

\noindent 

\noindent Due to bed load transport, sediment is exchanged between the top layer and the four horizontally neighboring top layers. Exchange with the water column and the top layer is due to erosion rate \textit{E} and deposition rate\textit{ D}. A mixed Eulerian/Lagrangian framework is proposed. Within the set of layers, one layer is defined as the variable layer. This is the only layer that has a variable total mass. All other layers have a constant total mass, which implies for a constant porosity a constant thickness. Above the variable layer, the layers move with the bed level (Lagrangian): upwards in case of aggradation and downwards in case of degradation. This vertical movement gives an advective flux with advection velocity equal to the bed level change \textit{A=dz/dt}. The variable layer is the transition to the lower layers, which are passive. The number of layers below the variable layer has thus no influence on the computation time. Note that diffusive processes within the bed are not considered yet. These could lead to fluxes between the layers below the variable layer.

\noindent 

\noindent The mass balance for the top layer can now be defined by:

\noindent 
\begin{equation} \label{6.21)} 
\begin{array}{l} {\frac{\partial M(1,j)}{\partial t} =dy\left\{p(1,j)S_{b} (j)\right\}_{W} -dy\left\{p(1,j)S_{b} (j)\right\}_{E} } \\ {\quad \quad \quad \quad dx\left\{p(1,j)S_{b} (j)\right\}_{S} -dx\left\{p(1,j)S_{b} (j)\right\}_{N} } \\ {\quad \quad \quad \quad -dxdyp(1,j)E(j)+dxdyD(j)+dxdy\delta Ap_{bot} (1,j)} \\ {} \\ {A=\sum _{j=1}^{J}\left\{\begin{array}{l} {dy\left\{p(1,j)S_{b} (j)\right\}_{W} -dy\left\{p(1,j)S_{b} (j)\right\}_{E} } \\ {\quad \quad dx\left\{p(1,j)S_{b} (j)\right\}_{S} -dx\left\{p(1,j)S_{b} (j)\right\}_{N} } \\ {\quad \quad -dxdyp(1,j)E(j)+dxdyD(j)} \end{array}\right\} } \end{array} 
\end{equation} 
 

\noindent in which \textit{S${}_{b}$} is the bed-load transport (e.g. Meyer-Peter-Muller), based on the sediment properties of the specific class. The dimensions of the grid cell are defined by \textit{dx} and \textit{dy}. The subscripts W, E, S and N refer to West, East, South and North indicating the four vertical faces of the bed cell. The horizontal faces are indicated with bot for the bottom of the cell and ceil for the ceiling of the cell.  As the fraction p is not defined at the faces but in the cell centers, the upstream fraction is required. For the bed load fluxes, the velocity direction used. For the vertical advection term, the upstream value is based on the bed level variation: in case of aggradation the value in the top layer is used and in case of degradation the value of the second layer is used. If the top layer is the variable layer, there is no advective flux: \textit{$\delta $=0} otherwise \textit{$\delta $=1}.

\noindent 

\noindent The mass balance for the layers in between the top layer and the variable layer is:

\noindent 
\begin{equation} \label{6.22)} 
\frac{\partial M(i,j)}{\partial t} =dxdyA\left(p_{bot} (i,j)-p_{ceil} (i,j)\right) 
\end{equation} 
 

\noindent and for the variable layer, it reads:

\noindent 
\begin{equation} \label{6.23)} 
\frac{\partial M(i,j)}{\partial t} =Ap_{ceil} (i,j) 
\end{equation} 
 

\noindent In order to avoid too thin a or too thick a variable layer, the variable layer is merged or split. If the thickness is smaller than the critical value \textit{$\Delta $${}_{merge}$}, the variable layer is merged with the lower layer. To keep the same number of cells, a cell is added at the bottom of the array, implying that \textit{z${}_{0}$ = z${}_{0}$ -- $\Delta $}. Similarly, the variable is split into two layers if the critical value $\Delta $${}_{split}$ is exceeded. Then, the array is shifted upwards: \textit{z${}_{0}$ = z${}_{0}$ + $\Delta $}. 

\noindent 

\noindent As the bed level update is explicit, the timestep is limited. A conservative estimate can be made by assuming that no more mass can be eroded than available in the top layer:

\noindent 
\begin{equation} \label{6.24)} 
\begin{array}{l} {dy\left\{p(1,j)S_{b} (j)\right\}_{W/E} +dx\left\{p(1,j)S_{b} (j)\right\}_{S/N} +dxdyp(1,j)E(j)<\frac{M(1,j)}{dt} } \\ {dt<\frac{dxdy\left(1-n_{p} \right)\rho _{s} \Delta (1)}{dyS_{b,W/E} (1)+dyS_{b,N/S} (1)+dxdyE(1)} } \end{array} 
\end{equation} 
 

\noindent The transport rate depends on the direction of the transport. The transport rates and erosion rates should be based on the formulation for the smallest fraction: \textit{j=1}. Note that the fraction p falls out. This time step restriction is less severe than the one for shallow water flow. Only in case of very top layers and/or the use of a morphological factor, this time step restriction might be relevant.\eject 


\subsubsection{ Boundary conditions}

\noindent At the start of the XBeach simulation, XBeach checks whether non-stationary varying wave boundary conditions are to be used. If this is the case, it next checks whether the wave spectrum of the wave boundary conditions is to change over time, or remain constant. If the wave spectrum is to remain constant, XBeach will only read from one input file to generate wave boundary conditions. If the wave spectrum is to vary in time, XBeach reads from multiple files. 

\noindent Whether or not the wave spectrum of the boundary conditions changes over time, the XBeach module requires a record length during which the current wave spectral parameters are applied. For the duration of the record length, boundary conditions are calculated at every boundary condition file time step. These time steps are not required to be the same as the time steps in the XBeach main program; XBeach will interpolate where necessary. The boundary condition time steps should therefore only be small enough to accurately describe the incoming bound long waves. The statistical data for the generation of the wave boundary conditions is read from user-specified files. The XBeach module tapers the beginning and end of the boundary condition file. This is done to ensure smooth transitions from one boundary condition file to the next.

\noindent The combination of a large record length and a small time step lead to large demands on the system memory. If the memory requirement is too large XBeach will shut down. The user must choose to either enlarge the boundary condition time step, or to reduce the record length. In case of the latter, several boundary condition files can be generated and read sequentially. It is unwise however to reduce the record length too much, as then the transitions between the boundary condition files may affect the model results.

\noindent Every time the XBeach wave boundary condition module is run, it outputs data to the local directory. Metadata about the wave boundary conditions are stored in list files: \textit{ebcflist.bcf} and \textit{qbcflist.bcf}. The main XBeach program uses the list files to know how and when to read and generate boundary condition files. The actual incoming short-wave energy and long-wave mass flux data is stored in other files. These files have \textit{E\_} and \textit{q\_} prefixes. The main XBeach program uses these files for the actual forcing along the offshore edge.

\noindent \eject 


\subsubsection{ Non-hydrostatic}


\paragraph{ Global continuity equation}

\noindent As was outlined in the previous chapter the global continuity equation, which describes the relation between the free surface and the depth averaged discharge, is given by

\noindent 
\begin{equation} \label{ZEqnNum983791} 
\frac{\partial \eta }{\partial t} +\frac{\partial }{\partial x} \left(UH\right)+\frac{\partial }{\partial x} \left(VH\right)=0 
\end{equation} 


\noindent A simple semi-discretisation of \eqref{ZEqnNum983791} using central differences for the space derivative and using the Hansen scheme for the coupling between velocity and free surface results in

\noindent 
\begin{equation} \label{ZEqnNum570839} 
\frac{\eta _{i,j}^{*} -\eta _{i,j}^{n} }{\Delta t} +\frac{{}^{x} q_{i+{\tfrac{1}{2}} ,j}^{*} -{}^{x} q_{i-{\tfrac{1}{2}} ,j}^{*} }{\Delta x} +\frac{{}^{y} q_{i,j+{\tfrac{1}{2}} }^{*} -{}^{y} q_{i,j-{\tfrac{1}{2}} }^{*} }{\Delta y} =0 
\end{equation} 


\noindent With ${}^{x} q_{i+{\tfrac{1}{2}} ,j}^{*} =H_{i+{\tfrac{1}{2}} ,j}^{n} U_{i+{\tfrac{1}{2}} ,j}^{n+{\tfrac{1}{2}} } $,${}^{y} q_{i,j+{\tfrac{1}{2}} }^{*} =H_{i,j+{\tfrac{1}{2}} }^{n} V_{i,j+{\tfrac{1}{2}} }^{n+{\tfrac{1}{2}} } $ and the water depth is defined by a first order accurate upwind interpolation 

\noindent 
\begin{equation} \label{ZEqnNum559032} 
H_{i+{\tfrac{1}{2}} ,j}^{n} =\left\{\begin{array}{c} {\zeta _{i,j}^{n} +d_{i,j} {\rm \; \; \; \; \; \; \; \; \; \; \; \; \; \; \; \; \; \; \; \; \; \; \; \; \; \; \; \; \; \; \; \; \; }if{\rm \; }U_{i+{\tfrac{1}{2}} ,j}^{n+{\tfrac{1}{2}} } >0} \\ {\zeta _{i+1,j}^{n} +d_{i+1,j} {\rm \; \; \; \; \; \; \; \; \; \; \; \; \; \; \; \; \; \; \; \; \; \; \; \; \; \; \; \; \; }if{\rm \; }U_{i+{\tfrac{1}{2}} ,j}^{n+{\tfrac{1}{2}} } <0} \\ {\max \left(\zeta _{i,j}^{n} ,\zeta _{i+1,j}^{n} \right)+\min \left(d_{i} ,d_{i+1,j} \right){\rm \; }if{\rm \; }U_{i+{\tfrac{1}{2}} ,j}^{n+{\tfrac{1}{2}} } =0} \end{array}\right. {\rm \; } 
\end{equation} 


\noindent The resulting scheme is only first order accurate by virtue of the upwind interpolations and mass conservative. When first order computations are considered accurate enough $\eta _{i,j}^{n+1} $ is set to $\eta _{i,j}^{n*} $. For higher order accuracy the first order prediction is corrected using a limited version of the McCormack scheme. The corrector step reads

\noindent 
\begin{equation} \label{ZEqnNum452818} 
\frac{\eta _{i,j}^{n+1} -\eta _{i,j}^{*} }{\Delta t} +\frac{{}^{x} \Delta q_{i+{\tfrac{1}{2}} ,j}^{*} -{}^{x} \Delta q_{i-{\tfrac{1}{2}} ,j}^{*} }{\Delta x} +\frac{{}^{y} \Delta q_{i,j+{\tfrac{1}{2}} }^{*} -{}^{y} \Delta q_{i,j-{\tfrac{1}{2}} }^{*} }{\Delta y} =0 
\end{equation} 


\noindent With ${}^{x} \Delta q_{i+{\tfrac{1}{2}} ,j}^{n*} =U_{i+{\tfrac{1}{2}} ,j}^{n+{\tfrac{1}{2}} } \Delta H_{i+{\tfrac{1}{2}} ,j}^{} $ and $\Delta H_{i+{\tfrac{1}{2}} ,j}^{} $ is given for positive flow as
\begin{equation} \label{ZEqnNum654284} 
\Delta H_{i+{\tfrac{1}{2}} ,j}^{n*} ={\tfrac{1}{2}} \psi \left(^{\zeta } r_{i+{\tfrac{1}{2}} } \right)\left(\zeta _{i+1,j}^{n*} -\zeta _{i,j}^{n} \right){\rm \; \; \; \; \; \; \; \; \; \; }^{\zeta } {\rm r}_{{\rm i+}{\tfrac{{\rm 1}}{{\rm 2}}} } =\frac{\zeta _{i,j}^{n*} -\zeta _{i-1,j}^{n} }{\zeta _{i+1,j}^{n*} -\zeta _{i,j}^{n} } {\rm \; \; \; \; \; \; \; \; \; \; }\psi \left(r\right)=\max \left(0,\min \left(r,1\right)\right) 
\end{equation} 


\noindent Here $\psi \left(r\right)$ denotes the minmod limiter. Similar expression can be constructed for negative flow. The expression for ${}^{y} \Delta q_{i,j+{\tfrac{1}{2}} }^{n*} $ and $\Delta H_{i,j+{\tfrac{1}{2}} }^{} $ are obtained in a similar manner. Note that the total flux ${}^{x} q_{i+{\tfrac{1}{2}} ,j}^{n+{\tfrac{1}{2}} } $at the cell boundaries thus reads

\noindent 
\begin{equation} \label{ZEqnNum673353} 
{}^{x} q_{i+{\tfrac{1}{2}} ,j}^{n+{\tfrac{1}{2}} } ={}^{x} q_{i+{\tfrac{1}{2}} ,j}^{*} +{}^{x} \Delta q_{i+{\tfrac{1}{2}} ,j}^{*} {\rm ,\; \; \; \; }{}^{y} q_{i,j+{\tfrac{1}{2}} }^{n+{\tfrac{1}{2}} } ={}^{y} q_{i,j+{\tfrac{1}{2}} }^{*} +{}^{y} \Delta q_{i,j+{\tfrac{1}{2}} }^{*} {\rm \; } 
\end{equation} 


\noindent The predictor-corrector set is second order accurate in regions where the solution is smooth, and reduces locally to first order accuracy near discontinuities. Furthermore, the method remains mass conservative. Note that other flux limiters can be used instead of the minmod limiter. However, as the minmod limiter performed adequately, this has not been investigated. ( For an overview of flux limiters see Hirsch, 2007)


\paragraph{ Local continuity equation}

\noindent The depth averaged local continuity equation is given by

\noindent 
\begin{equation} \label{ZEqnNum686131} 
\frac{\partial HU}{\partial x} +\frac{\partial HV}{\partial y} +\left. w\right|_{z=\eta } -\left. u\frac{\partial z}{\partial x} \right|_{z=-\zeta } -\left. v\frac{\partial z}{\partial y} \right|_{z=-\zeta } =0 
\end{equation} 


\noindent This equation is discretized using central differences

\noindent 
\begin{equation} \label{ZEqnNum863615} 
\frac{H_{i+{\tfrac{1}{2}} ,j}^{n+1} U_{i+{\tfrac{1}{2}} ,j}^{n+1{\tfrac{1}{2}} } -H_{i-{\tfrac{1}{2}} ,j}^{n+1} U_{i-{\tfrac{1}{2}} ,j}^{n+1{\tfrac{1}{2}} } }{\Delta x} +\frac{H_{i+{\tfrac{1}{2}} ,j}^{n+1} V_{i,j+{\tfrac{1}{2}} }^{n+1{\tfrac{1}{2}} } -H_{i,j-{\tfrac{1}{2}} }^{n+1} V_{i,j-{\tfrac{1}{2}} }^{n+1{\tfrac{1}{2}} } }{\Delta y} +w_{i,j,s}^{n+1{\tfrac{1}{2}} } -\bar{U}_{i,j}^{n+1{\tfrac{1}{2}} } \frac{\eta _{i+{\tfrac{1}{2}} ,j}^{n+1} -\eta _{i-{\tfrac{1}{2}} ,j}^{n+1} }{\Delta x} -\bar{V}_{i,j}^{n+1{\tfrac{1}{2}} } \frac{\eta _{i,j+{\tfrac{1}{2}} }^{n+1} -\eta _{i,j-{\tfrac{1}{2}} }^{n+1} }{\Delta x} =0 
\end{equation} 


\noindent Missing grid variables$\eta _{i+{\tfrac{1}{2}} ,j}^{n+1} ,\eta _{i,j+{\tfrac{1}{2}} }^{n+1} $ are approximated with upwind interpolation. Because there is no separate time evolution equation for the pressure the local continuity equation will be used to setup a discrete set of poison type equations in which the pressures are the only unknown quantities.


\paragraph{ Horizontal Momentum}

\noindent To obtain a conservative discretisation of the momentum equation the approach from Stelling and Duinmeijer (2003) is followed. However, to improve the accuracy of the method the combined space-time discretisation of the advection is done using a variant of the MacCormack (1969) is used. This scheme consists of a first order predictor step and a flux limited corrector step. The hydrostatic pressure is integrated using the midpoint rule and central differences, while the source terms and the turbulent stresses are integrated using an explicit Euler time integration. Formally the time integration is therefore first order accurate, but in regions where the turbulent stresses are negligible the scheme is of almost second order accuracy.

\noindent 

\noindent The depth averaged horizontal momentum equation for $HU$is given by

\noindent 
\begin{equation} \label{ZEqnNum973790} 
\frac{\partial }{\partial t} \left(HU\right)+\frac{\partial }{\partial x} \left(HU^{2} +{\tfrac{1}{2}} gH^{2} +H\bar{p}-\tau _{xx} \right)+\frac{\partial }{\partial y} \left(HUV-\tau _{yx} \right)=gH\frac{\partial d}{\partial x} -p\frac{\partial d}{\partial x} +S_{x}  
\end{equation} 


\noindent A first order accurate predictor step in time and space is then given as

\noindent 
\begin{equation} \label{ZEqnNum947132} 
\begin{array}{l} {\frac{\left(HU\right)_{i+{\tfrac{1}{2}} ,j}^{*} -\left(HU\right)_{i+{\tfrac{1}{2}} ,j}^{n+{\tfrac{1}{2}} } }{\Delta t} +\frac{{}^{x} \bar{q}_{i+1,j}^{n+{\tfrac{1}{2}} } U_{i+1,j}^{n+{\tfrac{1}{2}} } -{}^{x} \bar{q}_{i,j}^{n+{\tfrac{1}{2}} } U_{i,j}^{n+{\tfrac{1}{2}} } }{\Delta x} +\frac{{}^{y} \bar{q}_{i+{\tfrac{1}{2}} ,j+{\tfrac{1}{2}} }^{n+{\tfrac{1}{2}} } U_{i+{\tfrac{1}{2}} ,j+{\tfrac{1}{2}} }^{n+{\tfrac{1}{2}} } -{}^{y} \bar{q}_{i+{\tfrac{1}{2}} ,j-{\tfrac{1}{2}} }^{n+{\tfrac{1}{2}} } U_{i+{\tfrac{1}{2}} ,j-{\tfrac{1}{2}} }^{n+{\tfrac{1}{2}} } }{\Delta y} } \\ {+g\frac{\left(H^{2} \right)_{i+1,j}^{n+1} -\left(H^{2} \right)_{i,j}^{n+1} }{2\Delta x} =g\bar{H}_{i+{\tfrac{1}{2}} ,j}^{n} \frac{d_{i+{\tfrac{1}{2}} ,j} -d_{i-{\tfrac{1}{2}} ,j} }{\Delta x} +{}^{x} {\rm Pr}_{i+{\tfrac{1}{2}} ,j}^{n+{\tfrac{1}{2}} } +{}^{x} {\rm S}_{i+{\tfrac{1}{2}} ,j}^{n+{\tfrac{1}{2}} } +{}^{x} {\rm T}_{i+{\tfrac{1}{2}} ,j}^{n+{\tfrac{1}{2}} } } \end{array} 
\end{equation} 


\noindent Here Pr represents a discretisation of the dynamic pressure; T the effect of (turbulent) viscosity and S includes all other source terms. The discretisation of the (turbulent) viscous terms is given by central differences:
\begin{equation} \label{6.35)} 
\begin{array}{rcl} {{}^{x} T_{i+{\tfrac{1}{2}} ,j}^{n+{\tfrac{1}{2}} } } & {=} & {\frac{2}{\Delta x_{i+{\tfrac{1}{2}} } } \left[\nu _{i+1,j}^{n} H_{i+1,j}^{n+1} \frac{U_{i+1{\tfrac{1}{2}} ,j}^{n+{\tfrac{1}{2}} } -U_{i+{\tfrac{1}{2}} ,j}^{n+{\tfrac{1}{2}} } }{\Delta x_{i+1} } -\nu _{i,j}^{n} H_{i,j}^{n+1} \frac{U_{i+{\tfrac{1}{2}} ,j}^{n+{\tfrac{1}{2}} } -U_{i-{\tfrac{1}{2}} ,j}^{n+{\tfrac{1}{2}} } }{\Delta x_{i} } \right]} \\ {} & {} & {+\frac{1}{\Delta y_{i} } \left[\bar{\bar{\nu }}_{i+{\tfrac{1}{2}} ,j+{\tfrac{1}{2}} }^{n+{\tfrac{1}{2}} } \bar{\bar{H}}_{i+{\tfrac{1}{2}} ,j+{\tfrac{1}{2}} }^{n+{\tfrac{1}{2}} } \frac{U_{i+{\tfrac{1}{2}} ,j+1}^{n+1} -U_{i+{\tfrac{1}{2}} ,j}^{n+{\tfrac{1}{2}} } }{\Delta y_{i+{\tfrac{1}{2}} } } -\bar{\bar{\nu }}_{i+{\tfrac{1}{2}} ,j-{\tfrac{1}{2}} }^{n+{\tfrac{1}{2}} } \bar{\bar{H}}_{i+{\tfrac{1}{2}} ,j-{\tfrac{1}{2}} }^{n+1} \frac{U_{i+{\tfrac{1}{2}} ,j}^{n+{\tfrac{1}{2}} } -U_{i+{\tfrac{1}{2}} ,j-1}^{n+{\tfrac{1}{2}} } }{\Delta y_{i-{\tfrac{1}{2}} } } \right]} \\ {} & {} & {+\frac{1}{\Delta y_{i} } \left[\bar{\bar{\nu }}_{i+{\tfrac{1}{2}} ,j+{\tfrac{1}{2}} }^{n+{\tfrac{1}{2}} } \bar{\bar{H}}_{i+{\tfrac{1}{2}} ,j+{\tfrac{1}{2}} }^{n+{\tfrac{1}{2}} } \frac{V_{i+1,j+{\tfrac{1}{2}} }^{n+1} -V_{i,j+{\tfrac{1}{2}} }^{n+{\tfrac{1}{2}} } }{\Delta x_{i+{\tfrac{1}{2}} } } -\bar{\bar{\nu }}_{i+{\tfrac{1}{2}} ,j-{\tfrac{1}{2}} }^{n+{\tfrac{1}{2}} } \bar{\bar{H}}_{i+{\tfrac{1}{2}} ,j-{\tfrac{1}{2}} }^{n+1} \frac{V_{i+1,j-{\tfrac{1}{2}} }^{n+{\tfrac{1}{2}} } -V_{i,j-{\tfrac{1}{2}} }^{n+{\tfrac{1}{2}} } }{\Delta x_{i+{\tfrac{1}{2}} } } \right]} \end{array} 
\end{equation} 


\noindent Here $\bar{\bar{\nu }}_{i+{\tfrac{1}{2}} ,j+1}^{n+{\tfrac{1}{2}} } $ and $\bar{\bar{H}}_{i+{\tfrac{1}{2}} ,j+1{\tfrac{1}{2}} }^{n+{\tfrac{1}{2}} } $ are obtained from the surrounding points by simple linear interpolation.

\noindent 

\noindent Due to the incompressible flow assumption the dynamic pressure does not have a separate time evolution equation, but instead it satisfies an elliptical equation in space. As such its effect cannot be calculated explicitly using values at the previous time level. However to improve the accuracy of the predictor step the effect of the dynamic pressure is included explicitly. To do this first the unknown pressure is decomposed as:

\noindent 
\begin{equation} \label{ZEqnNum786367} 
p_{i,j}^{n+1{\tfrac{1}{2}} } =p_{i,j}^{n+{\tfrac{1}{2}} } +\Delta p_{i,j}^{n+1{\tfrac{1}{2}} }  
\end{equation} 


\noindent where the difference in pressure$\Delta p_{i,j}^{n+1{\tfrac{1}{2}} } $ is generally small. In the predictor step the effect of the pressure is included explicitly using$p_{i,j}^{n+{\tfrac{1}{2}} } $. In the corrector step the full Poisson equation is then solved for $\Delta p_{i,j}^{n+1{\tfrac{1}{2}} } $. The pressure term in the predictor step is thus given as 

\noindent 
\begin{equation} \label{6.37)} 
{}^{{\rm x}} {\rm Pr}_{i+{\tfrac{1}{2}} ,j}^{n+{\tfrac{1}{2}} } {\rm =}\frac{H_{i+1,j}^{n+1} \bar{p}_{i+1,j}^{n+{\tfrac{1}{2}} } -H_{i,j}^{n+1} \bar{p}_{i,j}^{n+{\tfrac{1}{2}} } }{\Delta x} -p_{i+{\tfrac{1}{2}} ,j}^{n+{\tfrac{1}{2}} } \frac{d_{i+{\tfrac{1}{2}} ,j} -d_{i-{\tfrac{1}{2}} ,j} }{\Delta x} =\frac{\left(\eta _{i+1,j}^{n+1} +d_{i,j}^{n+1} \right)p_{i+1,j}^{n+{\tfrac{1}{2}} } -\left(\eta _{i,j}^{n+1} -d_{i+1,j}^{n+1} \right)p_{i,j}^{n+{\tfrac{1}{2}} } }{2\Delta x}  
\end{equation} 


\noindent Here $\bar{p}_{i+1,j}^{n+{\tfrac{1}{2}} } $represents the average pressure over the vertical which is approximated with$\bar{p}_{i+1,j}^{n+{\tfrac{1}{2}} } ={\tfrac{1}{2}} p_{i+1,j}^{n+{\tfrac{1}{2}} } $, in which $p_{i+1,j}^{n+{\tfrac{1}{2}} } $ is the pressure at the bottom. Furthermore $p_{i+{\tfrac{1}{2}} ,j}^{n+{\tfrac{1}{2}} } $ is given as$p_{i+{\tfrac{1}{2}} ,j}^{n+{\tfrac{1}{2}} } ={\tfrac{1}{2}} \left(p_{i+1,j}^{n+{\tfrac{1}{2}} } +p_{i,j}^{n+{\tfrac{1}{2}} } \right)$.

\noindent 

\noindent Currently \eqref{ZEqnNum947132} is formulated with the depth integrated momentum as the primitive variable, and not the depth averaged velocity. To reformulate \eqref{ZEqnNum947132} in terms of $U$we use the method by Stelling and Duinmeijer (2003). First note that $\left(HU\right)_{i+{\tfrac{1}{2}} ,j}^{n+{\tfrac{1}{2}} } $ and $\left(HU\right)_{i+{\tfrac{1}{2}} ,j}^{*} $ are approximated as $\bar{H}_{i+{\tfrac{1}{2}} ,j}^{n} U_{i+{\tfrac{1}{2}} ,j}^{n+{\tfrac{1}{2}} } $ and $\bar{H}_{i+{\tfrac{1}{2}} ,j}^{n+1} U_{i+{\tfrac{1}{2}} ,j}^{*} $. Now using \eqref{ZEqnNum570839} $\left(HU\right)_{i+{\tfrac{1}{2}} ,j}^{n+{\tfrac{1}{2}} } $ is equivalent to:

\noindent 
\begin{equation} \label{ZEqnNum191993} 
\left(HU\right)_{i+{\tfrac{1}{2}} ,j}^{n+{\tfrac{1}{2}} } =\bar{H}_{i+{\tfrac{1}{2}} ,j}^{n+1} U_{i+{\tfrac{1}{2}} ,j}^{n+{\tfrac{1}{2}} } -U_{i+{\tfrac{1}{2}} ,j}^{n+{\tfrac{1}{2}} } \Delta t\frac{{}^{x} \bar{q}_{i+1,j}^{n+{\tfrac{1}{2}} } -{}^{x} \bar{q}_{i,j}^{n+{\tfrac{1}{2}} } }{\Delta x} -U_{i+{\tfrac{1}{2}} ,j}^{n+{\tfrac{1}{2}} } \Delta t\frac{{}^{y} \bar{q}_{i+{\tfrac{1}{2}} ,j+{\tfrac{1}{2}} }^{n+{\tfrac{1}{2}} } -{}^{y} \bar{q}_{i+{\tfrac{1}{2}} ,j-{\tfrac{1}{2}} }^{n+{\tfrac{1}{2}} } }{\Delta y}  
\end{equation} 

\[\bar{H}_{i+{\tfrac{1}{2}} ,j}^{n+1} ={\tfrac{1}{2}} \left(H_{i+1,j}^{n+1} +H_{i,j}^{n+1} \right){\rm ,\; \; }{}^{x} \bar{q}_{i,j}^{n+{\tfrac{1}{2}} } ={\tfrac{1}{2}} \left({}^{x} q_{i+{\tfrac{1}{2}} ,j}^{n+{\tfrac{1}{2}} } +{}^{x} q_{i-{\tfrac{1}{2}} ,j}^{n+{\tfrac{1}{2}} } \right){\rm ,\; \; \; }{}^{y} \bar{q}_{i+{\tfrac{1}{2}} ,j+{\tfrac{1}{2}} }^{n+{\tfrac{1}{2}} } ={\tfrac{1}{2}} \left({}^{y} q_{i+{\tfrac{1}{2}} ,j+{\tfrac{1}{2}} }^{n+{\tfrac{1}{2}} } +{}^{y} q_{i+{\tfrac{1}{2}} ,j-{\tfrac{1}{2}} }^{n+{\tfrac{1}{2}} } \right)\] 


\noindent Substituting \eqref{ZEqnNum191993} into \eqref{ZEqnNum947132} the full expressions (including those for $V_{i,j+{\tfrac{1}{2}} }^{*} $) become:
\begin{equation} \label{ZEqnNum184179} 
\begin{array}{r} {\frac{U_{i+{\tfrac{1}{2}} ,j}^{*} -U_{i+{\tfrac{1}{2}} ,j}^{n+{\tfrac{1}{2}} } }{\Delta t} +\frac{{}^{x} \bar{q}_{i+1,j}^{n+{\tfrac{1}{2}} } U_{i+1,j}^{n+{\tfrac{1}{2}} } -{}^{x} \bar{q}_{i,j}^{n+{\tfrac{1}{2}} } U_{i,j}^{n+{\tfrac{1}{2}} } }{\bar{H}_{i+{\tfrac{1}{2}} ,j}^{n+1} \Delta x} +\frac{{}^{y} \bar{q}_{i,j+1}^{n+{\tfrac{1}{2}} } U_{i,j+1}^{n+{\tfrac{1}{2}} } -{}^{y} \bar{q}_{i,j}^{n+{\tfrac{1}{2}} } U_{i,j}^{n+{\tfrac{1}{2}} } }{\bar{H}_{i+{\tfrac{1}{2}} ,j}^{n+1} \Delta y} -\frac{U_{i+{\tfrac{1}{2}} ,j}^{n+{\tfrac{1}{2}} } }{\bar{H}_{i+{\tfrac{1}{2}} ,j}^{n+1} } \frac{{}^{x} \bar{q}_{i+1,j}^{n+{\tfrac{1}{2}} } -{}^{x} \bar{q}_{i,j}^{n+{\tfrac{1}{2}} } }{\Delta x} } \\ {-\frac{U_{i+{\tfrac{1}{2}} ,j}^{n+{\tfrac{1}{2}} } }{\bar{H}_{i+{\tfrac{1}{2}} ,j}^{n+1} } \frac{{}^{y} \bar{q}_{i,j+1}^{n+{\tfrac{1}{2}} } -{}^{y} \bar{q}_{i,j}^{n+{\tfrac{1}{2}} } }{\Delta y} +g\frac{\eta _{i+1,j}^{n+1} -\eta _{i,j}^{n+1} }{\Delta x} =\frac{^{{\rm x}} {\rm Pr}_{i+{\tfrac{1}{2}} ,j}^{n+{\tfrac{1}{2}} } +{}^{x} {\rm S}_{i+{\tfrac{1}{2}} ,j}^{n+{\tfrac{1}{2}} } +{}^{x} {\rm T}_{i+{\tfrac{1}{2}} ,j}^{n+{\tfrac{1}{2}} } }{\bar{H}_{i+{\tfrac{1}{2}} ,j}^{n+1} } } \\ {\frac{V_{i,j+{\tfrac{1}{2}} }^{*} -V_{i,j+{\tfrac{1}{2}} }^{n+{\tfrac{1}{2}} } }{\Delta t} +\frac{{}^{y} \bar{q}_{i,j+1}^{n+{\tfrac{1}{2}} } V_{i,j+1}^{n+{\tfrac{1}{2}} } -{}^{y} \bar{q}_{i,j}^{n+{\tfrac{1}{2}} } V_{i,j}^{n+{\tfrac{1}{2}} } }{\bar{H}_{i,j+{\tfrac{1}{2}} }^{n+1} \Delta y} +\frac{{}^{x} \bar{q}_{i+1,j}^{n+{\tfrac{1}{2}} } V_{i+1,j}^{n+{\tfrac{1}{2}} } -{}^{x} \bar{q}_{i,j}^{n+{\tfrac{1}{2}} } V_{i,j}^{n+{\tfrac{1}{2}} } }{\bar{H}_{i,j+{\tfrac{1}{2}} }^{n+1} \Delta x} -\frac{V_{i+{\tfrac{1}{2}} ,j}^{n+{\tfrac{1}{2}} } }{\bar{H}_{i,j+{\tfrac{1}{2}} }^{n+1} } \frac{{}^{y} \bar{q}_{i,j+1}^{n+{\tfrac{1}{2}} } -{}^{y} \bar{q}_{i,j}^{n+{\tfrac{1}{2}} } }{\Delta y} } \\ {\frac{V_{i,j+{\tfrac{1}{2}} }^{n+{\tfrac{1}{2}} } }{\bar{H}_{i,j+{\tfrac{1}{2}} }^{n+1} } \frac{{}^{x} \bar{q}_{i+1,j}^{n+{\tfrac{1}{2}} } -{}^{x} \bar{q}_{i,j}^{n+{\tfrac{1}{2}} } }{\Delta x} +g\frac{\eta _{i,j+1}^{n+1} -\eta _{i,j}^{n+1} }{\Delta x} =\frac{{}^{y} {\rm Pr}_{i,j+{\tfrac{1}{2}} }^{n+{\tfrac{1}{2}} } +{}^{y} {\rm S}_{i,j+{\tfrac{1}{2}} }^{n+{\tfrac{1}{2}} } +{}^{y} {\rm T}_{i,j+{\tfrac{1}{2}} }^{n+{\tfrac{1}{2}} } }{\bar{H}_{i,j+{\tfrac{1}{2}} }^{n+1} } } \end{array} 
\end{equation} 
Where we again use a first order upwind interpolation for $U_{i+1,j}^{n+{\tfrac{1}{2}} } $and$U_{i,j+1}^{n+{\tfrac{1}{2}} } $. This is exactly the approximation used by Stelling and Duinmeijer (2003) and is fully momentum conservative. 

\noindent The predictor step \eqref{ZEqnNum184179} is first order accurate in both space and time due to the use of upwind approximations for and Euler explicit time integration for the advective terms, and first order time integration for the source/viscous terms. This level of accuracy is acceptable near shore, where strong non-linearity (wave breaking, flooding and drying) will force the use of small steps in space and time anyway. However, in the region where waves only slowly change (e.g. shoaling/refraction on mild slopes), the first order approximations suffer from significant numerical damping. To improve the accuracy of the numerical model in these regions a corrector step is implemented after the predictor step. 

\noindent 

\noindent The corrector step is given by:

\noindent 
\begin{equation} \label{6.40)} 
\begin{array}{l} {\frac{\left(HU\right)_{i+{\tfrac{1}{2}} ,j}^{n+1{\tfrac{1}{2}} } -\left(HU\right)_{i+{\tfrac{1}{2}} ,j}^{n*} }{\Delta t} +\frac{{}^{x} \bar{q}_{i+1,j}^{n+{\tfrac{1}{2}} } \Delta U_{i+1,j}^{} -{}^{x} \bar{q}_{i,j}^{n+{\tfrac{1}{2}} } \Delta U_{i,j}^{} }{\Delta x} +\frac{{}^{y} \bar{q}_{i,j+1}^{n+{\tfrac{1}{2}} } \Delta U_{i,j+1}^{} -{}^{y} \bar{q}_{i,j}^{n+{\tfrac{1}{2}} } \Delta U_{i,j}^{} }{\Delta y} +} \\ {\frac{\left(\eta _{i+1,j}^{n+1} +d_{i,j}^{n+1} \right)\Delta p_{i+1,j}^{n+1{\tfrac{1}{2}} } -\left(\eta _{i,j}^{n+1} -d_{i+1,j}^{n+1} \right)\Delta p_{i,j}^{n+1{\tfrac{1}{2}} } }{2\Delta x} =0} \end{array} 
\end{equation} 


\noindent Or, when formulated in terms of the depth averaged velocity

\noindent 
\begin{equation} \label{ZEqnNum382659} 
\begin{array}{r} {\frac{U_{i+{\tfrac{1}{2}} ,j}^{n+1{\tfrac{1}{2}} } -U_{i+{\tfrac{1}{2}} ,j}^{n*} }{\Delta t} +\frac{{}^{x} \bar{q}_{i+1,j}^{n+{\tfrac{1}{2}} } \Delta U_{i+1,j}^{} -{}^{x} \bar{q}_{i,j}^{n+{\tfrac{1}{2}} } \Delta U_{i,j}^{n*} }{\bar{H}_{i+{\tfrac{1}{2}} ,j}^{n+1} \Delta x} +\frac{{}^{y} \bar{q}_{i+{\tfrac{1}{2}} ,j+{\tfrac{1}{2}} }^{n+{\tfrac{1}{2}} } \Delta U_{i+{\tfrac{1}{2}} ,j+{\tfrac{1}{2}} }^{} -{}^{y} \bar{q}_{i+{\tfrac{1}{2}} ,j-{\tfrac{1}{2}} }^{n+{\tfrac{1}{2}} } \Delta U_{i+{\tfrac{1}{2}} ,j-{\tfrac{1}{2}} }^{} }{\bar{H}_{i+{\tfrac{1}{2}} ,j}^{n+1} \Delta y} +...} \\ {...+\frac{\left(\eta _{i+1,j}^{n+1} +d_{i,j}^{n+1} \right)\Delta p_{i+1,j}^{n+1{\tfrac{1}{2}} } -\left(\eta _{i,j}^{n+1} -d_{i+1,j}^{n+1} \right)\Delta p_{i,j}^{n+1{\tfrac{1}{2}} } }{2\bar{H}_{i+{\tfrac{1}{2}} ,j}^{n+1} \Delta x} =0} \\ {\frac{V_{i,j+{\tfrac{1}{2}} }^{n+1{\tfrac{1}{2}} } -V_{i,j+{\tfrac{1}{2}} }^{n*} }{\Delta t} +\frac{{}^{y} \bar{q}_{i,j+1}^{n+{\tfrac{1}{2}} } \Delta V_{i,j+1}^{} -{}^{y} \bar{q}_{i,j}^{n+{\tfrac{1}{2}} } \Delta V_{i,j}^{} }{\bar{H}_{i,j+{\tfrac{1}{2}} }^{n+1} \Delta y} +\frac{{}^{x} \bar{q}_{i+{\tfrac{1}{2}} ,j+{\tfrac{1}{2}} }^{n+{\tfrac{1}{2}} } \Delta V_{i+{\tfrac{1}{2}} ,j+{\tfrac{1}{2}} }^{} -{}^{y} \bar{q}_{i-{\tfrac{1}{2}} ,j+{\tfrac{1}{2}} }^{n+{\tfrac{1}{2}} } \Delta V_{i-{\tfrac{1}{2}} ,j+{\tfrac{1}{2}} }^{} }{\bar{H}_{i,j+{\tfrac{1}{2}} }^{n+1} \Delta y} +...} \\ {...+\frac{\left(\eta _{i,j+1}^{n+1} +d_{i,j}^{n+1} \right)\Delta p_{i,j+1}^{n+1{\tfrac{1}{2}} } -\left(\eta _{i,j}^{n+1} -d_{i,j+1}^{n+1} \right)\Delta p_{i,j}^{n+1{\tfrac{1}{2}} } }{2\bar{H}_{i,j+{\tfrac{1}{2}} }^{n+1} \Delta x} =0} \end{array} 
\end{equation} 


\noindent The values of $\Delta U_{i+1,j}^{n*} $ are obtained from slope limited expressions. For positive flow these read:

\noindent 
\begin{equation} \label{ZEqnNum771022} 
\Delta U_{i,j}^{n*} ={\tfrac{1}{2}} \psi \left(^{{\rm u}} {\rm r}_{{\rm i+}{\tfrac{{\rm 1}}{{\rm 2}}} }^{} \right)\left(U_{i+{\tfrac{1}{2}} ,j}^{*} -U_{i-{\tfrac{1}{2}} ,j}^{n} \right){\rm \; \; \; \; \; }^{{\rm u}} {\rm r}_{{\rm i+}{\tfrac{{\rm 1}}{{\rm 2}}} }^{} =\frac{U_{i-{\tfrac{1}{2}} ,j}^{*} -U_{i-1{\tfrac{1}{2}} ,j}^{n} }{U_{i+{\tfrac{1}{2}} ,j}^{*} -U_{i-{\tfrac{1}{2}} ,j}^{n} } {\rm \; \; \; \; \; \; \; \; if\; q}_{{\rm i},j}^{{\rm n+}{\tfrac{{\rm 1}}{{\rm 2}}} } >0 
\end{equation} 


\noindent Where $\psi $ again denotes the minmod limiter. Similar expressions can be constructed for$\Delta U_{i+{\tfrac{1}{2}} ,j+{\tfrac{1}{2}} }^{} $,$\Delta V_{i,j}^{} $ and$\Delta V_{i+{\tfrac{1}{2}} ,j+{\tfrac{1}{2}} }^{} $.

\noindent 

\noindent The predictor-corrector set is second order accurate in regions where the solution is smooth, and reduces to first order accuracy near sharp gradients in the solutions to avoid unwanted oscillations. Furthermore, the method remains momentum conservative.


\paragraph{ Vertical momentum equations}

\noindent The vertical momentum equation \eqref{ZEqnNum279142} is discretized in a similar manner to the horizontal momentum equations using the McCormack scheme. In terms of the depth averaged vertical velocity the predictor step is:

\noindent 
\begin{equation} \label{ZEqnNum523996} 
\begin{array}{r} {\frac{\bar{W}_{i,j}^{*} -\bar{W}_{i,j}^{n+{\tfrac{1}{2}} } }{\Delta t} +\frac{{}^{x} q_{i+{\tfrac{1}{2}} ,j}^{n+{\tfrac{1}{2}} } \bar{W}_{i+{\tfrac{1}{2}} ,j}^{n+{\tfrac{1}{2}} } -{}^{x} q_{i-{\tfrac{1}{2}} ,j}^{n+{\tfrac{1}{2}} } \bar{W}_{i-{\tfrac{1}{2}} ,j}^{n+{\tfrac{1}{2}} } }{H_{i,j}^{n+1} \Delta x} -\frac{\bar{W}_{i,j}^{n+{\tfrac{1}{2}} } }{H_{i,j}^{n+1} } \frac{{}^{x} q_{i+{\tfrac{1}{2}} ,j}^{n+{\tfrac{1}{2}} } -{}^{x} q_{i-{\tfrac{1}{2}} ,j}^{n+{\tfrac{1}{2}} } }{\Delta x} +\frac{{}^{y} q_{i,j+{\tfrac{1}{2}} }^{n+{\tfrac{1}{2}} } \bar{W}_{i,j+{\tfrac{1}{2}} }^{n+{\tfrac{1}{2}} } -{}^{y} q_{i,j-{\tfrac{1}{2}} }^{n+{\tfrac{1}{2}} } \bar{W}_{i,j-{\tfrac{1}{2}} }^{n+{\tfrac{1}{2}} } }{H_{i,j}^{n+1} \Delta y} } \\ {-\frac{\bar{W}_{i,j}^{n+{\tfrac{1}{2}} } }{H_{i,j}^{n+1} } \frac{{}^{y} q_{i,j+{\tfrac{1}{2}} }^{n+{\tfrac{1}{2}} } -{}^{y} q_{i,j-{\tfrac{1}{2}} }^{n+{\tfrac{1}{2}} } }{\Delta y} +\frac{p_{i,j,1}^{n+{\tfrac{1}{2}} } -p_{i,j,0}^{n+{\tfrac{1}{2}} } }{H_{i,j}^{n+1} } =\frac{{}^{w} {\rm S}_{i,j}^{n+{\tfrac{1}{2}} } +{}^{w} {\rm T}_{i,j}^{n+{\tfrac{1}{2}} } }{H_{i,j}^{n+1} } } \end{array} 
\end{equation} 


\noindent The pressures are defined on the cell faces and therefore do not have to be interpolated. Furthermore, we can exactly set the dynamic pressure at the free surface $p_{i,j,1}^{n+{\tfrac{1}{2}} } $ to zero. The vertical velocities are defined on the cell faces and therefore the depth averaged velocity $\bar{W}_{i,j}^{n+{\tfrac{1}{2}} } $ needs to be expressed in terms of the bottom and surface velocities. Using a simple central approximation gives

\noindent 
\begin{equation} \label{ZEqnNum287705} 
\bar{W}_{i,j}^{n+{\tfrac{1}{2}} } ={\tfrac{1}{2}} \left(w_{i,j,1}^{n+{\tfrac{1}{2}} } +w_{i,j,0}^{n+{\tfrac{1}{2}} } \right),{\rm \; \; \; \; \; \; }\bar{W}_{i,j}^{*} ={\tfrac{1}{2}} \left(w_{i,j,1}^{*} +w_{i,j,0}^{*} \right) 
\end{equation} 


\noindent At the bottom the kinematic boundary condition is used for the vertical velocity:

\noindent 
\begin{equation} \label{ZEqnNum183135} 
w_{i,j,0}^{*} ={\tfrac{1}{2}} \left(U_{i+{\tfrac{1}{2}} ,j}^{*} +U_{i-{\tfrac{1}{2}} ,j}^{*} \right)\frac{d_{i+{\tfrac{1}{2}} ,j} -d_{i-{\tfrac{1}{2}} ,j} }{\Delta x_{i} } +{\tfrac{1}{2}} \left(V_{i,j+{\tfrac{1}{2}} }^{*} +V_{i,j-{\tfrac{1}{2}} }^{*} \right)\frac{d_{i,j+{\tfrac{1}{2}} } -d_{i,j-{\tfrac{1}{2}} } }{\Delta x_{j} }  
\end{equation} 


\noindent Horizontal interpolation of $\bar{W}_{i+{\tfrac{1}{2}} ,j}^{n+{\tfrac{1}{2}} } $ and $\bar{W}_{i,j+{\tfrac{1}{2}} }^{n+{\tfrac{1}{2}} } $ is done using first order upwind similar to \eqref{ZEqnNum559032}. The turbulent stresses are again approximated using a central scheme as

\noindent 
\begin{equation} \label{6.46)} 
\begin{array}{rcl} {{}^{w} T_{i+{\tfrac{1}{2}} ,j}^{n+{\tfrac{1}{2}} } } & {=} & {\frac{1}{\Delta x_{i} } \left[{}^{x} \bar{\nu }_{i+{\tfrac{1}{2}} ,j}^{n} {}^{x} \bar{H}_{i+{\tfrac{1}{2}} ,j}^{n+1} \frac{W_{i+1,j}^{n+{\tfrac{1}{2}} } -W_{i,j}^{n+{\tfrac{1}{2}} } }{\Delta x_{i+{\tfrac{1}{2}} } } -{}^{x} \bar{\nu }_{i-{\tfrac{1}{2}} ,j}^{n} {}^{x} \bar{H}_{i-{\tfrac{1}{2}} ,j}^{n+1} \frac{W_{i,j}^{n+{\tfrac{1}{2}} } -W_{i-1,j}^{n+{\tfrac{1}{2}} } }{\Delta x_{i+{\tfrac{1}{2}} } } \right]} \\ {} & {} & {+\frac{1}{\Delta y_{i} } \left[{}^{y} \bar{\nu }_{i,j+{\tfrac{1}{2}} }^{n} {}^{y} \bar{H}_{i,j+{\tfrac{1}{2}} }^{n+1} \frac{W_{i,j+1}^{n+{\tfrac{1}{2}} } -W_{i,j}^{n+{\tfrac{1}{2}} } }{\Delta x_{j+{\tfrac{1}{2}} } } -{}^{y} \bar{\nu }_{i,j-{\tfrac{1}{2}} }^{n} {}^{y} \bar{H}_{i,j-{\tfrac{1}{2}} }^{n+1} \frac{W_{i,j}^{n+{\tfrac{1}{2}} } -W_{i,j-1}^{n+{\tfrac{1}{2}} } }{\Delta y_{j+{\tfrac{1}{2}} } } \right]} \end{array} 
\end{equation} 


\noindent Thus combining\eqref{ZEqnNum523996}, \eqref{ZEqnNum287705} and \eqref{ZEqnNum183135} explicit expressions for $w_{i,j,1}^{*} $ and $w_{i,j,0}^{*} $ are obtained. 

\noindent The predicted values are again corrected using a variant of the McCormack scheme and including the pressure difference implicitly gives the corrector step:

\noindent 
\begin{equation} \label{ZEqnNum858516} 
\frac{\bar{W}_{i,j}^{n+1{\tfrac{1}{2}} } -\bar{W}_{i,j,1}^{*} }{\Delta t} +\frac{{}^{x} q_{i+{\tfrac{1}{2}} ,j}^{n+{\tfrac{1}{2}} } \Delta \bar{W}_{i+{\tfrac{1}{2}} ,j}^{} -{}^{x} q_{i-{\tfrac{1}{2}} ,j}^{n+{\tfrac{1}{2}} } \Delta \bar{W}_{i-{\tfrac{1}{2}} ,j}^{} }{H_{i,j}^{n+1} \Delta x} +\frac{{}^{y} q_{i,j+{\tfrac{1}{2}} }^{n+{\tfrac{1}{2}} } \Delta \bar{W}_{i,j+{\tfrac{1}{2}} }^{} -{}^{y} q_{i,j-{\tfrac{1}{2}} }^{n+{\tfrac{1}{2}} } \Delta \bar{W}_{i,j-{\tfrac{1}{2}} }^{} }{H_{i,j}^{n+1} \Delta y} -\frac{\Delta p_{i,j}^{n+1{\tfrac{1}{2}} } }{H_{i,j}^{n+1} } =0 
\end{equation} 
Where $\Delta \bar{W}_{i+{\tfrac{1}{2}} ,j}^{} $ and $\Delta \bar{W}_{i,j+{\tfrac{1}{2}} }^{} $ are obtained using relations similar to \eqref{ZEqnNum771022}. Note that similar to \eqref{ZEqnNum287705}  $\bar{W}_{i,j}^{n+1{\tfrac{1}{2}} } ={\tfrac{1}{2}} \left(w_{i,j.1}^{n+1{\tfrac{1}{2}} } +w_{i,j,0}^{n+1{\tfrac{1}{2}} } \right)$and again the kinematic boundary conditions is substituted for $w_{i,j,0}^{n+1{\tfrac{1}{2}} } $.

\noindent 

\noindent The discrete vertical momentum balance of \eqref{ZEqnNum523996} and \eqref{ZEqnNum858516} looks very different from the relations found in Zijlema and Stelling (2005), Zijlema and Stelling (2008)  and Smit (2008). This is mainly due to the application of the McCormack scheme for the advection. The discretisation of the pressure term is numerically fully equivalent to either the Keller box scheme as used in Zijlema and Stelling (2005), Zijlema and Stelling (2008) or the Hermetian relation used in Smit (2008).

\noindent 


\end{document}

% == UNREGISTERED! == GrindEQ Word-to-LaTeX 2012 ==

